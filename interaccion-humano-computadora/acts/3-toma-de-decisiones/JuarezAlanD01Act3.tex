\documentclass[11pt, a4paper]{article} % Formato

% Language and font encodings
\usepackage[spanish]{babel}
\usepackage[utf8]{inputenc}
\usepackage[T1]{fontenc}
\usepackage{times} % Times New Roman

%% Sets page size and margins
\usepackage[margin=2.5cm, includefoot]{geometry}
%\setlength{\columnsep}{0.17in} % page columns separation

%% Useful packages
\usepackage{amsmath}
\usepackage{array} % <-- add this line for m{} column type
\usepackage[hidelinks]{hyperref} % hyperlinks support
\usepackage{graphicx} % images support
%\usepackage{listings} % codeblock support
%\usepackage{smartdiagram} % diagrams support
\usepackage[most]{tcolorbox} % callouts support
%\usepackage[colorinlistoftodos]{todonotes}
\usepackage[dvipsnames, table, xcdraw]{xcolor} % Tables support
%\usepackage{zed-csp} % cchemas support

%% Formating
\usepackage{authblk} % to add authors in maketitle
%\usepackage{blindtext} % to gen filler text
\usepackage[figurename=Fig.]{caption} % to change prefix of the image caption

%\usepackage{apacite}
\usepackage{cite} % useful to compress multiple quotations into a single entry
\usepackage{enumitem}
\usepackage{fancyhdr} % to set page style
\usepackage{indentfirst}
%\usepackage{natbib}
\usepackage{parskip} % remove first line tabulation
\usepackage{setspace}
%\usepackage{titlesec}
%\usepackage{titling} % to config maketitle

%% Variables
% Main images
\newcommand{\logoUdg}{logo-udg.jpg}
\newcommand{\logoCucei}{logo-cucei.jpg}
\newcommand{\newAmazonSection}{./img/productos-nuevos.jpg}

% School data
\newcommand{\universidad}{Universidad de Guadalajara}
\newcommand{\cede}{Centro Universitario de Ciencias Exactas e Ingenierías}

% Subject data
\newcommand{\materia}{Interacción Humano Computadora}
\newcommand{\carrera}{Ingeniería en Computación}
\newcommand{\division}{División de Tecnologías para la Integración CiberHumana}
\newcommand{\theTitle}{3. Comprender la Toma de Deciciones Humanas}
\newcommand{\profesor}{José Luis David Bonilla Carranza}
\newcommand{\seccion}{D01}
\newcommand{\nrc}{209754}
\newcommand{\clave}{IL367}
\newcommand{\startDate}{20 de septiembre de 2024}

% Author data
\newcommand{\theAuthor}{Juárez Rubio Alan Yahir}
\newcommand{\theAuthorCode}{218517809}
\newcommand{\theAuthorMail}{alan.juarez5178@alumnos.udg.mx}

%% Declaration
\date{}
\graphicspath{ {../../../img/} }
\addto\captionsspanish{\renewcommand{\contentsname}{Índice}}
\renewcommand{\lstlistingname}{Código} % to change prefix of the code caption
\renewcommand{\lstlistlistingname}{Índice de códigos} % to change listings index title

%% Styles

% Color declaration
\definecolor{greenPortada}{HTML}{69A84F}
\definecolor{LightGray}{gray}{0.9}
\definecolor{codegreen}{rgb}{0, 0.6, 0}
\definecolor{codegray}{rgb}{0.5, 0.5, 0.5}
\definecolor{codepurple}{rgb}{0.58, 0, 0.82}
\definecolor{backcolour}{rgb}{0.95, 0.95, 0.92}

% Hyperlinks
\hypersetup{
    colorlinks=true,
    linkcolor=black,
    filecolor=greenPortada,
    urlcolor=greenPortada,
    pdfpagemode=FullScreen,
}

\urlstyle{same}

% Codeblocks
\lstdefinestyle{mystyle}{
	backgroundcolor=\color{backcolour},
	commentstyle=\color{codegreen},
	keywordstyle=\color{magenta},
	numberstyle=\tiny\color{codegray},
	stringstyle=\color{codepurple},
	basicstyle=\ttfamily\footnotesize,
	breakatwhitespace=false,
	breaklines=true,
	captionpos=b,
	keepspaces=true,
	numbers=left,
	numbersep=5pt,
	showspaces=false,
	showstringspaces=false,
	showtabs=false,
	tabsize=2
}

% Tables
\let\oldtabular\tabular
\renewcommand{\tabular}{\small\oldtabular}
\renewcommand{\arraystretch}{1.2} % <-- Adjust vertical spacing
\addto\captionsspanish{\renewcommand{\tablename}{Tabla}}

\lstset{style=mystyle}

%% Spacing
\newcommand{\nl}{\par
\vspace{0.4cm}}
\renewcommand{\baselinestretch}{1.5} % Espaciado de línea anterior
\setlength{\parskip}{6pt} % Espaciado de línea anterior
\setlength{\parindent}{0pt} % Sangría

% Header and footer
\pagestyle{fancy}
\fancyhf{}
\renewcommand{\headrulewidth}{3pt}
\renewcommand{\headrule}{\hbox to\headwidth{\color{greenPortada}\leaders\hrule height \headrulewidth\hfill}}
\setlength{\headheight}{50pt} % Ajuste necesario para evitar warnings

% Header
\pagestyle{fancy}
\fancyhf{}
\lhead{
	\begin{minipage}[c][2cm][c]{1.3cm}
		\begin{flushleft}
			\includegraphics[width=5cm, height=1.4cm, keepaspectratio]{\logoUdg}
		\end{flushleft}
	\end{minipage}
	\begin{minipage}[c][2cm][c]{0.5\textwidth} % Adjust the height as needed
		\begin{flushleft}
			{\materia}
		\end{flushleft}
	\end{minipage}
}

\rhead{
	\begin{minipage}[c][2cm][c]{0.4\textwidth} % Adjust the height as needed
		\begin{flushright}
			{\theTitle}
		\end{flushright}
	\end{minipage}
	\begin{minipage}[c][2cm][c]{1.3cm}
		\begin{flushright}
			\includegraphics[width=5cm, height=1.4cm, keepaspectratio]{\logoCucei}
		\end{flushright}
	\end{minipage}
}

% Footer
\fancyfoot{}
\lfoot{\small\materia}
\cfoot{\thepage} % Paginación
\rfoot{\small Curso impartido por \profesor}

%% Title

\title{\fontsize{24}{28.8}\selectfont \theTitle}
\author{\theAuthor}

\affil{}


\begin{document}
    \setstretch{1} % Interlineado

    \begin{titlepage}
        \newgeometry{margin=2.5cm, left=3cm, right=3cm} % change margin
        \centering
        %\vspace*{-2cm}
        {\huge\textbf{\universidad}}\par
        \vspace{0.6cm}
        {\LARGE{\cede}}
        \vfill

        \begin{figure}[h]
            \begin{minipage}[t]{0.45\textwidth}
                \centering
                \includegraphics[width=130px, height=160px, keepaspectratio]{\logoUdg}
            \end{minipage}
            \hfill
            \begin{minipage}[t]{0.45\textwidth}
                \centering
                \includegraphics[width=130px, height=160px, keepaspectratio]{\logoCucei}
            \end{minipage}
        \end{figure}
        \vfill

        \Large{
			\division\vfill
			\textbf{\carrera}\vfill
			\textbf{\materia}\par\vspace{3pt}
			\seccion\ - \clave\ - \nrc\vfill
		}

        {\LARGE{\textbf{\theTitle}}}
        \vfill

		\begin{figure}[h]
			\centering
			\begin{minipage}[t]{0.75\textwidth}
				{\Large
					\textbf{Profesor}: \profesor\nl
					\textbf{Alumno}: \theAuthor\nl
					\textbf{Código}: \theAuthorCode\nl
					\textbf{Correo}: \theAuthorMail }
			\end{minipage}
		\end{figure}
		\vfill

        \begin{tcolorbox}
            [colback=red!5!white, colframe=red!75!black]
            \centering
            Este documento contiene información sensible.\\
			No debería ser impreso o compartido con terceras entidades.
        \end{tcolorbox}
        \vfill
        {\large \startDate}\par
    \end{titlepage}

    \restoregeometry % end changed margin

    %% Indexes
    \clearpage
    \tableofcontents

    \clearpage
    \listoffigures

    \clearpage
    %\listoftables

    %\clearpage
    %\lstlistoflistings

    %% Main Title
    \clearpage
    \vspace*{6pt}
	\centerline{\textbf{\huge \theTitle}}
    \vspace*{8pt}

    %% Content

	\section{Análisis de Sesgos Cognitivos en Interfaces UX}

	En el diseño de interfaces de usuario (UX), los sesgos cognitivos afectan la
	toma de decisiones de los usuarios. A continuación, se describen tres sesgos
	cognitivos comunes y cómo influyen en las interacciones con interfaces
	digitales:

	\subsection{Sesgo de Anclaje}

	El \textbf{sesgo de anclaje} se refiere a la tendencia a depender en gran
	medida de la primera información que se recibe (el ``ancla'') al tomar
	decisiones. En interfaces UX, la primera impresión de un precio o
	característica puede influir en cómo los usuarios perciben el valor o la
	relevancia de un producto o servicio.

	\begin{itemize}
		\item \textbf{Ejemplo}: En una tienda en línea, si un producto muestra un
			precio original alto junto a un precio rebajado, los usuarios pueden percibir
			que el descuento es significativo, incluso si no lo es.
	\end{itemize}

	\subsection{Sesgo de Confirmación}

	El \textbf{sesgo de confirmación} ocurre cuando los usuarios buscan,
	interpretan y recuerdan información que confirme sus creencias previas, ignorando
	datos que podrían contradecirlas. En el contexto UX, esto puede reforzarse cuando
	las plataformas recomiendan contenido basado en interacciones anteriores del usuario.

	\begin{itemize}
		\item \textbf{Ejemplo}: En una plataforma de streaming, las recomendaciones
			personalizadas tienden a reforzar los gustos y preferencias previas, limitando
			la exploración de nuevos géneros.
	\end{itemize}

	\subsection{Efecto de Recencia}

	El \textbf{efecto de recencia} describe la tendencia de las personas a
	recordar mejor la información que se presenta al final de una serie de
	elementos. En UX, las últimas opciones o productos mostrados pueden tener un
	mayor impacto en la decisión del usuario.

	\begin{itemize}
		\item \textbf{Ejemplo}: En una lista de productos en una tienda en línea,
			los productos más recientes o visibles al final de la página pueden
			influir más en la decisión de compra del usuario.
	\end{itemize}

	\section{Análisis de un Caso de Estudio: Amazon}

	Amazon es una de las plataformas de comercio electrónico más populares del mundo.
	Utiliza diversas técnicas de UX que influyen en la toma de decisiones de los
	usuarios, y en este análisis exploraremos cómo los sesgos cognitivos (anclaje,
	confirmación y recencia) afectan la experiencia de compra.

	\subsection{Sesgo de Anclaje en Amazon}

	Amazon utiliza precios ``rebajados'' en gran parte de sus productos, mostrando
	el precio original tachado junto al precio con descuento. Esta estrategia de anclaje
	hace que los usuarios perciban que están obteniendo una oferta especial, aun
	cuando el descuento no sea significativo. El primer precio que ven (el
	original) funciona como ``ancla'' para su percepción del valor del producto.

	\subsection{Sesgo de Confirmación en Amazon}

	Las recomendaciones de productos en Amazon están personalizadas en función de
	compras anteriores o productos vistos. Esto refuerza las creencias previas del
	usuario sobre sus preferencias, lo que puede limitar la exposición a nuevas
	categorías de productos o marcas.

	\subsection{Efecto de Recencia en Amazon}

	En las páginas de resultados de búsqueda o en la lista de productos recomendados,
	los productos más visibles al final de la página tienden a influir más en las
	decisiones de compra, especialmente si se destacan como ``Más vendidos'' o ``Populares''.
	El efecto de recencia aumenta la probabilidad de que los usuarios elijan estos
	productos.

	\section{Propuesta de Mejoras}

	\subsection{Área 1: Presentación de Precios (Sesgo de Anclaje)}

	\begin{itemize}
		\item \textbf{Problema identificado}: Amazon presenta el precio original de un
			producto como ancla, lo que afecta la percepción del usuario sobre la ``oferta''
			que está recibiendo.

		\item \textbf{Propuesta de mejora}: Se puede ofrecer una comparativa de precios
			con competidores o mostrar más detalles sobre la fluctuación de precios en
			el tiempo, para que el usuario no se quede anclado solo en el descuento, sino
			que tenga una visión más completa del valor del producto.
	\end{itemize}

	\subsection{Área 2: Recomendaciones Personalizadas (Sesgo de Confirmación)}

	\begin{itemize}
		\item \textbf{Problema identificado}: Las recomendaciones basadas en compras
			anteriores refuerzan el sesgo de confirmación, limitando la exploración de
			nuevas opciones.

		\item \textbf{Propuesta de mejora}: Incluir una sección de ``Descubre algo nuevo''
			o ``Sorpresas'', donde se ofrezcan productos de categorías que el usuario
			no ha explorado previamente. Esto incentivaría una mayor variedad de interacciones
			y reduciría el impacto del sesgo de confirmación.
	\end{itemize}

	\subsection{Área 3: Listado de Productos (Efecto de Recencia)}

	\begin{itemize}
		\item \textbf{Problema identificado}: Los productos al final de una lista de
			búsqueda o recomendaciones tienden a ser más influyentes debido al efecto de
			recencia.

		\item \textbf{Propuesta de mejora}: Implementar un diseño de ``scroll continuo''
			que no siga un orden lineal, o alternar el orden de los productos cada vez
			que el usuario recargue la página, para evitar que siempre los últimos elementos
			sean los más influyentes.
	\end{itemize}


	\section{Prototipo de la Solución}

	\begin{figure}[h]
		\centering
		\includegraphics[width=\textwidth]{\newAmazonSection}
		\caption{Sección de compras para descubrir nuevos productos para eliminar el \textbf{sesgo de confirmación}}
	\end{figure}

	\section{Conclusión}

	Los sesgos cognitivos, como el sesgo de anclaje, el sesgo de confirmación y el
	efecto de recencia, tienen un impacto significativo en cómo los usuarios interactúan
	con las interfaces de usuario. Al entender y mitigar estos sesgos, se puede mejorar
	la experiencia del usuario y ofrecer una interacción más justa y equilibrada.

	%% References

	\nocite{*} % to include uncited references of .bib file

	\clearpage
	\bibliographystyle{ieeetr}

	% Generated from .bib file
	\bibliography{ref}

\end{document}
