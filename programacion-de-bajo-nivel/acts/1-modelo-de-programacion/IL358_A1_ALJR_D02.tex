\documentclass[11pt, a4paper]{article} % Formato

% Language and font encodings
\usepackage[spanish]{babel}
\usepackage[utf8]{inputenc}
\usepackage[T1]{fontenc}
\usepackage{times} % Times New Roman

%% Sets page size and margins
\usepackage[margin=2.5cm, includefoot]{geometry}
%\setlength{\columnsep}{0.17in} % page columns separation

%% Useful packages
\usepackage{amsmath}
\usepackage{array} % <-- add this line for m{} column type
\usepackage[hidelinks]{hyperref} % hyperlinks support
\usepackage{graphicx} % images support
%\usepackage{listings} % codeblock support
%\usepackage{smartdiagram} % diagrams support
\usepackage[most]{tcolorbox} % callouts support
%\usepackage[colorinlistoftodos]{todonotes}
\usepackage[dvipsnames, table, xcdraw]{xcolor} % Tables support
%\usepackage{zed-csp} % cchemas support

%% Formating
\usepackage{authblk} % to add authors in maketitle
%\usepackage{blindtext} % to gen filler text
\usepackage[figurename=Fig.]{caption} % to change prefix of the image caption

%\usepackage{apacite}
\usepackage{cite} % useful to compress multiple quotations into a single entry
\usepackage{enumitem}
\usepackage{fancyhdr} % to set page style
\usepackage{indentfirst}
%\usepackage{natbib}
\usepackage{parskip} % remove first line tabulation
\usepackage{setspace}
%\usepackage{titlesec}
%\usepackage{titling} % to config maketitle

%% Variables
% Main images
\newcommand{\logoUdg}{logo-udg.jpg}
\newcommand{\logoCucei}{logo-cucei.jpg}
\newcommand{\newAmazonSection}{./img/productos-nuevos.jpg}

% School data
\newcommand{\universidad}{Universidad de Guadalajara}
\newcommand{\cede}{Centro Universitario de Ciencias Exactas e Ingenierías}

% Subject data
\newcommand{\materia}{Interacción Humano Computadora}
\newcommand{\carrera}{Ingeniería en Computación}
\newcommand{\division}{División de Tecnologías para la Integración CiberHumana}
\newcommand{\theTitle}{3. Comprender la Toma de Deciciones Humanas}
\newcommand{\profesor}{José Luis David Bonilla Carranza}
\newcommand{\seccion}{D01}
\newcommand{\nrc}{209754}
\newcommand{\clave}{IL367}
\newcommand{\startDate}{20 de septiembre de 2024}

% Author data
\newcommand{\theAuthor}{Juárez Rubio Alan Yahir}
\newcommand{\theAuthorCode}{218517809}
\newcommand{\theAuthorMail}{alan.juarez5178@alumnos.udg.mx}

%% Declaration
\date{}
\graphicspath{ {../../../img/} }
\addto\captionsspanish{\renewcommand{\contentsname}{Índice}}
\renewcommand{\lstlistingname}{Código} % to change prefix of the code caption
\renewcommand{\lstlistlistingname}{Índice de códigos} % to change listings index title

%% Styles

% Color declaration
\definecolor{greenPortada}{HTML}{69A84F}
\definecolor{LightGray}{gray}{0.9}
\definecolor{codegreen}{rgb}{0, 0.6, 0}
\definecolor{codegray}{rgb}{0.5, 0.5, 0.5}
\definecolor{codepurple}{rgb}{0.58, 0, 0.82}
\definecolor{backcolour}{rgb}{0.95, 0.95, 0.92}

% Hyperlinks
\hypersetup{
    colorlinks=true,
    linkcolor=black,
    filecolor=greenPortada,
    urlcolor=greenPortada,
    pdfpagemode=FullScreen,
}

\urlstyle{same}

% Codeblocks
\lstdefinestyle{mystyle}{
	backgroundcolor=\color{backcolour},
	commentstyle=\color{codegreen},
	keywordstyle=\color{magenta},
	numberstyle=\tiny\color{codegray},
	stringstyle=\color{codepurple},
	basicstyle=\ttfamily\footnotesize,
	breakatwhitespace=false,
	breaklines=true,
	captionpos=b,
	keepspaces=true,
	numbers=left,
	numbersep=5pt,
	showspaces=false,
	showstringspaces=false,
	showtabs=false,
	tabsize=2
}

% Tables
\let\oldtabular\tabular
\renewcommand{\tabular}{\small\oldtabular}
\renewcommand{\arraystretch}{1.2} % <-- Adjust vertical spacing
\addto\captionsspanish{\renewcommand{\tablename}{Tabla}}

\lstset{style=mystyle}

%% Spacing
\newcommand{\nl}{\par
\vspace{0.4cm}}
\renewcommand{\baselinestretch}{1.5} % Espaciado de línea anterior
\setlength{\parskip}{6pt} % Espaciado de línea anterior
\setlength{\parindent}{0pt} % Sangría

% Header and footer
\pagestyle{fancy}
\fancyhf{}
\renewcommand{\headrulewidth}{3pt}
\renewcommand{\headrule}{\hbox to\headwidth{\color{greenPortada}\leaders\hrule height \headrulewidth\hfill}}
\setlength{\headheight}{50pt} % Ajuste necesario para evitar warnings

% Header
\pagestyle{fancy}
\fancyhf{}
\lhead{
	\begin{minipage}[c][2cm][c]{1.3cm}
		\begin{flushleft}
			\includegraphics[width=5cm, height=1.4cm, keepaspectratio]{\logoUdg}
		\end{flushleft}
	\end{minipage}
	\begin{minipage}[c][2cm][c]{0.5\textwidth} % Adjust the height as needed
		\begin{flushleft}
			{\materia}
		\end{flushleft}
	\end{minipage}
}

\rhead{
	\begin{minipage}[c][2cm][c]{0.4\textwidth} % Adjust the height as needed
		\begin{flushright}
			{\theTitle}
		\end{flushright}
	\end{minipage}
	\begin{minipage}[c][2cm][c]{1.3cm}
		\begin{flushright}
			\includegraphics[width=5cm, height=1.4cm, keepaspectratio]{\logoCucei}
		\end{flushright}
	\end{minipage}
}

% Footer
\fancyfoot{}
\lfoot{\small\materia}
\cfoot{\thepage} % Paginación
\rfoot{\small Curso impartido por \profesor}

%% Title

\title{\fontsize{24}{28.8}\selectfont \theTitle}
\author{\theAuthor}

\affil{}


\begin{document}
	\setstretch{1} % Interlineado

	\begin{titlepage}
		\newgeometry{margin=2.5cm, left=3cm, right=3cm} % change margin
		\centering
		%\vspace*{-2cm}
		{\huge\textbf{\universidad}}\par
		\vspace{0.6cm}
		{\LARGE{\cede}}
		\vfill

		\begin{figure}[h]
			\begin{minipage}[t]{0.45\textwidth}
				\centering
				\includegraphics[width=130px, height=160px, keepaspectratio]{\logoUdg}
			\end{minipage}
			\hfill
			\begin{minipage}[t]{0.45\textwidth}
				\centering
				\includegraphics[width=130px, height=160px, keepaspectratio]{\logoCucei}
			\end{minipage}
		\end{figure}
		\vfill

		\Large{ \division\vfill \textbf{\carrera}\vfill \textbf{\materia}\par\vspace{3pt} \seccion\ - \clave\ - \nrc\vfill }

		\begin{figure}[h]
			\centering
			\begin{minipage}[t]{0.75\textwidth}
				{\Large \textbf{Profesor}: \profesor\nl \textbf{Alumno}: \theAuthor\nl \textbf{Código}: \theAuthorCode\nl \textbf{Correo}: \theAuthorMail }
			\end{minipage}
		\end{figure}
		\vfill
		{\LARGE{\textbf{\theTitle}}}
		\vfill

		\begin{tcolorbox}
			[colback=red!5!white, colframe=red!75!black]
			\centering
			Este documento contiene información sensible.\\ No debería ser impreso o
			compartido con terceras entidades.
		\end{tcolorbox}
		\vfill
		{\large \startDate}\par
	\end{titlepage}

	\restoregeometry % end changed margin

	%% Indexes
	\clearpage
	\tableofcontents

	%\clearpage
	%\listoffigures

	%\clearpage
	%\listoftables

	%\clearpage
	%\lstlistoflistings

	%% Main Title
	\clearpage
	\vspace*{6pt}
	\begin{center}
		{\textbf{\huge \theTitle}}
	\end{center}
	\vspace*{8pt}

	%% Content

	El estudio de los microprocesadores Intel, desde el icónico 8086 hasta el avanzado
	Pentium 4, revela la evolución impresionante de la arquitectura de los procesadores a lo
	largo de las décadas. En el núcleo de cada microprocesador se encuentra el modelo de
	programación, que define cómo interactúan los registros, las banderas y las
	instrucciones para llevar a cabo tareas computacionales.

	Comprender estos elementos es esencial para cualquier persona interesada en la
	arquitectura de computadores y el desarrollo de software a bajo nivel. A lo largo de este
	documento, se examinará cómo cada registro y bandera ha evolucionado y cómo su
	funcionalidad ha impactado el rendimiento y la capacidad de los microprocesadores.

	\clearpage
	\section{Registros}

	Los registros son pequeñas unidades de almacenamiento de datos dentro del microprocesador
	que se utilizan para realizar operaciones rápidas y eficientes. A diferencia de la
	memoria RAM, los registros son mucho más rápidos y están directamente integrados en la
	CPU, lo que les permite acceder y manipular datos con una latencia mínima.

	En los microprocesadores Intel, desde el 8086 hasta el Pentium 4, los registros juegan
	un papel fundamental en la ejecución de instrucciones, la manipulación de datos y el control
	del flujo de los programas. Existen varios tipos de registros, cada uno con funciones específicas

	\subsection{Registros de Propósito General}

	\begin{itemize}
		\item \textbf{AX (Acumulador)}: Es utilizado para operaciones aritméticas.

			En los microprocesadores 80386 y superiores, el registro EAX puede también
			guardar la dirección de desplazamiento de una posición en el sistema de
			memoria.

		\item \textbf{BX (Índice Base)}: Guarda la dirección de desplazamiento de una posición
			en el sistema de memoria.

			En el 80386 y superiores, EBX también puede direccionar datos de la memoria.

		\item \textbf{CX (Contador)}: Guarda la cuenta de varias instrucciones (e.g. bucles).
			También cuenta los desplazamientos en operaciones de rotación y desplazamiento
			de bits.

			En el 80386 y superiores, el registro ECX también puede guardar la dirección
			de desplazamiento de datos de la memoria.

		\item \textbf{DX (Datos)}: Se usa en operaciones de entrada/salida y también en algunas
			operaciones aritméticas, especialmente en multiplicaciones y divisiones donde
			se almacena el resultado o los residuos.

			En el 80386 y superiores, este registro también puede direccionar datos de la memoria.
	\end{itemize}

	\subsubsection{Apuntadores e Índices}

	\begin{itemize}
		\item \textbf{BP (Apuntador de la Base)}: Usado principalmente para acceder a los datos
			en la pila durante la ejecución de una subrutina. Apunta a una posición de
			memoria en todas las versiones del microprocesador para las transferencias de
			datos de memoria.

		\item \textbf{DI (Índice de Destino)}: Direcciona datos de destino de cadena para las
			instrucciones de cadenas. También funciona como un registro de propósito
			general de 32 y 16 bits.

		\item \textbf{SI (Índice de Origen)}: Es utilizado para direccionar datos de cadena
			de origen para las instrucciones de cadenas. Al igual que DI, también funciona
			como un registro de propósito general de 32 y 16 bits.
	\end{itemize}

	\subsection{Registros de Propósito Especial}

	\begin{itemize}
		\item \textbf{IP (Apuntador de Instrucciones)}: Almacena la dirección de la siguiente
			instrucción a ejecutar. Este puede ser modificado mediante un salto (jump) o
			una llamada (call).

			Este registro es IP cuando el microprocesador opera en modo real y 32 bits
			cuando el 80386 y superiores operan en modo protegido.

		\item \textbf{SP (Apuntador de Pila)}: Apunta a la cima de la pila. Se usa junto
			con SS para la gestión de la pila.

		\item \textbf{FLAGS (Banderas)}: Contienen banderas que indican el estado del procesador
			y el resultado de las operaciones aritméticas.

			Los microprocesadores 80386 y superiores contienen un registro EFLAG.
	\end{itemize}

	\subsection{Registros de Segmento}

	\begin{itemize}
		\item \textbf{CS (Código)}: Define la dirección base del segmento de memoria que contienen
			el ejecutable.

			El segmento de código está limitado a 64 KB de memoria en los microprocesadores
			del 8088 al 80286, ya 4 GB en los 80386 y superiores cuando operan en modo
			protegido.

		\item \textbf{DS (Datos)}: Contiene la sección en memoria que contiene los datos utilizados
			por un programa. Se accede a los datos en el segmento mediante una dirección
			de desplazamiento o el contenido.

		\item \textbf{ES (Extra)}: Segmento de datos adicional que puede ser tuilizado para
			operaciones de cadena y otras operaciones que requieren un segmento extra de datos.

		\item \textbf{FS y GS}: Estos son registros de segmento suplementario, disponibles
			en los microprocesadores del 80386 al Pentium 4 para que los programas puedan
			acceder a dos segmentos de memoria adicionales.

		\item \textbf{SS (Pila)}: Contiene la dirección base del segmento de memoria que define
			la pila, utilizada para almacenar direcciones de retorno, parámetros de subrutinas
			y otros datos temporales.
	\end{itemize}

	\clearpage
	\section{Bits de Registro de Bandera}

	Las banderas sonn bits de un registro especial llamado registro de estado o registro de
	banderas. Estos bits indican el estado o resultado de diversas operaciones aritméticas
	y lógicas realizadas por la unidad central de procesamiento (CPU). Las banderas permiten
	que el procesador tome decisiones basadas en esos resultados, controlando así el flujo
	de un programa.

	\begin{figure}[h]
		\centering
		\includegraphics[width=\textwidth]{\flags}
		\caption{El conteo de los registros EFLAG y FLAG para la familia completa de
		microprocesadores 80x86 y Pentium}
	\end{figure}

	\subsection{8086/8088/80186/80188 en adelante}

	\begin{itemize}
		\item \textbf{C (Acarreo)}: Este guarda el valor del acarreo después de la suma, o
			la sustracción después de la resta. La bandera de acarreo también indica condiciones
			de error, según lo indiquen algunos programas y procedimientos.

		\item \textbf{P (Paridad)}: Indica si el número de bits de 1 en el resultado de la
			última operación es par (PF = 1) o impar (PF = 0).

		\item \textbf{A (Acarreo Auxiliar)}: Guarda el acarreo (medio acarreo) después de la
			suma, o la sustracción después de la resta entre las posiciones de bit 3 y 4
			del resultado. Este bit de bandera altamente especializado lo comprueban las instrucciones
			DAA y DAS para ajustar el valor de AL después de una suma o resta BCD.

		\item \textbf{Z (Cero)}: Indica que el resultado de una operación aritmética o lógica
			es cero. Si Z = 1, el resultado es cero; si Z = 0, el resultado no es cero.

		\item \textbf{S (Signo)}: Guarda el signo aritmético del resultado después de la ejecución
			de una instrucción aritmética o lógica. Si S = 1, el bit de signo está activado
			o es negativo; si S = 0, el bit de signo está desactivo o es positivo.

		\item \textbf{T (Trampa)}: Habilita el atrapamiento a través de una característica
			de depuración intregrada en el chip. Si la bandera T está habilitada (1), el microprocesador
			interrumple el flujo del proggrama basándose en las condiciones indicadas por los
			registros de depuración y los registros de control. Si la bandera T es un 0
			lógico, se deshabilita la característica de atrapamiento (depuración).

		\item \textbf{I (Interrupción)}: La bandera de interrupción controla la operación de
			la terminal de entrada INTR (petición de interrupción). Si I = 1, se habilita la
			terminal INTR; si I = 0, se deshabilita la terminal INTR. El estado del bit de
			bandera I se controla mediante las instrucciones STI (establecer bandera I) y
			CLI (borrar bandera I).

		\item \textbf{D (Dirección)}: La bandera de dirección selecciona el modo de incremento
			o de decremento para los registros DI y/o SI durante las instrucciones de cadena.
			Si D = 1, los registros se decrementan automáticamente; si D = 0, los registros
			se incrementan automáticamente. La bandera D se establece con la instrucción
			STD (establecer dirección) y se borra con la instrucción CLD (borrar dirección).

		\item \textbf{O (Desbordamiento)}: Un desbordamiento ocurre cuando se suman o restan
			números con signo. Un desbordamiento indica que el resultado ha excedido la
			capacidad de la máquina. En las operaciones sin signo se ignora esta bandera.
	\end{itemize}

	\subsection{80286 en adelante}

	\begin{itemize}
		\item \textbf{IOPL (Nivel de Privilegio de E/S)}: se utiliza en operación de modo protegido
			para seleccionar el nivel de privilegio para los dispositivos de E/S. Si el
			nivel de privilegio actual es mayor o de más confianza que el IOPL, la operación
			de E/S se ejecuta sin impedimento. Si el IOPL es menor que el nivel de
			privilegio actual se produce una interrupción, haciendo que se suspenda la ejecución.
			Hay que tener en cuenta que un IOPL de 00 es el más alto o de mayor confianza,
			y un IOPL de 11 es el más bajo o de menor confianza.

		\item \textbf{NT (Tarea Anidada)}: indica que la tarea actual está anidada dentro de
			otra tarea en operación de modo protegido. Esta bandera se establece cuando la
			tarea se anida mediante software.
	\end{itemize}

	\subsection{80386/8986DX en adelante}

	\begin{itemize}
		\item \textbf{RF (Continuación)}: La bandera de continuación se utiliza con la depuración
			para controlar la continuación de la ejecución después de la siguiente instrucción.

		\item \textbf{VM (Modo Virtual)}: El bit de bandera VM selecciona la operación en modo
			virtual en un sistema de modo protegido. Un sistema de modo virtual permite
			que coexistan varias particiones de memoria de DOS de 1 Mbyte de longitud en
			el sistema de memoria. En esencia, esto permite al programa del sistema
			ejecutar varios programas de DOS. VM se utiliza para simular el DOS en el entorno
			moderno Windows.
	\end{itemize}

	\subsection{80486X}

	\begin{itemize}
		\item \textbf{AC (Comprobación de Alineación)}: se activa si se direcciona una palabra
			o doble palabra en un límite que no sea de palabra o de doble palabra. Sólo el
			microprocesador 80486SX contiene el bit de comprobación de alineación que es
			utilizado para sincronización principalmente por el coprocesador numérico
			80487SX que lo acompaña.
	\end{itemize}

	\subsection{Pentium/Pentium 4 en adelante}

	\begin{itemize}
		\item \textbf{VIF (Interrupción Virtual)}: Es una copia del bit de bandera de interrupción
			disponible para los microprocesadores del Pentium al Pentium 4.

		\item \textbf{VIP (Interrupción Virtual Pendiente)}: Proporciona información sobre
			una interrupción en modo virtual para los microprocesadores del Pentium al Pentium
			4. Se utiliza en entornos multitarea para proporcionar banderas de interrupción
			virtual al sistema operativo, además de información de interrupciones
			pendientes.

		\item \textbf{ID (Identificación)}: ndica que los microprocesadores del Pentium al
			Pentium 4 soportan la instrucción CPUID. Esta instrucción proporciona
			información al sistema sobre el microprocesador Pentium, como su número de versión
			y el fabricante.
	\end{itemize}

	\clearpage

	El recorrido a través de los registros y banderas de los microprocesadores Intel, desde
	el 8086 hasta el Pentium 4, permite apreciar la complejidad y la sofisticación
	alcanzadas en el diseño de la arquitectura x86. A lo largo de los años, Intel ha
	refinado y ampliado las capacidades de sus procesadores, adaptando y mejorando el modelo
	de programación para satisfacer las demandas crecientes de la industria de la computación.

	La evolución de los registros y banderas no solo refleja avances tecnológicos, sino
	también un compromiso constante con la compatibilidad y la optimización del
	rendimiento. Comprender estos elementos fundamentales proporciona una base sólida para
	explorar temas más avanzados en arquitectura de computadoras y programación a bajo nivel,
	y es un testimonio del progreso continuo en el diseño de microprocesadores.

	%% References

	\nocite{*} % to include uncited references of .bib file

	\clearpage
	\bibliographystyle{ieeetr}

	% Generated from .bib file
	\bibliography{ref}
\end{document}
