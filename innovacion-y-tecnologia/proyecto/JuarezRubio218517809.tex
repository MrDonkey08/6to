\documentclass[11pt, a4paper]{article} % Formato

% Language and font encodings
\usepackage[spanish]{babel}
\usepackage[utf8]{inputenc}
\usepackage[T1]{fontenc}
\usepackage{times} % Times New Roman

%% Sets page size and margins
\usepackage[margin=2.5cm, includefoot]{geometry}
%\setlength{\columnsep}{0.17in} % page columns separation

%% Useful packages
\usepackage{amsmath}
\usepackage{array} % <-- add this line for m{} column type
\usepackage[hidelinks]{hyperref} % hyperlinks support
\usepackage{graphicx} % images support
%\usepackage{listings} % codeblock support
%\usepackage{smartdiagram} % diagrams support
\usepackage[most]{tcolorbox} % callouts support
%\usepackage[colorinlistoftodos]{todonotes}
\usepackage[dvipsnames, table, xcdraw]{xcolor} % Tables support
%\usepackage{zed-csp} % cchemas support

%% Formating
\usepackage{authblk} % to add authors in maketitle
%\usepackage{blindtext} % to gen filler text
\usepackage[figurename=Fig.]{caption} % to change prefix of the image caption

%\usepackage{apacite}
\usepackage{cite} % useful to compress multiple quotations into a single entry
\usepackage{enumitem}
\usepackage{fancyhdr} % to set page style
\usepackage{indentfirst}
%\usepackage{natbib}
\usepackage{parskip} % remove first line tabulation
\usepackage{setspace}
%\usepackage{titlesec}
%\usepackage{titling} % to config maketitle

%% Variables
% Main images
\newcommand{\logoUdg}{logo-udg.jpg}
\newcommand{\logoCucei}{logo-cucei.jpg}
\newcommand{\newAmazonSection}{./img/productos-nuevos.jpg}

% School data
\newcommand{\universidad}{Universidad de Guadalajara}
\newcommand{\cede}{Centro Universitario de Ciencias Exactas e Ingenierías}

% Subject data
\newcommand{\materia}{Interacción Humano Computadora}
\newcommand{\carrera}{Ingeniería en Computación}
\newcommand{\division}{División de Tecnologías para la Integración CiberHumana}
\newcommand{\theTitle}{3. Comprender la Toma de Deciciones Humanas}
\newcommand{\profesor}{José Luis David Bonilla Carranza}
\newcommand{\seccion}{D01}
\newcommand{\nrc}{209754}
\newcommand{\clave}{IL367}
\newcommand{\startDate}{20 de septiembre de 2024}

% Author data
\newcommand{\theAuthor}{Juárez Rubio Alan Yahir}
\newcommand{\theAuthorCode}{218517809}
\newcommand{\theAuthorMail}{alan.juarez5178@alumnos.udg.mx}

%% Declaration
\date{}
\graphicspath{ {../../../img/} }
\addto\captionsspanish{\renewcommand{\contentsname}{Índice}}
\renewcommand{\lstlistingname}{Código} % to change prefix of the code caption
\renewcommand{\lstlistlistingname}{Índice de códigos} % to change listings index title

%% Styles

% Color declaration
\definecolor{greenPortada}{HTML}{69A84F}
\definecolor{LightGray}{gray}{0.9}
\definecolor{codegreen}{rgb}{0, 0.6, 0}
\definecolor{codegray}{rgb}{0.5, 0.5, 0.5}
\definecolor{codepurple}{rgb}{0.58, 0, 0.82}
\definecolor{backcolour}{rgb}{0.95, 0.95, 0.92}

% Hyperlinks
\hypersetup{
    colorlinks=true,
    linkcolor=black,
    filecolor=greenPortada,
    urlcolor=greenPortada,
    pdfpagemode=FullScreen,
}

\urlstyle{same}

% Codeblocks
\lstdefinestyle{mystyle}{
	backgroundcolor=\color{backcolour},
	commentstyle=\color{codegreen},
	keywordstyle=\color{magenta},
	numberstyle=\tiny\color{codegray},
	stringstyle=\color{codepurple},
	basicstyle=\ttfamily\footnotesize,
	breakatwhitespace=false,
	breaklines=true,
	captionpos=b,
	keepspaces=true,
	numbers=left,
	numbersep=5pt,
	showspaces=false,
	showstringspaces=false,
	showtabs=false,
	tabsize=2
}

% Tables
\let\oldtabular\tabular
\renewcommand{\tabular}{\small\oldtabular}
\renewcommand{\arraystretch}{1.2} % <-- Adjust vertical spacing
\addto\captionsspanish{\renewcommand{\tablename}{Tabla}}

\lstset{style=mystyle}

%% Spacing
\newcommand{\nl}{\par
\vspace{0.4cm}}
\renewcommand{\baselinestretch}{1.5} % Espaciado de línea anterior
\setlength{\parskip}{6pt} % Espaciado de línea anterior
\setlength{\parindent}{0pt} % Sangría

% Header and footer
\pagestyle{fancy}
\fancyhf{}
\renewcommand{\headrulewidth}{3pt}
\renewcommand{\headrule}{\hbox to\headwidth{\color{greenPortada}\leaders\hrule height \headrulewidth\hfill}}
\setlength{\headheight}{50pt} % Ajuste necesario para evitar warnings

% Header
\pagestyle{fancy}
\fancyhf{}
\lhead{
	\begin{minipage}[c][2cm][c]{1.3cm}
		\begin{flushleft}
			\includegraphics[width=5cm, height=1.4cm, keepaspectratio]{\logoUdg}
		\end{flushleft}
	\end{minipage}
	\begin{minipage}[c][2cm][c]{0.5\textwidth} % Adjust the height as needed
		\begin{flushleft}
			{\materia}
		\end{flushleft}
	\end{minipage}
}

\rhead{
	\begin{minipage}[c][2cm][c]{0.4\textwidth} % Adjust the height as needed
		\begin{flushright}
			{\theTitle}
		\end{flushright}
	\end{minipage}
	\begin{minipage}[c][2cm][c]{1.3cm}
		\begin{flushright}
			\includegraphics[width=5cm, height=1.4cm, keepaspectratio]{\logoCucei}
		\end{flushright}
	\end{minipage}
}

% Footer
\fancyfoot{}
\lfoot{\small\materia}
\cfoot{\thepage} % Paginación
\rfoot{\small Curso impartido por \profesor}

%% Title

\title{\fontsize{24}{28.8}\selectfont \theTitle}
\author{\theAuthor}

\affil{}


\begin{document}
\setstretch{1.5} % Interlineado

\begin{titlepage}
	\newgeometry{margin=2.5cm, left=3cm, right=3cm} % change margin
	\centering
	%\vspace*{-2cm}
	{\LARGE \textbf{\universidad}}\par
	\vspace{0.6cm}
	{\Large{\cede}}
	\vfill

	\begin{figure}[ht]
		\begin{minipage}[t]{0.45\textwidth}
			\centering
			\includegraphics[width=130px, height=160px, keepaspectratio]{\logoUdg}
		\end{minipage}
		\hfill
		\begin{minipage}[t]{0.45\textwidth}
			\centering
			\includegraphics[width=130px, height=160px, keepaspectratio]{\logoCucei}
		\end{minipage}
	\end{figure}
	\vfill

	\large{
		\division\vfill
		\textbf{\carrera}\vfill
		\textbf{\materia}\par\vspace{3pt}
		\seccion\ - \clave\ - \nrc\ - \generation \vfill
	}

	{\Large{\textbf{\theTitle}}}
	\vfill

	\begin{figure}[ht]
		\centering
		\begin{minipage}[t]{0.6\textwidth}
			{\large
				\textbf{Profesor}: \profesor\nl
				\textbf{Alumno}: \theAuthor\nl
				\textbf{Código}: \theAuthorCode\nl
				\textbf{Correo}: \theAuthorMail
			}
		\end{minipage}
	\end{figure}
	\vfill

	\begin{tcolorbox}
		[colback=red!5!white, colframe=red!75!black]
		\centering
		\large{
			Este documento contiene información sensible.\\
			No debería ser impreso o compartido con terceras entidades.
		}
	\end{tcolorbox}
	\vfill
	{\large \startDate}\par
\end{titlepage}

\restoregeometry % end changed margin

%% Indexes
\clearpage
\tableofcontents

\clearpage
\listoffigures

%\clearpage
%\listoftables

%\clearpage
%\lstlistoflistings

%% Main Title
\clearpage
\vspace*{-16pt}
\begin{center}
	{\textbf{\huge \theTitle}}
\end{center}
\vspace*{8pt}

%% Content

\section*{Indicaciones}

Para proyecto final se debe entregar un documento en el que se recopilen
todas las actividades vistas durante el semestre en orden cronológico, a parte
de las actividades el documento debe tener un texto de conexión entre cada
actividad, es decir, no contara sino solo se ponen las actividades, se debe de
poner un texto que interconecte a cada una de ellas, de tal forma en que se
pueda leer un texto corrido y se entienda que se está viendo en cada mapa o
figura que se muestre. Se pueden incluir descripciones de figuras y mapas.

Debe incluir:

\begin{itemize}
	\item Índice.
	\item Arial 12.
	\item Interlineado 1.5 líneas.
	\item Margen Superior, inferior y derecho de 2.5 cm.
	\item Margen izquierdo 3 cm.
	\item Texto justificado.
	\item Extensión mínima 17 paginas.
	\item Lineamientos de tareas.
\end{itemize}

% 1st task

\clearpage
\section{Antecedentes}

\subsection{Qué es Agiotech}

\textbf{Agiotech} es una empresa creada en Enero de 2012, por un grupo de
profesionales expertos en modelos eficientes de servicio ``Post-Venta''
orientado al mercado de electrónicos de consumo en la región de Latinoamérica, contando
con más de 15 años de experiencia en el soporte a compañías consideradas dentro de
las Top 25 OEM's en la Industria Electrónica.

Agiotech es una empresa dedicada principalmente a la reparación de dispositivos y
accesorios móviles. Manejan reparaciones de dispositivos de diversas marcas, tales
como Huawei, Motorola y su principal fuerte, Samsung. Las reparaciones provenien
de garantías de empresas como Elektra, Telcel y el mismo Samsung.

Actualmente plantea ser el principal fuente de reparaciones de la empresa de Samsung
en México, ofreciendo la mayor cantidad de reparaciones, en el menor tiempo cumpliendo
los estándares de calidad.

Si bien, hoy en día son una de las principales fuentes de reparación de Samsung,
considero que hay mucho trabajo por delante y muchos puntos de mejora para poder
cumplir con las expectativas de Samsung.

\subsection{Áreas de Agiotech}

La empresa de Agiotech se divide en varias áreas o departamentos de trabajo:

\begin{itemize}
	\item \textbf{Ingreso}: Área en donde los capturistas ingresan la información de
	      los dispositivos entrantes en los sistemas, asegurándose que haya
	      concordancia con la información respecto a los dispositivos

	\item \textbf{Sistemas}: Área enfocada en el desarrollo y mantenimiento del sistema
	      principal de la empresa. Este sistema es el encargado de llevar el registro
	      de cada una de las operaciones dentro de la empresa, tales como la llegada de
	      dispositivos, piezas, inventariado, uso, estatus, registro de procedimiento,
	      embarcamiento, etc. Adicionalmente, esta área se encarga de resolver
	      cualquier problema que se llegase a presentar dentro del sistema.

	\item \textbf{Recursos Humanos}: Recursos Humanos es el departamento encargado
	      de conseguir personal nuevo y llevar a cabo el proceso de la entrevista.
	      Adicionalmente manejan el control administrativo sobre los empleados.

	\item \textbf{Reparaciones}: Esta área se del diagnóstico de los dispositivos de
	      reparación, determinación de si un dispositivo entra en garantía, si es reparable
	      o no. Si el dispositivo entra en garantía o reparación pagada, este es reparado
	      cumpliendo con cada uno de los requisitos y acuerdos con el cliente o empresa.

	\item \textbf{Administración}: Este departamento a grandes ragos se encarga del
	      manejo de recursos de la empresa, tanto de la salida como la entrada de
	      dinero, es decir, el cómo se manejan las finanzas.

	\item \textbf{Almacen}: Área encagada de la recepción y entrega de paquetes,
	      paquetes que contienen tanto dispositivos como piezas. En general, tienen la
	      tarea principal del movimiento del inventario, registrando y entregando las piezas
	      solicitadas a los técnicos correspondientes.
\end{itemize}

Si bien existen diversas áreas y departamentos y tienen un sistema muy completo para
llevar a cabo garantías y reparaciones, cosidero que esta empresa tiene muchos
aspectos que mejorar, aspectos como la búsqueda de personal calificado, mejores
capacitaciones, uso de tecnologías más avanzada, entre otras cosas.

Por otra parte, considero que esta empresa tiene muchísimas oportunidades de crecimiento
y expansión. La creación de nuevas cedes, el manejo de más marcas, la reparación
y venta de otros dispositivos, tales como dispositivos de cómputo, entre otros
muchos aspectos.

% 2nd task

\clearpage

\section{Red de Clientes}

Como se mencionó anteriormente, actualmente \textit{Agiotech} es una empresa la cual
su red de clientes está compuesta por empresas tales como Elektra y Telcel y la misma
marca Samsung, ofreciéndoles los diferentes servicios de reparación para los
equipos de sus clientes. No obstante es importante modelar \textit{comportamiento
	de la red de clientes} y generar diferentes estrategias para expandir la red de
clientes, tal como se muestra en la fígura ~\ref{fig:Red de Clientes}

\begin{figure}[ht]
	\centering
	\includegraphics[width=\textwidth]{\figA}
	\caption{Mapa Conceputal del Comportamiento y Estrategias de la Red de
		Clientes de la Empresa de Agiotech}
	\label{fig:Red de Clientes}
\end{figure}

% 3th task

\clearpage
\section{Generador de Estrategías de Red de Clientes}

Una vez modelado la red de clientes de la empresa, es importante generar diferentes
estrategías para ampliar la red de clientes, estrategías que permitan una mayor
interacción y conectividad entre cada uno de los clientes y estrategías que
permitan obtener diferentes tipos de clientes. A continuación se muestra el \textit{generador
	de estrategias de red de clientes} para la empresa de \textit{agiotech}.

\subsection{Establecer Objetivos}

\subsubsection{Objetivos Específicos}

Desarrollo de plataformas digitales (e.g. sitios web) que permitan:

\begin{itemize}
	\item Crear una comunidad, recibir retroalimentación de los clientes, ofrecer y
	      publicitar nuestros servicios, y dar a conocer a la empresa.

	\item Incorparar clientes minoristas al modelo de negocio y, además, publicitar
	      e informar a los distintos clientes sobre los diferentes servicios con los que
	      cuenta la empresa.
\end{itemize}

\subsubsection{Objetivos de Orden Superior}

\begin{itemize}
	\item Implementar nuevos modelos de servicios para los distintos tipos de clientes
	      y atención personalizada.

	\item Actualizar los procesos existentes e implementar nuevos procesos que permitan
	      llevar a cabo cada uno de los servicios que se plantea que provea la empresa
	      a cada uno de los diferentes clientes (i.e. minoristas, mayoristas y socios).
\end{itemize}

\subsection{Selección y Enfoque del Cliente}

La obtención de cada uno de los objetivos planteados en la sección anterior
implica:

\begin{itemize}
	\item Desarrollar un nuevo sistema de atención para los clientes minoristas y,
	      además establecimiento de los diferentes servicios, procesos y respectivos costos
	      que se llevarán a cabo.

	\item Desarrollar un sistema de feedback, para que cada uno de los diferentes
	      clientes pueda calificarnos, dar su opinión y darnos recomendaciones de cómo
	      mejorar nuestros servicios.

	\item Implementación de nuevas metodologías de trabajo con el fin de mejorar los
	      procesos que permiten llevar a cabo cada uno de los servicios que se proveen
	      actualmente a los socios y mayoristas.
\end{itemize}

Nuestra empresa busca garantizar servicios de reparación y mantenmiento de equipos
móviles y sus respectivos accesorios, especialmente a equipos de la marca
Samsung, asegurándole a nuestros clientes de llevar a cabo estándares de calidad
de servicio y de cumplir con las fechas de entrega establecidas.

El cumplimiento de cada uno de estos objetivos implica un gran desafío para nuestra
empresa debido a que desafortunadamente no se cuenta con personal lo suficientemente
capacitado para mudar nuestros procesos a nuevos sistemas y a plataformas digitales.

Asímismo, la implementación de nuevos procesos, adquisición de nuevos clientes,
entre otras metas, significa un gran reto para nuestra empresa debido a que practicamente
se estaría redefiniendo cada uno de los aspectos de la empresa, con el fin de
mejorar y garantizar nuestros servicios y expandir y mejorar nuestra empresa.

\subsection{Selección de Estrategia}

\subsubsection{Acceso}

\begin{itemize}
	\item \textbf{Plataformas Web y Móvil}: El desarrollo de estas plataformas ayudarían
	      a tener un mayor alcance a clientes, de tal manera que los clientes puedan
	      tener una mayor información acerca de los servicios de reparación y, adicionalmente,
	      un sistema más controlado con el que puedan solicitar los servicios.

	\item \textbf{Cedes}: El desarrollo de nuevas cedes hace que el servicio sea más
	      accesible por más usuarios, de tal manera que los usuarios les es más
	      sencillo ir a visitar alguna de las cedes.
\end{itemize}

\subsubsection{Compromiso}

\begin{itemize}
	\item \textbf{Retroalimentación de Usuario}: La implementación de un usuario para
	      que pueda dar retroalimentación acerca del servicio otorgado garantiza una mayor
	      confiabilidad hacia los demás usuarios.

	\item \textbf{Videoblog a través de redes sociales}: Un videoblog en el que uno
	      o varios técnicos muestren en vivo las reparaciones de los equipos no solo ayuda
	      a que el usuario observe el proceso y tenga referencias, sino también le
	      garantiza una mayor confiabilidad y transparencia respecto a la empresa.
\end{itemize}

\subsubsection{Personalización}

\begin{itemize}
	\item \textbf{Servicio Personalizado}: La notificación del usuario acerca del proceso
	      de reparación, ya sea por mensajes, notificaciones o cualquier otro medio,
	      brinda al usuario una atención más amigable y personalizada, manteniendo al usuario
	      informado en todo momento.

	\item \textbf{Servicio Personalizado}: El ofrecer un accesorio gratuito (incluído
	      con el servicio) a selección del usuario (e.g.~funda, mica) no solo ofrece un
	      servicio más personalizado, sino también ofrece una mejor atención al
	      cliente.
\end{itemize}

\subsubsection{Conexión}

\begin{itemize}
	\item \textbf{Foro}: La implementación de un foro en las \emph{plataformas web
		      y movil} ayuda a que los clientes puedan comunicarse entre sí, tanto para hablar
	      de su experiencia con el servicio, resolver dudas y dar información extra

	\item \textbf{Redes Sociales}: El uso de redes sociales de la empresa es de suma
	      importancia porque, además de crear una marca personal como empresa, ayuda a
	      crear una comunidad propia en la que los clientes pueden compartir experiencias,
	      ideas y demás.
\end{itemize}

\subsubsection{Colaboración}

\begin{itemize}
	\item \textbf{Sección de Sugerencias}: Un sección de sugerencias (pública o
	      privada) da la posibilidad de que el cilente pueda colaborar de tal manera que
	      puede dar ideas para mejorar el servicio y la experiencia del usuario.

	\item \textbf{Asociasiones con negocios}: La posibilidad de que negocios más pequeños
	      puedan asociarse ya sea como cliente mayorista, pasado de clientes (e.g.~que
	      la empresa ofrece reparaciones de computadoras) o de alguna manera, ayuda a
	      crecer y expandir la empresa, no solo reduciendo la competitividad sino incrementando
	      la colaboración.
\end{itemize}

\subsection{Generador de Conceptos}

Para el desarrollo de la plataforma para ofrecer el servicio a los clientes
minoristas es necesario definir una serie de requerimientos:

\begin{itemize}
	\item El sistema permitirá a los usuarios registrarse e iniciar sesión mediante
	      formularios.

	\item El usuario, en primera instancia, se le ofrecerá un diagnóstico (sin
	      costos de envío) para determinar el problema y el estado del equipo

	\item El sistema les presentará a los usuarios los posibles presupuestos para cada
	      una de las distintas reparaciones.

	\item El usuario recibirá mensajes o notificaciones para informarle acerca del
	      estado del diagnóstico o reparación.

	\item En caso de haber diagnosticado un equipo, el usuario podrá optar por
	      continuar o rechazar la reparación.
\end{itemize}

% 4th task

\clearpage
\section{Mapa del Modelo de Negocio de Plataforma}

Si bien, el \textbf{generador de estrategías de red de clientes} es una herramienta
muy útil para expandir nuestra red de clientes, también es de vital importancia
elaborar un \textbf{mapa del modelo de negocio de la plataforma} debido a que nos
permite definir cada una de las interacciones que se encuentran en nuestro modelo
de negocio con cada uno de los diferentes clientes. En la fígura
\ref{fig: modelo de negocio} se muestra el mapa de negocio de Agiotech.

\begin{figure}[ht]
	\centering
	\includegraphics[width=\textwidth]{\figB}
	\caption{Mapa del modelo de negocio de Agiotech}
	\label{fig: modelo de negocio}
\end{figure}

% 5th task


\clearpage
\section{Tren de Valor Competitivo}

Si bien el \textbf{mapa de modelo de negocio} nos permite tener una mejor noción
de la interacciones que existen dentro de nuestra de red de clientes y la relación
que cada tipo de cliente tiene con nuestra empresa y cada uno de nuestros negocios,
es importante desarrollar el \textbf{tren de valor competitivo}.

Uno de los puntos más importantes del \textbf{tren de valor competitivo} es que nos permite
ver en qué posición de la cadena nos encontramos en el flujo de producción y entrega final del producto o servicio proporcionado al cliente objetivo y con ello, en un futuro, generar propuestas de valor para evitar la desintermediación (en caso de estar en un punto medio).

A cotinuación se encuentra el tren de valor de Agiotech que muestra la cadena que
existe entre cada uno de los servicios que proporciona repecto a cada uno de sus
clientes y socios.

\begin{figure}[ht]
	\centering
	\includegraphics[width=\textwidth]{\figC}
	\caption{Tren de valor competitivo de Agiotech de servicios de garantía, diagnóstico
		y reparación de télefonos Samsung}
\end{figure}

% 6th task

\clearpage
\section{El Generador de Valor de Datos: Una Herramienta Estratégica para la Transformación
  Digital}

Una vez realizado el \textbf{tren de valor competitivo} y cada uno de los puntos
mencionados en cada una de las secciones anteriores podemos proseguir con el generador
de valor de datos, el cual nos ayuda a generar nuevas opciones estratégicas para
las iniciativas de datos para la empresa de Agiotech

\subsection{Marco Teoríco}

En la era de la transformación digital, las organizaciones enfrentan el desafío de
convertir los datos en un recurso estratégico que impulse el crecimiento, la
eficiencia y la innovación. Este enfoque se estructura en cinco pasos fundamentales:
\textbf{definición del área de impacto y los KPIs}, \textbf{selección de la
	plantilla de valor}, \textbf{generación del concepto}, \textbf{auditoría de
	datos} y \textbf{plan de ejecución}. A continuación, exploramos cómo cada paso
contribuye al desarrollo de una estrategia de datos exitosa.

\subsubsection{Definición del Área de Impacto e Indicadores Clave de Rendimiento
	(KPIs)}

El primer paso en el Generador de Valor de Datos es identificar las áreas del negocio
donde el uso de datos puede tener mayor impacto. Es aquí donde se definen las
prioridades estratégicas, ya sea optimizar la eficiencia operativa, mejorar la
experiencia del cliente o aumentar los ingresos. Este análisis permite establecer
\textbf{indicadores clave de rendimiento (KPIs)} específicos y medibles que
alineen los esfuerzos de datos con los objetivos organizacionales. Por ejemplo,
un KPI podría ser reducir los tiempos de entrega en un 20\% o aumentar la tasa de
conversión de clientes en un 15\%. La claridad en este paso inicial asegura que todo
el proyecto de datos se enfoque en resultados tangibles y relevantes.

\subsubsection{Selección de la Plantilla de Valor}

Una vez definida el área de impacto, el siguiente paso es elegir una plantilla de
valor adecuada. Rogers presenta cinco modelos principales para crear valor con
datos: \textbf{agilización de procesos}, \textbf{mejora de la experiencia del
	cliente}, \textbf{innovación en productos o servicios}, \textbf{optimización de
	decisiones} y \textbf{monetización de datos}. Cada plantilla ofrece un enfoque
distinto, desde aumentar la eficiencia interna hasta generar nuevas fuentes de ingresos.
La elección correcta depende de las prioridades y el modelo de negocio de la
organización. Este marco proporciona claridad y dirección, asegurando que la
estrategia de datos esté alineada con las metas corporativas.

\subsubsection{Generación del Concepto}

Con la plantilla de valor seleccionada, el siguiente paso es traducirla en un
concepto claro y práctico. Aquí se combinan los objetivos estratégicos con la plantilla
de valor para desarrollar una idea inicial de cómo los datos se utilizarán en la
práctica. Por ejemplo, si se seleccionó la optimización de decisiones como plantilla
de valor, el concepto podría ser la implementación de un sistema de analítica
predictiva para gestionar inventarios en tiempo real. Este paso conecta la estrategia
con la operación, permitiendo que la visión general se traduzca en una
iniciativa concreta y realizable.

\subsubsection{Auditoría de Datos}

Ningún proyecto de datos puede avanzar sin una evaluación rigurosa de la calidad
y disponibilidad de los datos existentes. La auditoría de datos es un paso
crítico para identificar fuentes de datos internas y externas, evaluar su relevancia,
precisión y accesibilidad, y detectar posibles brechas. Este análisis
proporciona un diagnóstico de la infraestructura de datos de la organización y
señala las áreas que requieren mejora. Por ejemplo, una auditoría podría revelar
que los datos del cliente están desactualizados o que existen limitaciones en la
integración de sistemas. Este paso asegura que los datos sean fiables y útiles para
apoyar el concepto generado.

\subsubsection{Plan de Ejecución}

Finalmente, el Generador de Valor de Datos culmina en la creación de un plan de ejecución
detallado. Este plan incluye un roadmap con prioridades, roles definidos, plazos
específicos y recursos asignados. También considera aspectos como la gestión del
cambio organizacional y el diseño de un ciclo iterativo para implementar,
evaluar y ajustar la solución. Un ejemplo de un plan de ejecución eficaz podría ser
lanzar un proyecto piloto en una unidad de negocio antes de escalarlo a toda la organización.
Este enfoque estructurado garantiza que la estrategia no se quede en una idea
abstracta, sino que se convierta en acciones concretas y medibles.

\clearpage
\subsection{Aplicación del Generador de Valor de Datos}

En el competitivo mundo de la tecnología, las empresas deben adaptarse constantemente
para mantenerse relevantes. En este contexto, una empresa dedicada a la reparación
de celulares Samsung, que actualmente trabaja con grandes clientes como Samsung,
Elektra y Telcel, busca innovar ofreciendo servicios también a clientes
minoristas. Para lograrlo, se propone desarrollar una plataforma digital (web y móvil)
que permita a estos consumidores acceder fácilmente a reparaciones y accesorios.
Este ensayo explora cómo el \textbf{Generador de Valor de Datos}, propuesto por David
L. Rogers, puede guiar este proceso de innovación mediante cinco pasos estratégicos:
definir áreas de impacto y KPIs, seleccionar la plantilla de valor, generar un
concepto, realizar una auditoría de datos y ejecutar un plan.

\subsubsection{Área de Impacto e Indicadores Clave de Rendimiento (KPIs)}

La primera etapa consiste en identificar el impacto que se busca lograr con esta
transformación. En este caso, la prioridad es expandir el mercado hacia los
clientes minoristas y mejorar su experiencia. Mientras que la empresa ya domina el
segmento B2B, la incursión en el mercado B2C requiere un enfoque centrado en el
cliente. Para medir el éxito de esta iniciativa, se establecen indicadores clave
de rendimiento (KPIs) específicos, como aumentar en un 30\% los ingresos anuales
provenientes de clientes minoristas, registrar 5,000 nuevos usuarios en la plataforma
durante los primeros seis meses, y reducir en un 20\% el tiempo promedio de gestión
de reparaciones mediante la automatización. Estas metas proporcionan un marco
claro para medir el impacto de la innovación.

\subsubsection{Selección de la Plantilla de Valor}

El siguiente paso es elegir la plantilla de valor que mejor se alinee con los objetivos
de la empresa. Dada la naturaleza del proyecto, se seleccionan dos enfoques:
\textbf{mejorar la experiencia del cliente} y \textbf{agilizar los procesos
	operativos}. Por un lado, la creación de una plataforma que permita a los
clientes registrar solicitudes, realizar pagos y seguir el progreso de sus reparaciones
en tiempo real promete una experiencia fluida y confiable. Por otro lado, la automatización
de procesos internos, como la asignación de técnicos y la actualización de
inventarios, permitirá a la empresa manejar mayores volúmenes de solicitudes de manera
eficiente. Esta combinación asegura que tanto los clientes como la operación
interna se beneficien de la transformación.

\subsubsection{Generación del Concepto}

El concepto que surge de esta estrategia es una **plataforma digital integral que
sirva como punto de contacto para los clientes minoristas. En ella, los usuarios
podrán crear cuentas personalizadas, gestionar sus solicitudes de reparación, realizar
pagos en línea y recibir actualizaciones en tiempo real sobre el estado de sus equipos.
Además, incluirá un foro para resolver preguntas frecuentes y obtener soporte directo.
Este concepto no solo facilita el acceso a los servicios, sino que también posiciona
a la empresa como una opción moderna y confiable en el sector de reparaciones tecnológicas.
La plataforma será accesible tanto desde dispositivos móviles como desde
navegadores web, garantizando una experiencia inclusiva y adaptable.

\subsubsection{Auditoría de Datos}

Para implementar este concepto, es esencial evaluar la calidad y disponibilidad de
los datos existentes. Una auditoría de datos permitirá analizar fuentes internas,
como el historial de reparaciones, los tiempos promedio de servicio y la
disponibilidad de piezas proporcionadas por Samsung. También será necesario explorar
fuentes externas, como tendencias de mercado y opiniones de clientes potenciales.
Sin embargo, es probable que la empresa identifique brechas significativas en su
conocimiento sobre las necesidades y expectativas del cliente minorista, ya que este
segmento no ha sido atendido previamente. La recopilación de estos datos, a través
de encuestas o proyectos piloto, será crucial para personalizar la plataforma y
optimizar su funcionalidad.

\subsubsection{Plan de Ejecución}

El desarrollo de la plataforma se llevará a cabo en fases escalonadas. Durante
los primeros seis meses, se enfocará en la creación de un prototipo funcional,
con características básicas como la gestión de cuentas, la solicitud de servicios
y la integración de pagos en línea. Una vez desarrollado, se lanzará un programa
piloto en una región limitada para recopilar feedback de los usuarios y ajustar la
plataforma según sus necesidades. En una etapa posterior, la empresa escalará la
solución a nivel nacional, incorporando mejoras como programas de fidelidad y
promociones exclusivas para clientes recurrentes. Este plan de ejecución está diseñado
para garantizar una transición fluida y sostenible hacia el nuevo modelo de negocio,
al tiempo que minimiza riesgos.

\clearpage

% 7th task

\clearpage
\section{Innovación Mediante la Experimentación Rápida}

Una vez aplicado los puntos de previas secciones, ya tenemos bastante
entendimiento acerca de la empresa, de su funcionamiento, de su red de clientes,
cómo mejorarlo y todo lo que conlleva que ya estamos listos para innovar.

Antes de continuar a desarrollar cada uno de los tipos de innovación mediante la
experimentación rápida, primeramente veremos en grandes rasgos en qué consiste.

La \textbf{innovación} no es más que el proceso que conlleva la realización o actualización
de productos y servicios, los cuales deben resultar en impactos positivos.

El \textit{proceso de innovación} consiste principalmente en transformar ideas creativas
en resultados que mejoren la eficiencia o eficacia o responda a necesidades insatisfechas.

En la actualidad, la forma más práctica de innovar es mediante la \textbf{ experimentación
	rápida}, debido a que permite desarrollar o actualizar productos y servicios y
determinar los resultados de una manera rápida y barata. Esta se divide en dos tipos:

\begin{itemize}
	\item Experimentación convergente o experimentación formal

	\item Experimentación divergente o experimentación informal
\end{itemize}

\clearpage
\section{Método Experimental Convergente}

Como vimos en la sección previa, la experimentación rápida es un modelo que nos
permite experimentar en plazos cortos de tiempo y sin inversiones grandes y
arriesgadas, permitiéndolos probar cada una de nuestras ideas y, de esta manera,
averiguar si dicha innovación es viable y no y cómo ir llevándola a cabo.

A continuación, veremos en qué consite el método experimental convergente y cómo
ha sido aplicado a nuestra empresa Agiotech.

\subsection{Marco Teórico}

El \textbf{método experimental convergente} es útil para innovar productos, servicios
y procesos existentes, para optimizarlos y mejorarlos constantemente. Este método
se destaca debido a que pueden ser realizados rápidamente, en cuestión de horas
y minutos.

El \emph{método experimental convergente} se divide en siete pasos:

\begin{enumerate}
	\item Definir la pregunta y sus variables

	\item Elegir los probadores (\emph{testers})

	\item Aleatorizar la prueba y el control

	\item Validar la muestra

	\item Probar y analizar

	\item Decidir

	\item Compartir el aprendizaje
\end{enumerate}

\subsubsection{Definir la Pregunta y Sus Variables}

Para iniciar un \emph{experimento convergente} es necesario definir la pregunta
que se busca responder. Esta pregunta debe ser tan específica como sea posible.

Una vez planteada la pregunta, se tiene que traducir en dos tipos de variables

\begin{itemize}
	\item \textbf{Variable independiente (causa)}: Factor que se probará en el experimento.
	      El objetivo de este experimento es entender el efecto de la introducción de
	      esta innovación.

	\item \textbf{Variable dependiente (efecto)}: Factor en el que se espera que la
	      innovación pueda influir. Es una medida del impacto de lo que se está
	      cambiando.
\end{itemize}

\subsubsection{\texorpdfstring{Elegir los Probadores (\emph{Testers})}{Elegir
		los Probadores (Testers)}}

Este paso consiste en seleccionar los sujetos de prueba del experimento. Es
importante, pero no estrictamente necesario tener dos tipos de probadores:

\begin{itemize}
	\item \textbf{Probadores de caja blanca}: Personas que son parte del desarrollo
	      del experimento y, por ende, tienen conocimiento acerca del funcionamiento interno
	      del experimento.

	\item \textbf{Probadores de caja negra}: Personas que no son parte del desarrollo
	      del experimento y, por ende, no tienen conocimiento acerca del funcionamiento
	      interno del experimento.
\end{itemize}

El objetivo de las pruebas es recabar la suficiente información acerca del
experimento para su posterior análisis y toma de decisiones.

\subsubsection{Aleatorizar la Prueba y el Control}

Antes de llevar a cabo un experimento convergente, es importante identificar una
población cuyas repuestas deseas probar. Seguidamente asignas al azar miembros de
esa población a uno de estos dos grupos:

\begin{itemize}
	\item El grupo de prueba (o grupo de tratamiento), que realiza la experiencia
	      o recibe la oferta que estás probando.

	\item El grupo de control, al que no se aplica el factor testeado.
\end{itemize}

\subsubsection{Validar la Muestra}

Este paso consisten en asegurarse que se cuenta con un tamaño de muestra válido.
Para ello, primeramente es necesario identificar la unidad de análisis (i.e. producto,
servicio o proceso).

Seguidamente, se establece el tamaño de la muestra (i.e, número de unidades que
se coloca a cada uno de los grupos de prueba), la regla típica es $n = 100$, como
mínimo, en cada grupo que se compara.

\subsubsection{Probar y Analizar}

En este punto uno ya se encuentra listo para realizar la prueba. El equipo que
lleve a cabo tu experimento recogerá datos durante un período predeterminado. Luego
tendrá que analizar los datos para ver si hay diferencias en las variables dependientes
que se están midiendo y, si las hay, comprobar si esas diferencias son estadísticamente
significativas.

\subsubsection{Decidir}

Luego de haber analizado los resultados del experimento, es importante tomar una
decisión basada en las conclusiones:

\begin{itemize}
	\item \textbf{Abandonar el experimento}: Existen ocasiones en donde la mejor alternativa
	      es abandonar debido a que es inviable seguir llevándolo a cabo (i.e.~por tiempo,
	      costos u otros factores).

	\item \textbf{Mejorar el experimento}: Iterar el experimento y realizar pruebas
	      para analizar si hay mejoras respecto a previas iteraciones.

	\item \textbf{Finalizar el experimento}: Se finaliza el experimento una vez que
	      haya sido recabada la información suficiente para determinar la respuesta a la
	      pregunta planteada desde un inicio.
\end{itemize}

\subsubsection{Compartir el Aprendizaje}

Una vez finalizado el proyecto, es fundamental documentar lo que se haya
aprendido, esto con el objetivo de comunicar los resultados para que otras personas
de la misma organización que podían beneficiarse.

\subsection{Aplicación del Método Experimental Divergente}

La búsqueda de soluciones innovadoras para enfrentar desafíos empresariales es
una constante en el camino hacia la transformación digital. En este contexto,
Agiotech, una empresa especializada en la reparación de celulares y accesorios Samsung,
busca ampliar su base de clientes mediante estrategias disruptivas. Para lograrlo,
se implementará el método experimental divergente, descrito en \emph{Guía
	Estratégica Para la Transformación Digital} de David L. Rogers. Este método
permite explorar múltiples enfoques para resolver un problema, identificar la solución
más efectiva y escalarla para maximizar su impacto.

\subsubsection{Definir el Problema}

El principal desafío de Agiotech es \textbf{ampliar su red de clientes, específicamente
	en el segmento de minoristas}, quienes representan un mercado potencialmente
lucrativo. Sin embargo, este objetivo requiere superar barreras como la falta de
visibilidad de la marca y una oferta limitada de servicios personalizados que
atraigan a estos clientes.

\subsubsection{Establecer Límites}

Para garantizar un enfoque eficiente, se establecieron los siguientes límites:

\begin{itemize}
	\item \textbf{Tiempo:} El proyecto tendrá una duración máxima de 3 meses, desde
	      la ideación hasta la evaluación de resultados.

	\item \textbf{Presupuesto:} Los recursos asignados no superarán el 10\% de los
	      ingresos mensuales actuales de Agiotech.

	\item \textbf{Alcance:} Se priorizarán minoristas ubicados en un radio de 50
	      km de las oficinas de Agiotech.
\end{itemize}

\subsubsection{Escoger a la Gente}

Se seleccionó un equipo multidisciplinario compuesto por:

\begin{itemize}
	\item \textbf{Técnicos:} Para diseñar y evaluar las soluciones tecnológicas.

	\item \textbf{Personal de ventas:} Para aportar conocimiento sobre las
	      necesidades de los minoristas.

	\item \textbf{Clientes potenciales:} Representantes de pequeños negocios que
	      aportarán información clave sobre sus expectativas y preferencias.
\end{itemize}

Este enfoque garantiza una perspectiva integral durante todo el proceso.

\subsubsection{Observar}

Se realizaron observaciones directas y entrevistas con minoristas locales para
entender sus problemas y expectativas. Los hallazgos clave incluyeron:

\begin{itemize}
	\item La necesidad de un acceso más rápido y fácil a servicios de reparación.

	\item Interés en descuentos y paquetes personalizados.

	\item Falta de confianza en los servicios de reparación disponibles en la región.
\end{itemize}

\subsubsection{Generar Más de Una Solución}

Con base en las observaciones, se plantearon las siguientes soluciones:

\begin{enumerate}
	\item \textbf{Sitio web personalizado:} Similar al mencionado en el método
	      convergente, pero optimizado para incluir paquetes específicos para minoristas.

	\item \textbf{Plataforma móvil de reparaciones a domicilio:} Un servicio que
	      permite a los minoristas solicitar reparaciones sin necesidad de desplazarse.

	\item \textbf{Programa de lealtad:} Ofrecer incentivos como descuentos
	      acumulables o servicios gratuitos después de un número determinado de
	      órdenes.
\end{enumerate}

\subsubsection{Construir un MVP (Prototipo Mínimo Viable)}

Se desarrollaron prototipos básicos para cada solución:

\begin{enumerate}
	\item Una versión preliminar del sitio web, centrada en mostrar paquetes
	      minoristas.

	\item Una aplicación móvil funcional para programar visitas de reparación a domicilio.

	\item Un sistema simple de registro manual para probar el programa de lealtad.
\end{enumerate}

\subsubsection{Prueba de Campo}

Los MVP se probaron con un grupo de minoristas seleccionados:

\begin{itemize}
	\item \textbf{Sitio web:} Evaluado por su facilidad de uso y claridad en la
	      oferta de paquetes.

	\item \textbf{Plataforma móvil:} Implementada en una zona piloto para medir la
	      aceptación y efectividad del servicio a domicilio.

	\item \textbf{Programa de lealtad:} Ofrecido a los minoristas actuales para
	      medir el impacto en la frecuencia de sus órdenes.
\end{itemize}

\subsubsection{Decidir}

Tras analizar los resultados, se identificó que la solución más prometedora fue la
\textbf{plataforma móvil de reparaciones a domicilio}. Esta opción obtuvo las mejores
calificaciones en términos de satisfacción del cliente y aumento en el número de
órdenes. Sin embargo, el sitio web también mostró potencial como una herramienta
complementaria para captar nuevos clientes.

\subsubsection{Escalar}

Con base en los resultados, se decidió escalar la plataforma móvil a toda el área
de operación de Agiotech. Además, se integrará el sitio web como un canal adicional
para programar reparaciones y ofrecer promociones.

\subsubsection{Compartir}

Se documentaron los resultados y aprendizajes para compartirlos con todo el
equipo de Agiotech, destacando:

\begin{itemize}
	\item La importancia de ofrecer soluciones que prioricen la comodidad del cliente.

	\item El valor de combinar herramientas digitales (plataforma móvil y sitio
	      web) con incentivos como el programa de lealtad.

	\item Recomendaciones para mejorar la logística y garantizar tiempos de respuesta
	      más rápidos.e podrían beneficiarse.
\end{itemize}

% 8th task

\clearpage
\section{Método Experimental Divergente}

El método experimental divergente, a diferencia del método experimental convergente
es muy útil para aquellas innovaciones en donde se busca crear un producto o servicio
nuevo.

A continuación veremos más a fondo en qué consiste este método y en qué se diferencia
respecto al método experimental convergente y cómo ha sido desarrollado para la empresa
de Agiotech.

\subsection{Marco Teórico}

El \textbf{método experimental divergente}, es útil para las innovaciones menos
definidas desde el principio (e.g.~productos, servicios y procesos comerciales nuevos
de una organización).

Es relevante mencionar que los proyectos de innovación que utilizan esta
metodología tienden a ser altamente iterativos y pueden durar semanas o meses.

El \emph{método experimental divergente} se divide en diez pasos:

\begin{enumerate}
	\item Definir el problema

	\item Establecer límites

	\item Escoger a la gente

	\item Observar

	\item Generar más de una solución

	\item Construir un MVP (Prototipo Mínimo Viable)

	\item Prueba de campo

	\item Decidir

	\item Escalar

	\item Compartir
\end{enumerate}

\subsubsection{Definir el Problema}

Para comenzar un \emph{experimento divergente} primeramente es necesario definir
el problema que se desea resolver. El problema debe estar basado en una necesidad
observada del cliente o en una oportunidad de mercado y debe ser un desafío que la
organización esté particularmente capacitada para resolver.

La definición del problema puede incluir un objetivo cuantificado, pero ese
objetivo debe ser a la vez desafiante y amplio.

\subsubsection{Establecer Límites}

Aquí se determinan los parámetros que enmarcarán la experimentación. Esto
incluye establecer restricciones como el tiempo, los recursos disponibles y los criterios
de éxito. Los límites ayudan a mantener el enfoque en las soluciones más
prácticas y realistas, evitando que los experimentos se extiendan indefinidamente
o se vuelvan inmanejables.

\subsubsection{Escoger a la Gente}

El último paso de la fase de preparación es elegir qué personas trabajarán en tu
experimento de innovación. Este equipo debe estar compuesto por individuos con habilidades
diversas y perspectivas complementarias. La diversidad del equipo fomenta la
generación de ideas innovadoras y asegura que las soluciones propuestas
consideren múltiples puntos de vista.

\subsubsection{Observar}

En este paso, se recopila información detallada sobre el problema y su contexto.
Esto puede implicar observar directamente el comportamiento de los usuarios,
analizar datos o investigar tendencias relevantes. La observación profunda permite
descubrir necesidades ocultas, patrones clave y oportunidades de mejora que las
soluciones deben abordar.

\subsubsection{Generar Más de Una Solución}

El método divergente enfatiza la importancia de no limitarse a una única solución
desde el principio. En esta etapa, el equipo desarrolla múltiples ideas o enfoques
para resolver el problema. La variedad de opciones amplía las posibilidades de
encontrar una solución efectiva y fomenta la creatividad en el proceso.

\subsubsection{Construir un MVP (Prototipo Mínimo Viable)}

Cada solución propuesta se convierte en un prototipo básico que puede ser probado
rápidamente. Un MVP permite evaluar la viabilidad de una idea con una inversión
mínima de tiempo y recursos. El enfoque está en crear algo funcional que capture
la esencia de la solución sin desarrollarla completamente.

\subsubsection{Prueba de Campo}

Los MVP se someten a pruebas en condiciones reales o en entornos controlados que
reflejen el uso previsto. Este paso implica recopilar datos y comentarios sobre cómo
las soluciones funcionan en la práctica. La prueba de campo proporciona
información valiosa sobre la eficacia de cada solución y cómo podría mejorarse.

\subsubsection{Decidir}

Con base en los resultados de las pruebas, se toma una decisión informada sobre
cuál de las soluciones experimentadas tiene el mayor potencial. Este paso incluye
analizar datos y retroalimentación para seleccionar la opción que mejor aborda el
problema definido inicialmente.

\subsubsection{Escalar}

Una vez que se ha identificado una solución ganadora, el siguiente paso es ampliarla
para su implementación en un contexto más amplio. Esto puede implicar optimizar
el diseño, desarrollar funcionalidades adicionales o invertir en recursos para desplegarla
a gran escala.

\subsubsection{Compartir}

Finalmente, el conocimiento y las lecciones aprendidas durante todo el proceso se
documentan y comparten dentro de la organización. Este paso asegura que la
experiencia acumulada a lo largo del método experimental divergente se convierta
en una base para futuros proyectos, fomentando una cultura de aprendizaje continuo
e innovación.

\subsection{Aplicación Del Método Experimental Divergente}

La transformación digital es un pilar fundamental para la competitividad de las
empresas en el siglo XXI. En el caso de Agiotech, una empresa dedicada a la
reparación de celulares y accesorios Samsung, se ha propuesto innovar mediante el
desarrollo de un sitio web para captar clientes minoristas y ofrecer servicios personalizados.
Para evaluar esta estrategia, se aplicará el método experimental convergente descrito
por David L. Rogers en su obra \emph{Guía Estratégica Para la Transformación
	Digital}. Este ensayo detalla el desarrollo de los siete pasos del método para
validar la viabilidad de la propuesta.

\subsubsection{Definir la Pregunta y Sus Variables}

El experimento parte de la siguiente pregunta: \emph{¿El desarrollo de un sitio
	web personalizado incrementará la cantidad de clientes minoristas y mejorará su
	satisfacción con los servicios ofrecidos?} Para responderla, se definen las
siguientes variables:

\begin{itemize}
	\item \textbf{Variable independiente}: La implementación del sitio web personalizado.

	\item \textbf{Variables dependientes}:

	      \begin{itemize}
		      \item Incremento en el número de clientes minoristas registrados.

		      \item Mejora en el nivel de satisfacción del cliente.
	      \end{itemize}
\end{itemize}

Estas variables permitirán evaluar si la innovación tecnológica cumple con los objetivos
estratégicos de Agiotech.

\subsubsection{\texorpdfstring{Elegir los Probadores (\emph{Testers})}{Elegir
		los Probadores (Testers)}}

Para garantizar resultados relevantes, se seleccionaron dos grupos de probadores:

\begin{enumerate}
	\item \textbf{Clientes potenciales minoristas}: Incluyen pequeños negocios que
	      podrían beneficiarse de los servicios de Agiotech, como tiendas de
	      reparación y comercios independientes.

	\item \textbf{Clientes actuales}: Quienes ya interactúan con Agiotech y sirven
	      como referencia para medir mejoras en su experiencia.
\end{enumerate}

Estos grupos representan una muestra adecuada del mercado objetivo de Agiotech.

\subsubsection{Aleatorizar la Prueba y el Control}

Se dividió la muestra en dos grupos de manera aleatoria:

\begin{itemize}
	\item \textbf{Grupo de prueba}: Tendrán acceso al sitio web personalizado.

	\item \textbf{Grupo de control}: Continuarán utilizando los métodos actuales de
	      interacción con la empresa (llamadas telefónicas o visitas físicas).
\end{itemize}

La aleatorización asegura que los resultados del experimento no estén
influenciados por factores externos o sesgos en la selección de los
participantes.

\subsubsection{Validar la Muestra}

Para garantizar que los resultados sean significativos, se verificó que los
grupos fueran representativos del mercado. Los criterios de inclusión incluyeron:

\begin{itemize}
	\item Ubicación geográfica (zonas donde opera Agiotech).

	\item Tamaño del negocio (enfocándose en minoristas).

	\item Frecuencia de uso de servicios de reparación.
\end{itemize}

Se definió un tamaño mínimo de 30 participantes por grupo, logrando así una muestra
suficiente para obtener datos estadísticamente relevantes.

\subsubsection{Probar y Analizar}

La fase de prueba se llevó a cabo durante un período de cuatro semanas. En este tiempo,
se recolectaron las siguientes métricas:

\begin{itemize}
	\item Número de nuevos clientes registrados en el sitio web.

	\item Volumen de órdenes procesadas a través del sitio.

	\item Resultados de encuestas de satisfacción, en las que se evaluaron
	      factores como facilidad de uso del sitio y calidad del servicio.
\end{itemize}

Los datos recolectados se analizaron utilizando herramientas como Excel para obtener
comparaciones entre el grupo de prueba y el de control.

\subsubsection{Decidir}

Con base en los datos obtenidos, se procedió a comparar los resultados de ambos grupos.
Por ejemplo, si el grupo de prueba mostró un incremento del 30\% en nuevos
clientes minoristas y un aumento en la satisfacción promedio de 3.5 a 4.5 puntos
en una escala de 5, se concluyó que el sitio web personalizado fue efectivo.

Esta decisión permitió justificar la escalabilidad de la estrategia, sugiriendo
su implementación completa como parte de las operaciones de Agiotech.

\subsubsection{Compartir el Aprendizaje}

Finalmente, los resultados del experimento fueron documentados y compartidos con
el equipo directivo de Agiotech. Entre los aprendizajes clave se destacaron:

\begin{enumerate}
	\item La importancia de optimizar ciertas funcionalidades del sitio web basadas
	      en comentarios de los usuarios.

	\item La efectividad de la personalización para atraer nuevos clientes.

	\item Recomendaciones para integrar el sitio web con otras herramientas de gestión
	      de clientes.
\end{enumerate}

Estos hallazgos sentaron las bases para futuras iniciativas de transformación
digital en Agiotech.

%% References

\nocite{*} % to include uncited references of .bib file

\clearpage
\bibliographystyle{apalike}

% Generated from .bib file
\bibliography{ref}
\end{document}
