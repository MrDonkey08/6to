\documentclass[11pt, a4paper]{article} % Formato

% Language and font encodings
\usepackage[spanish]{babel}
\usepackage[utf8]{inputenc}
\usepackage[T1]{fontenc}
\usepackage{times} % Times New Roman

%% Sets page size and margins
\usepackage[margin=2.5cm, includefoot]{geometry}
%\setlength{\columnsep}{0.17in} % page columns separation

%% Useful packages
\usepackage{amsmath}
\usepackage{array} % <-- add this line for m{} column type
\usepackage[hidelinks]{hyperref} % hyperlinks support
\usepackage{graphicx} % images support
\usepackage{listings} % codeblock support
%\usepackage{smartdiagram} % diagrams support
\usepackage[most]{tcolorbox} % callouts support
%\usepackage[colorinlistoftodos]{todonotes}
\usepackage[dvipsnames, table, xcdraw]{xcolor} % Tables support
%\usepackage{zed-csp} % cchemas support

%% Formating
\usepackage{authblk} % to add authors in maketitle
%\usepackage{blindtext} % to gen filler text
\usepackage[figurename=Fig.]{caption} % to change prefix of the image caption

%\usepackage{apacite}
\usepackage{cite} % useful to compress multiple quotations into a single entry
\usepackage{enumitem}
\usepackage{fancyhdr} % to set page style
\usepackage{indentfirst}
\usepackage{../nasm/lang}  % include custom language for NASM assembly.
\usepackage{../nasm/style} % include custom style for NASM assembly.
%\usepackage{natbib}
\usepackage{parskip} % remove first line tabulation
\usepackage{setspace}
%\usepackage{titlesec}
%\usepackage{titling} % to config maketitle

%% Variables
% Main images
\newcommand{\logoUdg}{logo-udg.jpg}
\newcommand{\logoCucei}{logo-cucei.jpg}

% Figures
\newcommand{\figA}{./img/1-start.jpg}
\newcommand{\figB}{./img/2-creating_and_opening.jpg}
\newcommand{\figC}{./img/3-writing.jpg}
\newcommand{\figD}{./img/4-attributes.jpg}
\newcommand{\figE}{./img/5-closing.jpg}
\newcommand{\figF}{./img/6-removing.jpg}

% School data
\newcommand{\universidad}{Universidad de Guadalajara}
\newcommand{\cede}{Centro Universitario de Ciencias Exactas e Ingenierías}

% Subject data
\newcommand{\materia}{Programación de Bajo Nivel}
\newcommand{\carrera}{Ingeniería en Computación}
\newcommand{\division}{División de Tecnologías para la Integración CiberHumana}
\newcommand{\theTitle}{4. Manipulación de Archivos}
\newcommand{\profesor}{José Juan Meza Espinoza}
\newcommand{\seccion}{D02}
\newcommand{\nrc}{209850}
\newcommand{\clave}{IL358}
\newcommand{\startDate}{20 de octubre de 2024}

% Author data
\newcommand{\theAuthor}{Alan Yahir Juárez Rubio}
\newcommand{\theAuthorCode}{218517809}
\newcommand{\theAuthorMail}{alan.juarez5178@alumnos.udg.mx}

%% Declaration
\date{}
\graphicspath{ {../../../img/} }
\addto\captionsspanish{\renewcommand{\contentsname}{Índice}}
\renewcommand{\lstlistingname}{Código} % to change prefix of the code caption
\renewcommand{\lstlistlistingname}{Índice de códigos} % to change listings index title

%% Styles

% Color declaration
\definecolor{greenPortada}{HTML}{69A84F}
\definecolor{LightGray}{gray}{0.9}
\definecolor{codegreen}{rgb}{0, 0.6, 0}
\definecolor{codegray}{rgb}{0.5, 0.5, 0.5}
\definecolor{codepurple}{rgb}{0.58, 0, 0.82}
\definecolor{backcolour}{rgb}{0.95, 0.95, 0.92}

% Hyperlinks
\hypersetup{
    colorlinks=true,
    linkcolor=black,
    filecolor=greenPortada,
    urlcolor=greenPortada,
    pdfpagemode=FullScreen,
}

\urlstyle{same}

% Codeblocks
\lstdefinestyle{mystyle}{
	backgroundcolor=\color{backcolour},
	commentstyle=\color{codegreen},
	keywordstyle=\color{magenta},
	numberstyle=\tiny\color{codegray},
	stringstyle=\color{codepurple},
	basicstyle=\ttfamily\footnotesize,
	breakatwhitespace=false,
	breaklines=true,
	captionpos=b,
	keepspaces=true,
	numbers=left,
	numbersep=5pt,
	showspaces=false,
	showstringspaces=false,
	showtabs=false,
	tabsize=4
}

% Tables
\let\oldtabular\tabular
\renewcommand{\tabular}{\small\oldtabular}
\renewcommand{\arraystretch}{1.1} % <-- Adjust vertical spacing
\addto\captionsspanish{\renewcommand{\tablename}{Tabla}}

\lstset{style=mystyle}

%% Listings

\lstset{
  literate={á}{{\'a}}1 {é}{{\'e}}1 {í}{{\'i}}1 {ó}{{\'o}}1 {ú}{{\'u}}1
           {Á}{{\'A}}1 {É}{{\'E}}1 {Í}{{\'I}}1 {Ó}{{\'O}}1 {Ú}{{\'U}}1
           {ñ}{{\~n}}1 {Ñ}{{\~N}}1
}

%% Spacing
\newcommand{\nl}{\par
\vspace{0.4cm}}
\renewcommand{\baselinestretch}{1.5} % Espaciado de línea anterior
\setlength{\parskip}{6pt} % Espaciado de línea anterior
\setlength{\parindent}{0pt} % Sangría

% Header and footer
\pagestyle{fancy}
\fancyhf{}
\renewcommand{\headrulewidth}{3pt}
\renewcommand{\headrule}{\hbox to\headwidth{\color{greenPortada}\leaders\hrule height \headrulewidth\hfill}}
\setlength{\headheight}{50pt} % Ajuste necesario para evitar warnings

% Header
\setlength{\headheight}{59.9055pt}
\addtolength{\topmargin}{-9.9055pt}

\lhead{
	\begin{minipage}[c][2cm][c]{1.3cm}
		\begin{flushleft}
			\includegraphics[width=5cm, height=1.4cm, keepaspectratio]{\logoUdg}
		\end{flushleft}
	\end{minipage}
	\begin{minipage}[c][2cm][c]{0.5\textwidth} % Adjust the height as needed
		\begin{flushleft}
			{\materia}
		\end{flushleft}
	\end{minipage}
}

\rhead{
	\begin{minipage}[c][2cm][c]{0.4\textwidth} % Adjust the height as needed
		\begin{flushright}
			{\theTitle}
		\end{flushright}
	\end{minipage}
	\begin{minipage}[c][2cm][c]{1.3cm}
		\begin{flushright}
			\includegraphics[width=5cm, height=1.4cm, keepaspectratio]{\logoCucei}
		\end{flushright}
	\end{minipage}
}

% Footer
\fancyfoot{}
\setlength{\footskip}{35.27028pt}

\lfoot{
	\begin{minipage}[c][2cm][c]{0.4\textwidth} % Adjust the height as needed
		\begin{flushleft}
			{\small Elaborado por \theAuthor}
		\end{flushleft}
	\end{minipage}
}

\cfoot{\thepage} % Paginación

\rfoot{
	\begin{minipage}[c][2cm][c]{0.4\textwidth} % Adjust the height as needed
		\begin{flushright}
			{\small Curso impartido por \profesor}
		\end{flushright}
	\end{minipage}
}

%% Title

\title{\fontsize{24}{28.8}\selectfont \theTitle}
\author{\theAuthor}

\affil{}


\begin{document}
	\setstretch{1} % Interlineado

	\begin{titlepage}
		\newgeometry{margin=2.5cm, left=3cm, right=3cm} % change margin
		\centering
		%\vspace*{-2cm}
		{\LARGE \textbf{\universidad}}\par
		\vspace{0.6cm}
		{\Large{\cede}}
		\vfill

		\begin{figure}[ht]
			\begin{minipage}[t]{0.45\textwidth}
				\centering
				\includegraphics[width=130px, height=160px, keepaspectratio]{\logoUdg}
			\end{minipage}
			\hfill
			\begin{minipage}[t]{0.45\textwidth}
				\centering
				\includegraphics[width=130px, height=160px, keepaspectratio]{\logoCucei}
			\end{minipage}
		\end{figure}
		\vfill

		\large{
			\division\vfill
			\textbf{\carrera}\vfill
			\textbf{\materia}\par\vspace{3pt}
			\seccion\ - \clave\ - \nrc\ - \generation \vfill
		}

		{\Large{\textbf{\theTitle}}}
		\vfill

		\begin{figure}[ht]
			\centering
			\begin{minipage}[t]{0.65\textwidth}
				{\large
					\textbf{Profesor}: \profesor\nl
					\textbf{Alumno}: \theAuthor\nl
					\textbf{Código}: \theAuthorCode\nl
					\textbf{Correo}: \theAuthorMail
				}
			\end{minipage}
		\end{figure}
		\vfill

		\begin{tcolorbox}
			[colback=red!5!white, colframe=red!75!black]
			\centering
			\large{
				Este documento contiene información sensible.\\
				No debería ser impreso o compartido con terceras entidades.
			}
		\end{tcolorbox}
		\vfill
		{\large \startDate}\par
	\end{titlepage}

	\restoregeometry % end changed margin

	%% Indexes
	\clearpage
	\tableofcontents

	%\clearpage
	%\listoffigures

	%\clearpage
	%\listoftables

	%\clearpage
	%\lstlistoflistings

	%% Main Title
	\clearpage
	\vspace*{-16pt}
	\begin{center}
		{\textbf{\huge \theTitle}}
	\end{center}
	\vspace*{8pt}

	%% Content

	\section*{Indicaciones}

	Elaborar hoja y media como extensión mínimo para el desarrollo de estos puntos
	utilizar arial 12. pagina 203

	\clearpage
	\section{Innovación Mediante la Experimentación Rápida}

	La \textbf{innovación} no es más que el proceso que conlleva la realización
	o actualización de productos y servicios, los cuales deben resultar en impactos
	positivos.

	El \textit{proceso de innovación} consiste principalmente en transformar ideas
	creativas en resultados que mejoren la eficiencia o eficacia o responda a
	necesidades insatisfechas.

	En la actualidad, la forma más práctica de innovar es mediante la \textbf{
	experimentación rápida}, debido a que permite desarrollar o actualizar
	productos y servicios y determinar los resultados de una manera rápida y
	barata. Esta se divide en dos tipos:

	\begin{itemize}
		\item Experimentación convergente o experimentación formal
		\item Experimentación divergente o experimentación informal
	\end{itemize}

	\clearpage
	\section{Método Experimental Divergente}

	El \textbf{método experimental divergente}, es útil para las innovaciones menos
	definidas desde el principio (e.g. productos, servicios y procesos comerciales
	nuevos de una organización).

	Es relevante mencionar que los proyectos de innovación que utilizan esta
	metodología tienden a ser altamente iterativos y pueden durar semanas o meses.

	El \textit{método experimental divergente} se divide en diez pasos:

	\begin{enumerate}
		\item Definir el problema
		\item Establecer límites
		\item Escoger a la gente
		\item Observar
		\item Generar más de una solución
		\item Construir un MVP (Prototipo Mínimo Viable)
		\item Prueba de campo
		\item Decidir
		\item Escalar
		\item Compartir
	\end{enumerate}

	\subsection{Definir el Problema}

	Para comenzar un \textit{experimento divergente} primeramente es necesario
	definir el problema que se desea resolver. El problema debe estar basado en una
	necesidad observada del cliente o en una oportunidad de mercado y debe ser un
	desafío que la organización esté particularmente capacitada para resolver.

	La definición del problema puede incluir un objetivo cuantificado, pero ese
	objetivo debe ser a la vez desafiante y amplio.

	\subsection{Establecer Límites}

	Aquí se determinan los parámetros que enmarcarán la experimentación. Esto
	incluye establecer restricciones como el tiempo, los recursos disponibles y los
	criterios de éxito. Los límites ayudan a mantener el enfoque en las soluciones
	más prácticas y realistas, evitando que los experimentos se extiendan
	indefinidamente o se vuelvan inmanejables.

	\subsection{Escoger a la Gente}

	El último paso de la fase de preparación es elegir qué personas trabajarán en
	tu experimento de innovación. Este equipo debe estar compuesto por individuos
	con habilidades diversas y perspectivas complementarias. La diversidad del
	equipo fomenta la generación de ideas innovadoras y asegura que las soluciones
	propuestas consideren múltiples puntos de vista.

	\subsection{Observar}

	En este paso, se recopila información detallada sobre el problema y su
	contexto. Esto puede implicar observar directamente el comportamiento de los
	usuarios, analizar datos o investigar tendencias relevantes. La observación
	profunda permite descubrir necesidades ocultas, patrones clave y oportunidades
	de mejora que las soluciones deben abordar.

	\subsection{Generar Más de Una Solución}

	El método divergente enfatiza la importancia de no limitarse a una única
	solución desde el principio. En esta etapa, el equipo desarrolla múltiples
	ideas o enfoques para resolver el problema. La variedad de opciones amplía las
	posibilidades de encontrar una solución efectiva y fomenta la creatividad en el
	proceso.

	\subsection{Construir un MVP (Prototipo Mínimo Viable)}

	Cada solución propuesta se convierte en un prototipo básico que puede ser
	probado rápidamente. Un MVP permite evaluar la viabilidad de una idea con una
	inversión mínima de tiempo y recursos. El enfoque está en crear algo funcional
	que capture la esencia de la solución sin desarrollarla completamente.

	\subsection{Prueba de Campo}

	Los MVP se someten a pruebas en condiciones reales o en entornos controlados
	que reflejen el uso previsto. Este paso implica recopilar datos y comentarios
	sobre cómo las soluciones funcionan en la práctica. La prueba de campo
	proporciona información valiosa sobre la eficacia de cada solución y cómo
	podría mejorarse.

	\subsection{Decidir}

	Con base en los resultados de las pruebas, se toma una decisión informada sobre
	cuál de las soluciones experimentadas tiene el mayor potencial. Este paso
	incluye analizar datos y retroalimentación para seleccionar la opción que mejor
	aborda el problema definido inicialmente.

	\subsection{Escalar}

	Una vez que se ha identificado una solución ganadora, el siguiente paso es
	ampliarla para su implementación en un contexto más amplio. Esto puede implicar
	optimizar el diseño, desarrollar funcionalidades adicionales o invertir en
	recursos para desplegarla a gran escala.

	\subsection{Compartir}

	Finalmente, el conocimiento y las lecciones aprendidas durante todo el proceso
	se documentan y comparten dentro de la organización. Este paso asegura que la
	experiencia acumulada a lo largo del método experimental divergente se
	convierta en una base para futuros proyectos, fomentando una cultura de
	aprendizaje continuo e innovación.

	%% References

	\nocite{*} % to include uncited references of .bib file

	\clearpage
	\bibliographystyle{apalike}

	% Generated from .bib file
	\bibliography{ref}
\end{document}
