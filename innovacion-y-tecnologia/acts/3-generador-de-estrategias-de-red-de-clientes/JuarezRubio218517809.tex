\documentclass[11pt, a4paper]{article} % Formato

% Language and font encodings
\usepackage[spanish]{babel}
\usepackage[utf8]{inputenc}
\usepackage[T1]{fontenc}
\usepackage{times} % Times New Roman

%% Sets page size and margins
\usepackage[margin=2.5cm, includefoot]{geometry}
%\setlength{\columnsep}{0.17in} % page columns separation

%% Useful packages
\usepackage{amsmath}
\usepackage{array} % <-- add this line for m{} column type
\usepackage[hidelinks]{hyperref} % hyperlinks support
\usepackage{graphicx} % images support
\usepackage{listings} % codeblock support
%\usepackage{smartdiagram} % diagrams support
\usepackage[most]{tcolorbox} % callouts support
%\usepackage[colorinlistoftodos]{todonotes}
\usepackage[dvipsnames, table, xcdraw]{xcolor} % Tables support
%\usepackage{zed-csp} % cchemas support

%% Formating
\usepackage{authblk} % to add authors in maketitle
%\usepackage{blindtext} % to gen filler text
\usepackage[figurename=Fig.]{caption} % to change prefix of the image caption

%\usepackage{apacite}
\usepackage{cite} % useful to compress multiple quotations into a single entry
\usepackage{enumitem}
\usepackage{fancyhdr} % to set page style
\usepackage{indentfirst}
\usepackage{../nasm/lang}  % include custom language for NASM assembly.
\usepackage{../nasm/style} % include custom style for NASM assembly.
%\usepackage{natbib}
\usepackage{parskip} % remove first line tabulation
\usepackage{setspace}
%\usepackage{titlesec}
%\usepackage{titling} % to config maketitle

%% Variables
% Main images
\newcommand{\logoUdg}{logo-udg.jpg}
\newcommand{\logoCucei}{logo-cucei.jpg}

% Figures
\newcommand{\figA}{./img/1-start.jpg}
\newcommand{\figB}{./img/2-creating_and_opening.jpg}
\newcommand{\figC}{./img/3-writing.jpg}
\newcommand{\figD}{./img/4-attributes.jpg}
\newcommand{\figE}{./img/5-closing.jpg}
\newcommand{\figF}{./img/6-removing.jpg}

% School data
\newcommand{\universidad}{Universidad de Guadalajara}
\newcommand{\cede}{Centro Universitario de Ciencias Exactas e Ingenierías}

% Subject data
\newcommand{\materia}{Programación de Bajo Nivel}
\newcommand{\carrera}{Ingeniería en Computación}
\newcommand{\division}{División de Tecnologías para la Integración CiberHumana}
\newcommand{\theTitle}{4. Manipulación de Archivos}
\newcommand{\profesor}{José Juan Meza Espinoza}
\newcommand{\seccion}{D02}
\newcommand{\nrc}{209850}
\newcommand{\clave}{IL358}
\newcommand{\startDate}{20 de octubre de 2024}

% Author data
\newcommand{\theAuthor}{Alan Yahir Juárez Rubio}
\newcommand{\theAuthorCode}{218517809}
\newcommand{\theAuthorMail}{alan.juarez5178@alumnos.udg.mx}

%% Declaration
\date{}
\graphicspath{ {../../../img/} }
\addto\captionsspanish{\renewcommand{\contentsname}{Índice}}
\renewcommand{\lstlistingname}{Código} % to change prefix of the code caption
\renewcommand{\lstlistlistingname}{Índice de códigos} % to change listings index title

%% Styles

% Color declaration
\definecolor{greenPortada}{HTML}{69A84F}
\definecolor{LightGray}{gray}{0.9}
\definecolor{codegreen}{rgb}{0, 0.6, 0}
\definecolor{codegray}{rgb}{0.5, 0.5, 0.5}
\definecolor{codepurple}{rgb}{0.58, 0, 0.82}
\definecolor{backcolour}{rgb}{0.95, 0.95, 0.92}

% Hyperlinks
\hypersetup{
    colorlinks=true,
    linkcolor=black,
    filecolor=greenPortada,
    urlcolor=greenPortada,
    pdfpagemode=FullScreen,
}

\urlstyle{same}

% Codeblocks
\lstdefinestyle{mystyle}{
	backgroundcolor=\color{backcolour},
	commentstyle=\color{codegreen},
	keywordstyle=\color{magenta},
	numberstyle=\tiny\color{codegray},
	stringstyle=\color{codepurple},
	basicstyle=\ttfamily\footnotesize,
	breakatwhitespace=false,
	breaklines=true,
	captionpos=b,
	keepspaces=true,
	numbers=left,
	numbersep=5pt,
	showspaces=false,
	showstringspaces=false,
	showtabs=false,
	tabsize=4
}

% Tables
\let\oldtabular\tabular
\renewcommand{\tabular}{\small\oldtabular}
\renewcommand{\arraystretch}{1.1} % <-- Adjust vertical spacing
\addto\captionsspanish{\renewcommand{\tablename}{Tabla}}

\lstset{style=mystyle}

%% Listings

\lstset{
  literate={á}{{\'a}}1 {é}{{\'e}}1 {í}{{\'i}}1 {ó}{{\'o}}1 {ú}{{\'u}}1
           {Á}{{\'A}}1 {É}{{\'E}}1 {Í}{{\'I}}1 {Ó}{{\'O}}1 {Ú}{{\'U}}1
           {ñ}{{\~n}}1 {Ñ}{{\~N}}1
}

%% Spacing
\newcommand{\nl}{\par
\vspace{0.4cm}}
\renewcommand{\baselinestretch}{1.5} % Espaciado de línea anterior
\setlength{\parskip}{6pt} % Espaciado de línea anterior
\setlength{\parindent}{0pt} % Sangría

% Header and footer
\pagestyle{fancy}
\fancyhf{}
\renewcommand{\headrulewidth}{3pt}
\renewcommand{\headrule}{\hbox to\headwidth{\color{greenPortada}\leaders\hrule height \headrulewidth\hfill}}
\setlength{\headheight}{50pt} % Ajuste necesario para evitar warnings

% Header
\setlength{\headheight}{59.9055pt}
\addtolength{\topmargin}{-9.9055pt}

\lhead{
	\begin{minipage}[c][2cm][c]{1.3cm}
		\begin{flushleft}
			\includegraphics[width=5cm, height=1.4cm, keepaspectratio]{\logoUdg}
		\end{flushleft}
	\end{minipage}
	\begin{minipage}[c][2cm][c]{0.5\textwidth} % Adjust the height as needed
		\begin{flushleft}
			{\materia}
		\end{flushleft}
	\end{minipage}
}

\rhead{
	\begin{minipage}[c][2cm][c]{0.4\textwidth} % Adjust the height as needed
		\begin{flushright}
			{\theTitle}
		\end{flushright}
	\end{minipage}
	\begin{minipage}[c][2cm][c]{1.3cm}
		\begin{flushright}
			\includegraphics[width=5cm, height=1.4cm, keepaspectratio]{\logoCucei}
		\end{flushright}
	\end{minipage}
}

% Footer
\fancyfoot{}
\setlength{\footskip}{35.27028pt}

\lfoot{
	\begin{minipage}[c][2cm][c]{0.4\textwidth} % Adjust the height as needed
		\begin{flushleft}
			{\small Elaborado por \theAuthor}
		\end{flushleft}
	\end{minipage}
}

\cfoot{\thepage} % Paginación

\rfoot{
	\begin{minipage}[c][2cm][c]{0.4\textwidth} % Adjust the height as needed
		\begin{flushright}
			{\small Curso impartido por \profesor}
		\end{flushright}
	\end{minipage}
}

%% Title

\title{\fontsize{24}{28.8}\selectfont \theTitle}
\author{\theAuthor}

\affil{}


\begin{document}
	\setstretch{1} % Interlineado

	\begin{titlepage}
		\newgeometry{margin=2.5cm, left=3cm, right=3cm} % change margin
		\centering
		%\vspace*{-2cm}
		{\huge\textbf{\universidad}}\par
		\vspace{0.6cm}
		{\LARGE{\cede}}
		\vfill

		\begin{figure}[h]
			\begin{minipage}[t]{0.45\textwidth}
				\centering
				\includegraphics[width=130px, height=160px, keepaspectratio]{\logoUdg}
			\end{minipage}
			\hfill
			\begin{minipage}[t]{0.45\textwidth}
				\centering
				\includegraphics[width=130px, height=160px, keepaspectratio]{\logoCucei}
			\end{minipage}
		\end{figure}
		\vfill

		\Large{
			\division\vfill
			\textbf{\carrera}\vfill
			\textbf{\materia}\par\vspace{3pt}
			\seccion\ - \clave\ - \nrc\ - \generation \vfill
		}

		{\LARGE{\textbf{\theTitle}}}
		\vfill

		\begin{figure}[h]
			\centering
			\begin{minipage}[t]{0.60\textwidth}
				{\Large
					\textbf{Profesor}: \profesor\nl
					\textbf{Alumno}: \theAuthor\nl
					\textbf{Código}: \theAuthorCode\nl
					\textbf{Correo}: \theAuthorMail
				}
			\end{minipage}
		\end{figure}
		\vfill

		\begin{tcolorbox}
			[colback=red!5!white, colframe=red!75!black]
			\centering
			Este documento contiene información sensible.\\
			No debería ser impreso o compartido con terceras entidades.
		\end{tcolorbox}
		\vfill
		{\large \startDate}\par
	\end{titlepage}

	\restoregeometry % end changed margin

	%% Indexes
	\clearpage
	\tableofcontents

	%\clearpage
	%\listoffigures

	%\clearpage
	%\listoftables

	%\clearpage
	%\lstlistoflistings

	%% Main Title
	\clearpage
	\vspace*{6pt}
	\centerline{\textbf{\huge \theTitle}}
	\vspace*{8pt}

	%% Content
	\section{Establecer Objetivos}

	\subsection{Objetivos Específicos}

	Desarrollo de plataformas digitales (e.g. sitios web) que permitan:

	\begin{itemize}
		\item Crear una comunidad, recibir retroalimentación de los
			clientes, ofrecer y publicitar nuestros servicios, y dar a
			conocer a la empresa.

		\item Incorparar clientes minoristas al modelo de negocio y,
			además, publicitar e informar a los distintos clientes sobre los
			diferentes servicios con los que cuenta la empresa.
	\end{itemize}

	\subsection{Objetivos de Orden Superior}

	\begin{itemize}
		\item Implementar nuevos modelos de servicios para los distintos tipos
			de clientes y atención personalizada.

		\item Actualizar los procesos existentes e implementar nuevos procesos
			que permitan llevar a cabo cada uno de los servicios que se plantea
			que provea la empresa a cada uno de los diferentes clientes (i.e.
			minoristas, mayoristas y socios).
	\end{itemize}

	\section{Selección y Enfoque del Cliente}

	La obtención de cada uno de los objetivos planteados en la sección
	anterior implica:

	\begin{itemize}
		\item Desarrollar un nuevo sistema de atención para los clientes minoristas
			y, además establecimiento de los diferentes servicios, procesos
			y respectivos costos que se llevarán a cabo.

		\item Desarrollar un sistema de feedback, para que cada uno de
			los diferentes clientes pueda calificarnos, dar su opinión y darnos
			recomendaciones de cómo mejorar nuestros servicios.

		\item Implementación de nuevas metodologías de trabajo con el fin
			de mejorar los procesos que permiten llevar a cabo cada uno de los
			servicios que se proveen actualmente a los socios y mayoristas.
	\end{itemize}

	Nuestra empresa busca garantizar servicios de reparación y mantenmiento
	de equipos móviles y sus respectivos accesorios, especialmente a
	equipos de la marca Samsung, asegurándole a nuestros clientes de llevar
	a cabo estándares de calidad de servicio y de cumplir con las fechas
	de entrega establecidas.

	El cumplimiento de cada uno de estos objetivos implica un gran desafío
	para nuestra empresa debido a que desafortunadamente no se cuenta con
	personal lo suficientemente capacitado para mudar nuestros procesos
	a nuevos sistemas y a plataformas digitales.

	Asímismo, la implementación de nuevos procesos, adquisición de nuevos
	clientes, entre otras metas, significa un gran reto para nuestra empresa
	debido a que practicamente se estaría redefiniendo cada uno de los aspectos
	de la empresa, con el fin de mejorar y garantizar nuestros
	servicios y expandir y mejorar nuestra empresa.

	\section{Selección de Estrategia}

	\subsection{Acceso}

	\begin{itemize}
		\item \textbf{Plataformas Web y Móvil}: El desarrollo de estas plataformas
			ayudarían a tener un mayor alcance a clientes, de tal manera
			que los clientes puedan tener una mayor información acerca de
			los servicios de reparación y, adicionalmente, un sistema más
			controlado con el que puedan solicitar los servicios.

		\item \textbf{Cedes}: El desarrollo de nuevas cedes hace que el servicio
			sea más accesible por más usuarios, de tal manera que los
			usuarios les es más sencillo ir a visitar alguna de las cedes.
	\end{itemize}

	\subsection{Compromiso}

	\begin{itemize}
		\item \textbf{Retroalimentación de Usuario}: La implementación de
			un usuario para que pueda dar retroalimentación acerca del servicio
			otorgado garantiza una mayor confiabilidad hacia los demás usuarios.

		\item \textbf{Videoblog a través de redes sociales}: Un videoblog
			en el que uno o varios técnicos muestren en vivo las reparaciones
			de los equipos no solo ayuda a que el usuario observe el proceso
			y tenga referencias, sino también le garantiza una mayor
			confiabilidad y transparencia respecto a la empresa.
	\end{itemize}

	\subsection{Personalización}

	\begin{itemize}
		\item \textbf{Servicio Personalizado}: La notificación del usuario
			acerca del proceso de reparación, ya sea por mensajes, notificaciones
			o cualquier otro medio, brinda al usuario una atención más
			amigable y personalizada, manteniendo al usuario informado en todo
			momento.

		\item \textbf{Servicio Personalizado}: El ofrecer un accesorio gratuito
			(incluído con el servicio) a selección del usuario (e.g.~funda,
			mica) no solo ofrece un servicio más personalizado, sino
			también ofrece una mejor atención al cliente.
	\end{itemize}

	\subsection{Conexión}

	\begin{itemize}
		\item \textbf{Foro}: La implementación de un foro en las \emph{plataformas
			web y movil} ayuda a que los clientes puedan comunicarse entre
			sí, tanto para hablar de su experiencia con el servicio,
			resolver dudas y dar información extra

		\item \textbf{Redes Sociales}: El uso de redes sociales de la empresa
			es de suma importancia porque, además de crear una marca
			personal como empresa, ayuda a crear una comunidad propia en la
			que los clientes pueden compartir experiencias, ideas y demás.
	\end{itemize}

	\subsection{Colaboración}

	\begin{itemize}
		\item \textbf{Sección de Sugerencias}: Un sección de sugerencias (pública
			o privada) da la posibilidad de que el cilente pueda colaborar de
			tal manera que puede dar ideas para mejorar el servicio y la experiencia
			del usuario.

		\item \textbf{Asociasiones con negocios}: La posibilidad de que negocios
			más pequeños puedan asociarse ya sea como cliente mayorista,
			pasado de clientes (e.g.~que la empresa ofrece reparaciones de
			computadoras) o de alguna manera, ayuda a crecer y expandir la
			empresa, no solo reduciendo la competitividad sino incrementando
			la colaboración.
	\end{itemize}

	\section{Generador de Conceptos}

	Para el desarrollo de la plataforma para ofrecer el servicio a los
	clientes minoristas es necesario definir una serie de
	requerimientos:

	\begin{itemize}
		\item El sistema permitirá a los usuarios registrarse e iniciar sesión
			mediante formularios.

		\item El usuario, en primera instancia, se le ofrecerá un diagnóstico
			(sin costos de envío) para determinar el problema y el estado del
			equipo

		\item El sistema les presentará a los usuarios los posibles presupuestos
			para cada una de las distintas reparaciones.

		\item El usuario recibirá mensajes o notificaciones para informarle
			acerca del estado del diagnóstico o reparación.

		\item En caso de haber diagnosticado un equipo, el usuario podrá
			optar por continuar o rechazar la reparación.
	\end{itemize}
\end{document}
