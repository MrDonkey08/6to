\documentclass[11pt, a4paper]{article} % Formato

% Language and font encodings
\usepackage[spanish]{babel}
\usepackage[utf8]{inputenc}
\usepackage[T1]{fontenc}
\usepackage{times} % Times New Roman

%% Sets page size and margins
\usepackage[margin=2.5cm, includefoot]{geometry}
%\setlength{\columnsep}{0.17in} % page columns separation

%% Useful packages
\usepackage{amsmath}
\usepackage{array} % <-- add this line for m{} column type
\usepackage[hidelinks]{hyperref} % hyperlinks support
\usepackage{graphicx} % images support
\usepackage{listings} % codeblock support
%\usepackage{smartdiagram} % diagrams support
\usepackage[most]{tcolorbox} % callouts support
%\usepackage[colorinlistoftodos]{todonotes}
\usepackage[dvipsnames, table, xcdraw]{xcolor} % Tables support
%\usepackage{zed-csp} % cchemas support

%% Formating
\usepackage{authblk} % to add authors in maketitle
%\usepackage{blindtext} % to gen filler text
\usepackage[figurename=Fig.]{caption} % to change prefix of the image caption

%\usepackage{apacite}
\usepackage{cite} % useful to compress multiple quotations into a single entry
\usepackage{enumitem}
\usepackage{fancyhdr} % to set page style
\usepackage{indentfirst}
\usepackage{../nasm/lang}  % include custom language for NASM assembly.
\usepackage{../nasm/style} % include custom style for NASM assembly.
%\usepackage{natbib}
\usepackage{parskip} % remove first line tabulation
\usepackage{setspace}
%\usepackage{titlesec}
%\usepackage{titling} % to config maketitle

%% Variables
% Main images
\newcommand{\logoUdg}{logo-udg.jpg}
\newcommand{\logoCucei}{logo-cucei.jpg}

% Figures
\newcommand{\figA}{./img/1-start.jpg}
\newcommand{\figB}{./img/2-creating_and_opening.jpg}
\newcommand{\figC}{./img/3-writing.jpg}
\newcommand{\figD}{./img/4-attributes.jpg}
\newcommand{\figE}{./img/5-closing.jpg}
\newcommand{\figF}{./img/6-removing.jpg}

% School data
\newcommand{\universidad}{Universidad de Guadalajara}
\newcommand{\cede}{Centro Universitario de Ciencias Exactas e Ingenierías}

% Subject data
\newcommand{\materia}{Programación de Bajo Nivel}
\newcommand{\carrera}{Ingeniería en Computación}
\newcommand{\division}{División de Tecnologías para la Integración CiberHumana}
\newcommand{\theTitle}{4. Manipulación de Archivos}
\newcommand{\profesor}{José Juan Meza Espinoza}
\newcommand{\seccion}{D02}
\newcommand{\nrc}{209850}
\newcommand{\clave}{IL358}
\newcommand{\startDate}{20 de octubre de 2024}

% Author data
\newcommand{\theAuthor}{Alan Yahir Juárez Rubio}
\newcommand{\theAuthorCode}{218517809}
\newcommand{\theAuthorMail}{alan.juarez5178@alumnos.udg.mx}

%% Declaration
\date{}
\graphicspath{ {../../../img/} }
\addto\captionsspanish{\renewcommand{\contentsname}{Índice}}
\renewcommand{\lstlistingname}{Código} % to change prefix of the code caption
\renewcommand{\lstlistlistingname}{Índice de códigos} % to change listings index title

%% Styles

% Color declaration
\definecolor{greenPortada}{HTML}{69A84F}
\definecolor{LightGray}{gray}{0.9}
\definecolor{codegreen}{rgb}{0, 0.6, 0}
\definecolor{codegray}{rgb}{0.5, 0.5, 0.5}
\definecolor{codepurple}{rgb}{0.58, 0, 0.82}
\definecolor{backcolour}{rgb}{0.95, 0.95, 0.92}

% Hyperlinks
\hypersetup{
    colorlinks=true,
    linkcolor=black,
    filecolor=greenPortada,
    urlcolor=greenPortada,
    pdfpagemode=FullScreen,
}

\urlstyle{same}

% Codeblocks
\lstdefinestyle{mystyle}{
	backgroundcolor=\color{backcolour},
	commentstyle=\color{codegreen},
	keywordstyle=\color{magenta},
	numberstyle=\tiny\color{codegray},
	stringstyle=\color{codepurple},
	basicstyle=\ttfamily\footnotesize,
	breakatwhitespace=false,
	breaklines=true,
	captionpos=b,
	keepspaces=true,
	numbers=left,
	numbersep=5pt,
	showspaces=false,
	showstringspaces=false,
	showtabs=false,
	tabsize=4
}

% Tables
\let\oldtabular\tabular
\renewcommand{\tabular}{\small\oldtabular}
\renewcommand{\arraystretch}{1.1} % <-- Adjust vertical spacing
\addto\captionsspanish{\renewcommand{\tablename}{Tabla}}

\lstset{style=mystyle}

%% Listings

\lstset{
  literate={á}{{\'a}}1 {é}{{\'e}}1 {í}{{\'i}}1 {ó}{{\'o}}1 {ú}{{\'u}}1
           {Á}{{\'A}}1 {É}{{\'E}}1 {Í}{{\'I}}1 {Ó}{{\'O}}1 {Ú}{{\'U}}1
           {ñ}{{\~n}}1 {Ñ}{{\~N}}1
}

%% Spacing
\newcommand{\nl}{\par
\vspace{0.4cm}}
\renewcommand{\baselinestretch}{1.5} % Espaciado de línea anterior
\setlength{\parskip}{6pt} % Espaciado de línea anterior
\setlength{\parindent}{0pt} % Sangría

% Header and footer
\pagestyle{fancy}
\fancyhf{}
\renewcommand{\headrulewidth}{3pt}
\renewcommand{\headrule}{\hbox to\headwidth{\color{greenPortada}\leaders\hrule height \headrulewidth\hfill}}
\setlength{\headheight}{50pt} % Ajuste necesario para evitar warnings

% Header
\setlength{\headheight}{59.9055pt}
\addtolength{\topmargin}{-9.9055pt}

\lhead{
	\begin{minipage}[c][2cm][c]{1.3cm}
		\begin{flushleft}
			\includegraphics[width=5cm, height=1.4cm, keepaspectratio]{\logoUdg}
		\end{flushleft}
	\end{minipage}
	\begin{minipage}[c][2cm][c]{0.5\textwidth} % Adjust the height as needed
		\begin{flushleft}
			{\materia}
		\end{flushleft}
	\end{minipage}
}

\rhead{
	\begin{minipage}[c][2cm][c]{0.4\textwidth} % Adjust the height as needed
		\begin{flushright}
			{\theTitle}
		\end{flushright}
	\end{minipage}
	\begin{minipage}[c][2cm][c]{1.3cm}
		\begin{flushright}
			\includegraphics[width=5cm, height=1.4cm, keepaspectratio]{\logoCucei}
		\end{flushright}
	\end{minipage}
}

% Footer
\fancyfoot{}
\setlength{\footskip}{35.27028pt}

\lfoot{
	\begin{minipage}[c][2cm][c]{0.4\textwidth} % Adjust the height as needed
		\begin{flushleft}
			{\small Elaborado por \theAuthor}
		\end{flushleft}
	\end{minipage}
}

\cfoot{\thepage} % Paginación

\rfoot{
	\begin{minipage}[c][2cm][c]{0.4\textwidth} % Adjust the height as needed
		\begin{flushright}
			{\small Curso impartido por \profesor}
		\end{flushright}
	\end{minipage}
}

%% Title

\title{\fontsize{24}{28.8}\selectfont \theTitle}
\author{\theAuthor}

\affil{}


\begin{document}
	\setstretch{1} % Interlineado

	\begin{titlepage}
		\newgeometry{margin=2.5cm, left=3cm, right=3cm} % change margin
		\centering
		%\vspace*{-2cm}
		{\LARGE \textbf{\universidad}}\par
		\vspace{0.6cm}
		{\Large{\cede}}
		\vfill

		\begin{figure}[ht]
			\begin{minipage}[t]{0.45\textwidth}
				\centering
				\includegraphics[width=130px, height=160px, keepaspectratio]{\logoUdg}
			\end{minipage}
			\hfill
			\begin{minipage}[t]{0.45\textwidth}
				\centering
				\includegraphics[width=130px, height=160px, keepaspectratio]{\logoCucei}
			\end{minipage}
		\end{figure}
		\vfill

		\large{
			\division\vfill
			\textbf{\carrera}\vfill
			\textbf{\materia}\par\vspace{3pt}
			\seccion\ - \clave\ - \nrc\ - \generation \vfill
		}

		{\Large{\textbf{\theTitle}}}
		\vfill

		\begin{figure}[ht]
			\centering
			\begin{minipage}[t]{0.6\textwidth}
				{\large
					\textbf{Profesor}: \profesor\nl
					\textbf{Alumno}: \theAuthor\nl
					\textbf{Código}: \theAuthorCode\nl
					\textbf{Correo}: \theAuthorMail
				}
			\end{minipage}
		\end{figure}
		\vfill

		\begin{tcolorbox}
			[colback=red!5!white, colframe=red!75!black]
			\centering
			\large{
				Este documento contiene información sensible.\\
				No debería ser impreso o compartido con terceras entidades.
			}
		\end{tcolorbox}
		\vfill
		{\large \startDate}\par
	\end{titlepage}

	\restoregeometry % end changed margin

	%% Indexes
	%\clearpage
	%\tableofcontents

	%\clearpage
	%\listoffigures

	%\clearpage
	%\listoftables

	%\clearpage
	%\lstlistoflistings

	%% Main Title
	\clearpage
	\vspace*{-16pt}
	\begin{center}
		{\textbf{\huge \theTitle}}
	\end{center}
	\vspace*{8pt}

	%% Content
	\subsection*{Indicaciones}

	De la página 160 en delante se pueden leer más a detalle cada uno de los
	puntos que tienen que desarrollar para el generador de valor de datos, recuerden
	que, en conjunto (los 5 puntos) deben desarrollar mínimo dos hojas de
	información, ya sea de preguntas (que aparezcan en el libro) o en forma de resumen
	con letra arial 12. Iterlineado maximo (doble).

	\clearpage
	\section{El Generador de Valor de Datos: Una Herramienta Estratégica para la Transformación
	Digital}
	En la era de la transformación digital, las organizaciones enfrentan el
	desafío de convertir los datos en un recurso estratégico que impulse el
	crecimiento, la eficiencia y la innovación. Este enfoque se estructura en
	cinco pasos fundamentales: \textbf{definición del área de impacto y los KPIs},
	\textbf{selección de la plantilla de valor}, \textbf{generación del concepto},
	\textbf{auditoría de datos} y \textbf{plan de ejecución}. A continuación, exploramos
	cómo cada paso contribuye al desarrollo de una estrategia de datos exitosa.

	\subsection{Definición del Área de Impacto e Indicadores Clave de Rendimiento
	(KPIs)}

	El primer paso en el Generador de Valor de Datos es identificar las áreas del
	negocio donde el uso de datos puede tener mayor impacto. Es aquí donde se definen
	las prioridades estratégicas, ya sea optimizar la eficiencia operativa, mejorar
	la experiencia del cliente o aumentar los ingresos. Este análisis permite
	establecer \textbf{indicadores clave de rendimiento (KPIs)} específicos y medibles
	que alineen los esfuerzos de datos con los objetivos organizacionales. Por
	ejemplo, un KPI podría ser reducir los tiempos de entrega en un 20\% o
	aumentar la tasa de conversión de clientes en un 15\%. La claridad en este
	paso inicial asegura que todo el proyecto de datos se enfoque en resultados
	tangibles y relevantes.

	\subsection{Selección de la Plantilla de Valor}

	Una vez definida el área de impacto, el siguiente paso es elegir una plantilla
	de valor adecuada. Rogers presenta cinco modelos principales para crear valor con
	datos: \textbf{agilización de procesos}, \textbf{mejora de la experiencia del
	cliente}, \textbf{innovación en productos o servicios}, \textbf{optimización
	de decisiones} y \textbf{monetización de datos}. Cada plantilla ofrece un enfoque
	distinto, desde aumentar la eficiencia interna hasta generar nuevas fuentes de
	ingresos. La elección correcta depende de las prioridades y el modelo de negocio
	de la organización. Este marco proporciona claridad y dirección, asegurando que
	la estrategia de datos esté alineada con las metas corporativas.

	\subsection{Generación del Concepto}

	Con la plantilla de valor seleccionada, el siguiente paso es traducirla en un concepto
	claro y práctico. Aquí se combinan los objetivos estratégicos con la plantilla
	de valor para desarrollar una idea inicial de cómo los datos se utilizarán en
	la práctica. Por ejemplo, si se seleccionó la optimización de decisiones como
	plantilla de valor, el concepto podría ser la implementación de un sistema de analítica
	predictiva para gestionar inventarios en tiempo real. Este paso conecta la
	estrategia con la operación, permitiendo que la visión general se traduzca en una
	iniciativa concreta y realizable.

	\subsection{Auditoría de Datos}

	Ningún proyecto de datos puede avanzar sin una evaluación rigurosa de la
	calidad y disponibilidad de los datos existentes. La auditoría de datos es un paso
	crítico para identificar fuentes de datos internas y externas, evaluar su
	relevancia, precisión y accesibilidad, y detectar posibles brechas. Este análisis
	proporciona un diagnóstico de la infraestructura de datos de la organización y
	señala las áreas que requieren mejora. Por ejemplo, una auditoría podría revelar
	que los datos del cliente están desactualizados o que existen limitaciones en la
	integración de sistemas. Este paso asegura que los datos sean fiables y útiles
	para apoyar el concepto generado.

	\subsection{Plan de Ejecución}

	Finalmente, el Generador de Valor de Datos culmina en la creación de un plan
	de ejecución detallado. Este plan incluye un roadmap con prioridades, roles
	definidos, plazos específicos y recursos asignados. También considera aspectos
	como la gestión del cambio organizacional y el diseño de un ciclo iterativo
	para implementar, evaluar y ajustar la solución. Un ejemplo de un plan de
	ejecución eficaz podría ser lanzar un proyecto piloto en una unidad de negocio
	antes de escalarlo a toda la organización. Este enfoque estructurado garantiza
	que la estrategia no se quede en una idea abstracta, sino que se convierta en
	acciones concretas y medibles.

	\clearpage
	\section{Aplicación del Generador de Valor de Datos}

	En el competitivo mundo de la tecnología, las empresas deben adaptarse
	constantemente para mantenerse relevantes. En este contexto, una empresa
	dedicada a la reparación de celulares Samsung, que actualmente trabaja con grandes
	clientes como Samsung, Elektra y Telcel, busca innovar ofreciendo servicios también
	a clientes minoristas. Para lograrlo, se propone desarrollar una plataforma digital
	(web y móvil) que permita a estos consumidores acceder fácilmente a reparaciones
	y accesorios. Este ensayo explora cómo el \textbf{Generador de Valor de Datos},
	propuesto por David L. Rogers, puede guiar este proceso de innovación mediante
	cinco pasos estratégicos: definir áreas de impacto y KPIs, seleccionar la
	plantilla de valor, generar un concepto, realizar una auditoría de datos y
	ejecutar un plan.

	\subsection{Área de Impacto e Indicadores Clave de Rendimiento (KPIs)}

	La primera etapa consiste en identificar el impacto que se busca lograr con esta
	transformación. En este caso, la prioridad es expandir el mercado hacia los clientes
	minoristas y mejorar su experiencia. Mientras que la empresa ya domina el
	segmento B2B, la incursión en el mercado B2C requiere un enfoque centrado en el
	cliente. Para medir el éxito de esta iniciativa, se establecen indicadores clave
	de rendimiento (KPIs) específicos, como aumentar en un 30\% los ingresos anuales
	provenientes de clientes minoristas, registrar 5,000 nuevos usuarios en la
	plataforma durante los primeros seis meses, y reducir en un 20\% el tiempo
	promedio de gestión de reparaciones mediante la automatización. Estas metas proporcionan
	un marco claro para medir el impacto de la innovación.

	\subsection{Selección de la Plantilla de Valor}

	El siguiente paso es elegir la plantilla de valor que mejor se alinee con los
	objetivos de la empresa. Dada la naturaleza del proyecto, se seleccionan dos
	enfoques: \textbf{mejorar la experiencia del cliente} y \textbf{agilizar los
	procesos operativos}. Por un lado, la creación de una plataforma que permita a
	los clientes registrar solicitudes, realizar pagos y seguir el progreso de sus
	reparaciones en tiempo real promete una experiencia fluida y confiable. Por otro
	lado, la automatización de procesos internos, como la asignación de técnicos y
	la actualización de inventarios, permitirá a la empresa manejar mayores
	volúmenes de solicitudes de manera eficiente. Esta combinación asegura que tanto
	los clientes como la operación interna se beneficien de la transformación.

	\subsection{Generación del Concepto}

	El concepto que surge de esta estrategia es una **plataforma digital integral
	que sirva como punto de contacto para los clientes minoristas. En ella, los
	usuarios podrán crear cuentas personalizadas, gestionar sus solicitudes de reparación,
	realizar pagos en línea y recibir actualizaciones en tiempo real sobre el
	estado de sus equipos. Además, incluirá un foro para resolver preguntas
	frecuentes y obtener soporte directo. Este concepto no solo facilita el acceso
	a los servicios, sino que también posiciona a la empresa como una opción
	moderna y confiable en el sector de reparaciones tecnológicas. La plataforma será
	accesible tanto desde dispositivos móviles como desde navegadores web,
	garantizando una experiencia inclusiva y adaptable.

	\subsection{Auditoría de Datos}

	Para implementar este concepto, es esencial evaluar la calidad y
	disponibilidad de los datos existentes. Una auditoría de datos permitirá analizar
	fuentes internas, como el historial de reparaciones, los tiempos promedio de servicio
	y la disponibilidad de piezas proporcionadas por Samsung. También será
	necesario explorar fuentes externas, como tendencias de mercado y opiniones de
	clientes potenciales. Sin embargo, es probable que la empresa identifique brechas
	significativas en su conocimiento sobre las necesidades y expectativas del cliente
	minorista, ya que este segmento no ha sido atendido previamente. La recopilación
	de estos datos, a través de encuestas o proyectos piloto, será crucial para personalizar
	la plataforma y optimizar su funcionalidad.

	\subsection{Plan de Ejecución}

	El desarrollo de la plataforma se llevará a cabo en fases escalonadas. Durante
	los primeros seis meses, se enfocará en la creación de un prototipo funcional,
	con características básicas como la gestión de cuentas, la solicitud de
	servicios y la integración de pagos en línea. Una vez desarrollado, se lanzará
	un programa piloto en una región limitada para recopilar feedback de los
	usuarios y ajustar la plataforma según sus necesidades. En una etapa posterior,
	la empresa escalará la solución a nivel nacional, incorporando mejoras como programas
	de fidelidad y promociones exclusivas para clientes recurrentes. Este plan de
	ejecución está diseñado para garantizar una transición fluida y sostenible
	hacia el nuevo modelo de negocio, al tiempo que minimiza riesgos.

	\clearpage
	\subsection*{Conclusión}

	El Generador de Valor de Datos de David L. Rogers es una herramienta poderosa
	para cualquier organización que busque maximizar el impacto de sus datos. A través
	de sus cinco pasos, permite a las empresas conectar la estrategia con la
	ejecución, enfocándose en resultados tangibles y alineados con los objetivos corporativos.
	En un mundo impulsado por datos, este enfoque no solo es esencial, sino
	también una ventaja competitiva indispensable para la transformación digital.

	%% References

	\nocite{*} % to include uncited references of .bib file

	\clearpage
	\bibliographystyle{apalike}

	% Generated from .bib file
	\bibliography{ref}
\end{document}
