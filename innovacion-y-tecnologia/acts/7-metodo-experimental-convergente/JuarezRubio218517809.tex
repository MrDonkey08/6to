\documentclass[11pt, a4paper]{article} % Formato

% Language and font encodings
\usepackage[spanish]{babel}
\usepackage[utf8]{inputenc}
\usepackage[T1]{fontenc}
\usepackage{times} % Times New Roman

%% Sets page size and margins
\usepackage[margin=2.5cm, includefoot]{geometry}
%\setlength{\columnsep}{0.17in} % page columns separation

%% Useful packages
\usepackage{amsmath}
\usepackage{array} % <-- add this line for m{} column type
\usepackage[hidelinks]{hyperref} % hyperlinks support
\usepackage{graphicx} % images support
%\usepackage{listings} % codeblock support
%\usepackage{smartdiagram} % diagrams support
\usepackage[most]{tcolorbox} % callouts support
%\usepackage[colorinlistoftodos]{todonotes}
\usepackage[dvipsnames, table, xcdraw]{xcolor} % Tables support
%\usepackage{zed-csp} % cchemas support

%% Formating
\usepackage{authblk} % to add authors in maketitle
%\usepackage{blindtext} % to gen filler text
\usepackage[figurename=Fig.]{caption} % to change prefix of the image caption

%\usepackage{apacite}
\usepackage{cite} % useful to compress multiple quotations into a single entry
\usepackage{enumitem}
\usepackage{fancyhdr} % to set page style
\usepackage{indentfirst}
%\usepackage{natbib}
\usepackage{parskip} % remove first line tabulation
\usepackage{setspace}
%\usepackage{titlesec}
%\usepackage{titling} % to config maketitle

%% Variables
% Main images
\newcommand{\logoUdg}{logo-udg.jpg}
\newcommand{\logoCucei}{logo-cucei.jpg}
\newcommand{\newAmazonSection}{./img/productos-nuevos.jpg}

% School data
\newcommand{\universidad}{Universidad de Guadalajara}
\newcommand{\cede}{Centro Universitario de Ciencias Exactas e Ingenierías}

% Subject data
\newcommand{\materia}{Interacción Humano Computadora}
\newcommand{\carrera}{Ingeniería en Computación}
\newcommand{\division}{División de Tecnologías para la Integración CiberHumana}
\newcommand{\theTitle}{3. Comprender la Toma de Deciciones Humanas}
\newcommand{\profesor}{José Luis David Bonilla Carranza}
\newcommand{\seccion}{D01}
\newcommand{\nrc}{209754}
\newcommand{\clave}{IL367}
\newcommand{\startDate}{20 de septiembre de 2024}

% Author data
\newcommand{\theAuthor}{Juárez Rubio Alan Yahir}
\newcommand{\theAuthorCode}{218517809}
\newcommand{\theAuthorMail}{alan.juarez5178@alumnos.udg.mx}

%% Declaration
\date{}
\graphicspath{ {../../../img/} }
\addto\captionsspanish{\renewcommand{\contentsname}{Índice}}
\renewcommand{\lstlistingname}{Código} % to change prefix of the code caption
\renewcommand{\lstlistlistingname}{Índice de códigos} % to change listings index title

%% Styles

% Color declaration
\definecolor{greenPortada}{HTML}{69A84F}
\definecolor{LightGray}{gray}{0.9}
\definecolor{codegreen}{rgb}{0, 0.6, 0}
\definecolor{codegray}{rgb}{0.5, 0.5, 0.5}
\definecolor{codepurple}{rgb}{0.58, 0, 0.82}
\definecolor{backcolour}{rgb}{0.95, 0.95, 0.92}

% Hyperlinks
\hypersetup{
    colorlinks=true,
    linkcolor=black,
    filecolor=greenPortada,
    urlcolor=greenPortada,
    pdfpagemode=FullScreen,
}

\urlstyle{same}

% Codeblocks
\lstdefinestyle{mystyle}{
	backgroundcolor=\color{backcolour},
	commentstyle=\color{codegreen},
	keywordstyle=\color{magenta},
	numberstyle=\tiny\color{codegray},
	stringstyle=\color{codepurple},
	basicstyle=\ttfamily\footnotesize,
	breakatwhitespace=false,
	breaklines=true,
	captionpos=b,
	keepspaces=true,
	numbers=left,
	numbersep=5pt,
	showspaces=false,
	showstringspaces=false,
	showtabs=false,
	tabsize=2
}

% Tables
\let\oldtabular\tabular
\renewcommand{\tabular}{\small\oldtabular}
\renewcommand{\arraystretch}{1.2} % <-- Adjust vertical spacing
\addto\captionsspanish{\renewcommand{\tablename}{Tabla}}

\lstset{style=mystyle}

%% Spacing
\newcommand{\nl}{\par
\vspace{0.4cm}}
\renewcommand{\baselinestretch}{1.5} % Espaciado de línea anterior
\setlength{\parskip}{6pt} % Espaciado de línea anterior
\setlength{\parindent}{0pt} % Sangría

% Header and footer
\pagestyle{fancy}
\fancyhf{}
\renewcommand{\headrulewidth}{3pt}
\renewcommand{\headrule}{\hbox to\headwidth{\color{greenPortada}\leaders\hrule height \headrulewidth\hfill}}
\setlength{\headheight}{50pt} % Ajuste necesario para evitar warnings

% Header
\pagestyle{fancy}
\fancyhf{}
\lhead{
	\begin{minipage}[c][2cm][c]{1.3cm}
		\begin{flushleft}
			\includegraphics[width=5cm, height=1.4cm, keepaspectratio]{\logoUdg}
		\end{flushleft}
	\end{minipage}
	\begin{minipage}[c][2cm][c]{0.5\textwidth} % Adjust the height as needed
		\begin{flushleft}
			{\materia}
		\end{flushleft}
	\end{minipage}
}

\rhead{
	\begin{minipage}[c][2cm][c]{0.4\textwidth} % Adjust the height as needed
		\begin{flushright}
			{\theTitle}
		\end{flushright}
	\end{minipage}
	\begin{minipage}[c][2cm][c]{1.3cm}
		\begin{flushright}
			\includegraphics[width=5cm, height=1.4cm, keepaspectratio]{\logoCucei}
		\end{flushright}
	\end{minipage}
}

% Footer
\fancyfoot{}
\lfoot{\small\materia}
\cfoot{\thepage} % Paginación
\rfoot{\small Curso impartido por \profesor}

%% Title

\title{\fontsize{24}{28.8}\selectfont \theTitle}
\author{\theAuthor}

\affil{}


\begin{document}
\setstretch{1} % Interlineado

\begin{titlepage}
	\newgeometry{margin=2.5cm, left=3cm, right=3cm} % change margin
	\centering
	%\vspace*{-2cm}
	{\LARGE \textbf{\universidad}}\par
	\vspace{0.6cm}
	{\Large{\cede}}
	\vfill

	\begin{figure}[ht]
		\begin{minipage}[t]{0.45\textwidth}
			\centering
			\includegraphics[width=130px, height=160px, keepaspectratio]{\logoUdg}
		\end{minipage}
		\hfill
		\begin{minipage}[t]{0.45\textwidth}
			\centering
			\includegraphics[width=130px, height=160px, keepaspectratio]{\logoCucei}
		\end{minipage}
	\end{figure}
	\vfill

	\large{
		\division\vfill
		\textbf{\carrera}\vfill
		\textbf{\materia}\par\vspace{3pt}
		\seccion\ - \clave\ - \nrc\ - \generation \vfill
	}

	{\Large{\textbf{\theTitle}}}
	\vfill

	\begin{figure}[ht]
		\centering
		\begin{minipage}[t]{0.65\textwidth}
			{\large
				\textbf{Profesor}: \profesor\nl
				\textbf{Alumno}: \theAuthor\nl
				\textbf{Código}: \theAuthorCode\nl
				\textbf{Correo}: \theAuthorMail
			}
		\end{minipage}
	\end{figure}
	\vfill

	\begin{tcolorbox}
		[colback=red!5!white, colframe=red!75!black]
		\centering
		\large{
			Este documento contiene información sensible.\\
			No debería ser impreso o compartido con terceras entidades.
		}
	\end{tcolorbox}
	\vfill
	{\large \startDate}\par
\end{titlepage}

\restoregeometry % end changed margin

%% Indexes
\clearpage
\tableofcontents

%\clearpage
%\listoffigures

%\clearpage
%\listoftables

%\clearpage
%\lstlistoflistings

%% Main Title
\clearpage
\vspace*{-16pt}
\begin{center}
	{\textbf{\huge \theTitle}}
\end{center}
\vspace*{8pt}

%% Content

Para proyecto final se debe entregar un documento en el que se recopilen
todas las actividades vistas durante el semestre en orden cronológico, a parte
de las actividades el documento debe tener un texto de conexión entre cada
actividad, es decir, no contara sino solo se ponen las actividades, se debe de
poner un texto que interconecte a cada una de ellas, de tal forma en que se
pueda leer un texto corrido y se entienda que se está viendo en cada mapa o
figura que se muestre. Se pueden incluir descripciones de figuras y mapas.

Debe incluir:


Índice.

Arial 12.

Interlineado 1.5 líneas.


Margen Superior, inferior y derecho de 2.5 cm.


Margen izquierdo 3 cm.

Texto justificado.

Extensión mínima 17 paginas.


Lineamientos de tareas.

\section*{Indicaciones}

Elaborar hoja y media como extensión mínimo para el desarrollo de estos puntos
(19. Innovar mediante la experimentacion rapida 3), utilizar arial 12. pagina
197.

\clearpage
\section{Innovación Mediante la Experimentación Rápida}

La \textbf{innovación} no es más que el proceso que conlleva la realización
o actualización de productos y servicios, los cuales deben resultar en impactos
positivos.

El \textit{proceso de innovación} consiste principalmente en transformar ideas
creativas en resultados que mejoren la eficiencia o eficacia o responda a
necesidades insatisfechas.

En la actualidad, la forma más práctica de innovar es mediante la \textbf{
	experimentación rápida}, debido a que permite desarrollar o actualizar
productos y servicios y determinar los resultados de una manera rápida y
barata. Esta se divide en dos tipos:

\begin{itemize}
	\item Experimentación convergente o experimentación formal
	\item Experimentación divergente o experimentación informal
\end{itemize}

\clearpage
\section{Método Experimental Convergente}

El \textbf{método experimental convergente}	es útil para innovar productos,
servicios y procesos existentes, para optimizarlos y mejorarlos
constantemente. Este método se destaca debido a que pueden ser realizados
rápidamente, en cuestión de horas y minutos.

El \textit{método experimental convergente} se divide en siete pasos:

\begin{enumerate}
	\item Definir la pregunta y sus variables
	\item Elegir los probadores (\textit{testers})
	\item Aleatorizar la prueba y el control
	\item Validar la muestra
	\item Probar y analizar
	\item Decidir
	\item Compartir el aprendizaje
\end{enumerate}

\subsection{Definir la Pregunta y Sus Variables}

Para iniciar un \textit{experimento convergente} es necesario definir la
pregunta que se busca responder. Esta pregunta debe ser tan específica como sea
posible.

Una vez planteada la pregunta, se tiene que traducir en dos tipos de variables

\begin{itemize}
	\item \textbf{Variable independiente (causa)}: Factor que se probará en el
	      experimento. El objetivo de este experimento es entender el efecto de la
	      introducción de esta innovación.

	\item \textbf{Variable dependiente (efecto)}: Factor en el que se espera que
	      la innovación pueda influir. Es una medida del impacto de lo que se está
	      cambiando.
\end{itemize}

\subsection{Elegir los Probadores (\textit{Testers})}

Este paso consiste en seleccionar los sujetos de prueba del experimento.
Es importante, pero no estrictamente necesario tener dos tipos de probadores:

\begin{itemize}
	\item \textbf{Probadores de caja blanca}: Personas que son parte del
	      desarrollo del experimento y, por ende, tienen conocimiento acerca del
	      funcionamiento interno del experimento.

	\item \textbf{Probadores de caja negra}: Personas que no son parte del
	      desarrollo del experimento y, por ende, no tienen conocimiento acerca del
	      funcionamiento interno del experimento.
\end{itemize}


El objetivo de las pruebas es recabar la suficiente información acerca del
experimento para su posterior análisis y toma de decisiones.

\subsection{Aleatorizar la Prueba y el Control}

Antes de llevar a cabo un experimento convergente, es imporante identificar una
población cuyas repuestas deseas probar. Seguidamente asignas al azar miembros
de esa población a uno de estos dos grupos:

\begin{itemize}
	\item El grupo de prueba (o grupo de tratamiento), que realiza la experiencia
	      o recibe la oferta que estás probando.

	\item El grupo de control, al que no se aplica el factor testeado.
\end{itemize}

\subsection{Validar la Muestra}

Este paso consisten en asegurarse que se cuenta con un tamaño de muestra
válido. Para ello, primeramente es necesario identificar la unidad de análisis
(i.e. producto, servicio o proceso).

Seguidamente, se establece el tamaño de la muestra (i.e, número de unidades que
se coloca a cada uno de los grupos de prueba), la regla típica es $n = 100$,
como mínimo, en cada grupo que se compara.

\subsection{Probar y Analizar}

En este punto uno ya se encuentra listo para realizar la prueba. El equipo que
lleve a cabo tu experimento recogerá datos durante un período predeterminado.
Luego tendrá que analizar los datos para ver si hay diferencias en las
variables dependientes que se están midiendo y, si las hay, comprobar si esas
diferencias son estadísticamente significativas.

\subsection{Decidir}

Luego de haber analizado los resultados del experimento, es importante tomar
una decisión basada en las conclusiones:

\begin{itemize}
	\item \textbf{Abandonar el experimento}: Existen ocasiones en donde la mejor
	      alternativa es abandonar debido a que es inviable seguir llevándolo a cabo
	      (i.e. por tiempo, costos u otros factores).

	\item \textbf{Mejorar el experimento}: Iterar el experimento y realizar
	      pruebas para analizar si hay mejoras respecto a previas iteraciones.

	\item \textbf{Finalizar el experimento}: Se finaliza el experimento una vez
	      que haya sido recabada la información suficiente para determinar la respuesta
	      a la pregunta planteada desde un inicio.
\end{itemize}

\subsection{Compartir el Aprendizaje}

Una vez finalizado el proyecto, es fundamental documentar lo que se haya
aprendido, esto con el objetivo de comunicar los resultados para que otras
personas de la misma organización que podrían beneficiarse.

%% References

\nocite{*} % to include uncited references of .bib file

\clearpage
\bibliographystyle{apalike}

% Generated from .bib file
\bibliography{ref}
\end{document}
