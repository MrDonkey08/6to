\documentclass[11pt, a4paper]{article} % Formato

% Language and font encodings
\usepackage[spanish]{babel}
\usepackage[utf8]{inputenc}
\usepackage[T1]{fontenc}
\usepackage{times} % Times New Roman

%% Sets page size and margins
\usepackage[margin=2.5cm, includefoot]{geometry}
%\setlength{\columnsep}{0.17in} % page columns separation

%% Useful packages
\usepackage{amsmath}
\usepackage{array} % <-- add this line for m{} column type
\usepackage[hidelinks]{hyperref} % hyperlinks support
\usepackage{graphicx} % images support
\usepackage{listings} % codeblock support
%\usepackage{smartdiagram} % diagrams support
\usepackage[most]{tcolorbox} % callouts support
%\usepackage[colorinlistoftodos]{todonotes}
\usepackage[dvipsnames, table, xcdraw]{xcolor} % Tables support
%\usepackage{zed-csp} % cchemas support

%% Formating
\usepackage{authblk} % to add authors in maketitle
%\usepackage{blindtext} % to gen filler text
\usepackage[figurename=Fig.]{caption} % to change prefix of the image caption

%\usepackage{apacite}
\usepackage{cite} % useful to compress multiple quotations into a single entry
\usepackage{enumitem}
\usepackage{fancyhdr} % to set page style
\usepackage{indentfirst}
\usepackage{../nasm/lang}  % include custom language for NASM assembly.
\usepackage{../nasm/style} % include custom style for NASM assembly.
%\usepackage{natbib}
\usepackage{parskip} % remove first line tabulation
\usepackage{setspace}
%\usepackage{titlesec}
%\usepackage{titling} % to config maketitle

%% Variables
% Main images
\newcommand{\logoUdg}{logo-udg.jpg}
\newcommand{\logoCucei}{logo-cucei.jpg}

% Figures
\newcommand{\figA}{./img/1-start.jpg}
\newcommand{\figB}{./img/2-creating_and_opening.jpg}
\newcommand{\figC}{./img/3-writing.jpg}
\newcommand{\figD}{./img/4-attributes.jpg}
\newcommand{\figE}{./img/5-closing.jpg}
\newcommand{\figF}{./img/6-removing.jpg}

% School data
\newcommand{\universidad}{Universidad de Guadalajara}
\newcommand{\cede}{Centro Universitario de Ciencias Exactas e Ingenierías}

% Subject data
\newcommand{\materia}{Programación de Bajo Nivel}
\newcommand{\carrera}{Ingeniería en Computación}
\newcommand{\division}{División de Tecnologías para la Integración CiberHumana}
\newcommand{\theTitle}{4. Manipulación de Archivos}
\newcommand{\profesor}{José Juan Meza Espinoza}
\newcommand{\seccion}{D02}
\newcommand{\nrc}{209850}
\newcommand{\clave}{IL358}
\newcommand{\startDate}{20 de octubre de 2024}

% Author data
\newcommand{\theAuthor}{Alan Yahir Juárez Rubio}
\newcommand{\theAuthorCode}{218517809}
\newcommand{\theAuthorMail}{alan.juarez5178@alumnos.udg.mx}

%% Declaration
\date{}
\graphicspath{ {../../../img/} }
\addto\captionsspanish{\renewcommand{\contentsname}{Índice}}
\renewcommand{\lstlistingname}{Código} % to change prefix of the code caption
\renewcommand{\lstlistlistingname}{Índice de códigos} % to change listings index title

%% Styles

% Color declaration
\definecolor{greenPortada}{HTML}{69A84F}
\definecolor{LightGray}{gray}{0.9}
\definecolor{codegreen}{rgb}{0, 0.6, 0}
\definecolor{codegray}{rgb}{0.5, 0.5, 0.5}
\definecolor{codepurple}{rgb}{0.58, 0, 0.82}
\definecolor{backcolour}{rgb}{0.95, 0.95, 0.92}

% Hyperlinks
\hypersetup{
    colorlinks=true,
    linkcolor=black,
    filecolor=greenPortada,
    urlcolor=greenPortada,
    pdfpagemode=FullScreen,
}

\urlstyle{same}

% Codeblocks
\lstdefinestyle{mystyle}{
	backgroundcolor=\color{backcolour},
	commentstyle=\color{codegreen},
	keywordstyle=\color{magenta},
	numberstyle=\tiny\color{codegray},
	stringstyle=\color{codepurple},
	basicstyle=\ttfamily\footnotesize,
	breakatwhitespace=false,
	breaklines=true,
	captionpos=b,
	keepspaces=true,
	numbers=left,
	numbersep=5pt,
	showspaces=false,
	showstringspaces=false,
	showtabs=false,
	tabsize=4
}

% Tables
\let\oldtabular\tabular
\renewcommand{\tabular}{\small\oldtabular}
\renewcommand{\arraystretch}{1.1} % <-- Adjust vertical spacing
\addto\captionsspanish{\renewcommand{\tablename}{Tabla}}

\lstset{style=mystyle}

%% Listings

\lstset{
  literate={á}{{\'a}}1 {é}{{\'e}}1 {í}{{\'i}}1 {ó}{{\'o}}1 {ú}{{\'u}}1
           {Á}{{\'A}}1 {É}{{\'E}}1 {Í}{{\'I}}1 {Ó}{{\'O}}1 {Ú}{{\'U}}1
           {ñ}{{\~n}}1 {Ñ}{{\~N}}1
}

%% Spacing
\newcommand{\nl}{\par
\vspace{0.4cm}}
\renewcommand{\baselinestretch}{1.5} % Espaciado de línea anterior
\setlength{\parskip}{6pt} % Espaciado de línea anterior
\setlength{\parindent}{0pt} % Sangría

% Header and footer
\pagestyle{fancy}
\fancyhf{}
\renewcommand{\headrulewidth}{3pt}
\renewcommand{\headrule}{\hbox to\headwidth{\color{greenPortada}\leaders\hrule height \headrulewidth\hfill}}
\setlength{\headheight}{50pt} % Ajuste necesario para evitar warnings

% Header
\setlength{\headheight}{59.9055pt}
\addtolength{\topmargin}{-9.9055pt}

\lhead{
	\begin{minipage}[c][2cm][c]{1.3cm}
		\begin{flushleft}
			\includegraphics[width=5cm, height=1.4cm, keepaspectratio]{\logoUdg}
		\end{flushleft}
	\end{minipage}
	\begin{minipage}[c][2cm][c]{0.5\textwidth} % Adjust the height as needed
		\begin{flushleft}
			{\materia}
		\end{flushleft}
	\end{minipage}
}

\rhead{
	\begin{minipage}[c][2cm][c]{0.4\textwidth} % Adjust the height as needed
		\begin{flushright}
			{\theTitle}
		\end{flushright}
	\end{minipage}
	\begin{minipage}[c][2cm][c]{1.3cm}
		\begin{flushright}
			\includegraphics[width=5cm, height=1.4cm, keepaspectratio]{\logoCucei}
		\end{flushright}
	\end{minipage}
}

% Footer
\fancyfoot{}
\setlength{\footskip}{35.27028pt}

\lfoot{
	\begin{minipage}[c][2cm][c]{0.4\textwidth} % Adjust the height as needed
		\begin{flushleft}
			{\small Elaborado por \theAuthor}
		\end{flushleft}
	\end{minipage}
}

\cfoot{\thepage} % Paginación

\rfoot{
	\begin{minipage}[c][2cm][c]{0.4\textwidth} % Adjust the height as needed
		\begin{flushright}
			{\small Curso impartido por \profesor}
		\end{flushright}
	\end{minipage}
}

%% Title

\title{\fontsize{24}{28.8}\selectfont \theTitle}
\author{\theAuthor}

\affil{}


\begin{document}
    \setstretch{1} % Interlineado

    \begin{titlepage}
        \newgeometry{margin=2.5cm, left=3cm, right=3cm} % change margin
        \centering
        %\vspace*{-2cm}
        {\huge\textbf{\universidad}}\par
        \vspace{0.6cm}
        {\LARGE{\cede}}
        \vfill

        \begin{figure}[h]
            \begin{minipage}[t]{0.45\textwidth}
                \centering
                \includegraphics[width=130px, height=160px, keepaspectratio]{\logoUdg}
            \end{minipage}
            \hfill
            \begin{minipage}[t]{0.45\textwidth}
                \centering
                \includegraphics[width=130px, height=160px, keepaspectratio]{\logoCucei}
            \end{minipage}
        \end{figure}
        \vfill

        \Large{ \division\vfill \textbf{\carrera}\vfill \textbf{\materia}\par\vspace{3pt} \seccion\ - \clave\ - \nrc\ - \generation \vfill }

        \begin{figure}[h]
            \centering
            \begin{minipage}[t]{0.75\textwidth}
                {\Large \textbf{Profesor}: \profesor\nl \textbf{Alumno}: \theAuthor\nl \textbf{Código}: \theAuthorCode\nl \textbf{Correo}: \theAuthorMail }
            \end{minipage}
        \end{figure}
        \vfill

        {\LARGE{\textbf{\theTitle}}}
        \vfill

        \begin{tcolorbox}
            [colback=red!5!white, colframe=red!75!black]
            \centering
            Este documento contiene información sensible.\\ No debería ser impreso o
            compartido con terceras entidades.
        \end{tcolorbox}
        \vfill
        {\large \startDate}\par
    \end{titlepage}

    \restoregeometry % end changed margin

    %% Indexes
    \clearpage
    \tableofcontents

    %\clearpage
    %\listoffigures

    %\clearpage
    %\listoftables

    %\clearpage
    %\lstlistoflistings

    %% Main Title
    \clearpage
    \vspace*{6pt}
    \centerline{\textbf{\huge \theTitle}}
    \vspace*{8pt}

    %% Content

    \section{Qué es Agiotech}

    Agiotech es una empresa creada en Enero de 2012, por un grupo de profesionales
    expertos en modelos eficientes de servicio "Post-Venta" orientado al mercado de
    electrónicos de consumo en la región de Latinoamérica, contando con más de 15 años de experiencia
    en el soporte a compañías consideradas dentro de las Top 25 OEM's en la Industria Electrónica.

    Agiotech es una empresa dedicada principalmente a la reparación de dispositivos y accesorios
    móviles. Manejan reparaciones de dispositivos de diversas marcas, tales como Huawei,
    Motorola y su principal fuerte, Samsung. Las reparaciones provenien de garantías de
    empresas como Elektra, Telcel y el mismo Samsung.

    Actualmente plantea ser el principal fuente de reparaciones de la empresa de Samsung en
    México, ofreciendo la mayor cantidad de reparaciones, en el menor tiempo cumpliendo los
    estándares de calidad.

    Si bien, hoy en día son una de las principales fuentes de reparación de Samsung, considero
    que hay mucho trabajo por delante y muchos puntos de mejora para poder cumplir con las
    expectativas de Samsung.

    \section{Áreas de Agiotech}

    La empresa de Agiotech se divide en varias áreas o departamentos de trabajo:

    \begin{itemize}
        \item \textbf{Ingreso}: Área en donde los capturistas ingresan la información de los
            dispositivos entrantes en los sistemas, asegurándose que haya concordancia con
            la información respecto a los dispositivos

        \item \textbf{Sistemas}: Área enfocada en el desarrollo y mantenimiento del sistema
            principal de la empresa. Este sistema es el encargado de llevar el registro de
            cada una de las operaciones dentro de la empresa, tales como la llegada de dispositivos,
            piezas, inventariado, uso, estatus, registro de procedimiento, embarcamiento,
            etc. Adicionalmente, esta área se encarga de resolver cualquier problema que
            se llegase a presentar dentro del sistema.

        \item \textbf{Recursos Humanos}: Recursos Humanos es el departamento encargado de conseguir
            personal nuevo y llevar a cabo el proceso de la entrevista. Adicionalmente
            manejan el control administrativo sobre los empleados.

        \item \textbf{Reparacines}: Esta área se del diagnóstico de los dispositivos de reparación,
            determinación de si un dispositivo entra en garantía, si es reparable o no. Si
            el dispositivo entra en garantía o reparación pagada, este es reparado cumpliendo
            con cada uno de los requisitos y acuerdos con el cliente o empresa.

        \item \textbf{Administración}: Este departamento a grandes ragos se encarga del manejo
            de recursos de la empresa, tanto de la salida como la entrada de dinero, es decir,
            el cómo se manejan las finanzas.

        \item \textbf{Almacen}: Área encagada de la recepción y entrega de paquetes,
            paquetes que contienen tanto dispositivos como piezas. En general, tienen la
            tarea principal del movimiento del inventario, registrando y entregando las piezas
            solicitadas a los técnicos correspondientes.
    \end{itemize}

    Si bien existen diversas áreas y departamentos y tienen un sistema muy completo para llevar
    a cabo garantías y reparaciones, cosidero que esta empresa tiene muchos aspectos que
    mejorar, aspectos como la búsqueda de personal calificado, mejores capacitaciones, uso
    de tecnologías más.

    Por otra parte, considero que esta empresa tiene muchísimas oportunidades de
    crecimiento y expansión. La creación de nuevas cedes, el manejo de más marcas, la reparación
    y venta de otros dispositivos, tales como dispositivos de cómputo, entre otros muchos aspectos.
\end{document}