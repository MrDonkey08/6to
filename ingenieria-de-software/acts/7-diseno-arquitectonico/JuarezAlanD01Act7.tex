\documentclass[11pt, a4paper]{article} % Formato

% Language and font encodings
\usepackage[spanish]{babel}
\usepackage[utf8]{inputenc}
\usepackage[T1]{fontenc}
\usepackage{times} % Times New Roman

%% Sets page size and margins
\usepackage[margin=2.5cm, includefoot]{geometry}
%\setlength{\columnsep}{0.17in} % page columns separation

%% Useful packages
\usepackage{amsmath}
\usepackage{array} % <-- add this line for m{} column type
\usepackage[hidelinks]{hyperref} % hyperlinks support
\usepackage{graphicx} % images support
\usepackage{listings} % codeblock support
%\usepackage{smartdiagram} % diagrams support
\usepackage[most]{tcolorbox} % callouts support
%\usepackage[colorinlistoftodos]{todonotes}
\usepackage[dvipsnames, table, xcdraw]{xcolor} % Tables support
%\usepackage{zed-csp} % cchemas support

%% Formating
\usepackage{authblk} % to add authors in maketitle
%\usepackage{blindtext} % to gen filler text
\usepackage[figurename=Fig.]{caption} % to change prefix of the image caption

%\usepackage{apacite}
\usepackage{cite} % useful to compress multiple quotations into a single entry
\usepackage{enumitem}
\usepackage{fancyhdr} % to set page style
\usepackage{indentfirst}
\usepackage{../nasm/lang}  % include custom language for NASM assembly.
\usepackage{../nasm/style} % include custom style for NASM assembly.
%\usepackage{natbib}
\usepackage{parskip} % remove first line tabulation
\usepackage{setspace}
%\usepackage{titlesec}
%\usepackage{titling} % to config maketitle

%% Variables
% Main images
\newcommand{\logoUdg}{logo-udg.jpg}
\newcommand{\logoCucei}{logo-cucei.jpg}

% Figures
\newcommand{\figA}{./img/1-start.jpg}
\newcommand{\figB}{./img/2-creating_and_opening.jpg}
\newcommand{\figC}{./img/3-writing.jpg}
\newcommand{\figD}{./img/4-attributes.jpg}
\newcommand{\figE}{./img/5-closing.jpg}
\newcommand{\figF}{./img/6-removing.jpg}

% School data
\newcommand{\universidad}{Universidad de Guadalajara}
\newcommand{\cede}{Centro Universitario de Ciencias Exactas e Ingenierías}

% Subject data
\newcommand{\materia}{Programación de Bajo Nivel}
\newcommand{\carrera}{Ingeniería en Computación}
\newcommand{\division}{División de Tecnologías para la Integración CiberHumana}
\newcommand{\theTitle}{4. Manipulación de Archivos}
\newcommand{\profesor}{José Juan Meza Espinoza}
\newcommand{\seccion}{D02}
\newcommand{\nrc}{209850}
\newcommand{\clave}{IL358}
\newcommand{\startDate}{20 de octubre de 2024}

% Author data
\newcommand{\theAuthor}{Alan Yahir Juárez Rubio}
\newcommand{\theAuthorCode}{218517809}
\newcommand{\theAuthorMail}{alan.juarez5178@alumnos.udg.mx}

%% Declaration
\date{}
\graphicspath{ {../../../img/} }
\addto\captionsspanish{\renewcommand{\contentsname}{Índice}}
\renewcommand{\lstlistingname}{Código} % to change prefix of the code caption
\renewcommand{\lstlistlistingname}{Índice de códigos} % to change listings index title

%% Styles

% Color declaration
\definecolor{greenPortada}{HTML}{69A84F}
\definecolor{LightGray}{gray}{0.9}
\definecolor{codegreen}{rgb}{0, 0.6, 0}
\definecolor{codegray}{rgb}{0.5, 0.5, 0.5}
\definecolor{codepurple}{rgb}{0.58, 0, 0.82}
\definecolor{backcolour}{rgb}{0.95, 0.95, 0.92}

% Hyperlinks
\hypersetup{
    colorlinks=true,
    linkcolor=black,
    filecolor=greenPortada,
    urlcolor=greenPortada,
    pdfpagemode=FullScreen,
}

\urlstyle{same}

% Codeblocks
\lstdefinestyle{mystyle}{
	backgroundcolor=\color{backcolour},
	commentstyle=\color{codegreen},
	keywordstyle=\color{magenta},
	numberstyle=\tiny\color{codegray},
	stringstyle=\color{codepurple},
	basicstyle=\ttfamily\footnotesize,
	breakatwhitespace=false,
	breaklines=true,
	captionpos=b,
	keepspaces=true,
	numbers=left,
	numbersep=5pt,
	showspaces=false,
	showstringspaces=false,
	showtabs=false,
	tabsize=4
}

% Tables
\let\oldtabular\tabular
\renewcommand{\tabular}{\small\oldtabular}
\renewcommand{\arraystretch}{1.1} % <-- Adjust vertical spacing
\addto\captionsspanish{\renewcommand{\tablename}{Tabla}}

\lstset{style=mystyle}

%% Listings

\lstset{
  literate={á}{{\'a}}1 {é}{{\'e}}1 {í}{{\'i}}1 {ó}{{\'o}}1 {ú}{{\'u}}1
           {Á}{{\'A}}1 {É}{{\'E}}1 {Í}{{\'I}}1 {Ó}{{\'O}}1 {Ú}{{\'U}}1
           {ñ}{{\~n}}1 {Ñ}{{\~N}}1
}

%% Spacing
\newcommand{\nl}{\par
\vspace{0.4cm}}
\renewcommand{\baselinestretch}{1.5} % Espaciado de línea anterior
\setlength{\parskip}{6pt} % Espaciado de línea anterior
\setlength{\parindent}{0pt} % Sangría

% Header and footer
\pagestyle{fancy}
\fancyhf{}
\renewcommand{\headrulewidth}{3pt}
\renewcommand{\headrule}{\hbox to\headwidth{\color{greenPortada}\leaders\hrule height \headrulewidth\hfill}}
\setlength{\headheight}{50pt} % Ajuste necesario para evitar warnings

% Header
\setlength{\headheight}{59.9055pt}
\addtolength{\topmargin}{-9.9055pt}

\lhead{
	\begin{minipage}[c][2cm][c]{1.3cm}
		\begin{flushleft}
			\includegraphics[width=5cm, height=1.4cm, keepaspectratio]{\logoUdg}
		\end{flushleft}
	\end{minipage}
	\begin{minipage}[c][2cm][c]{0.5\textwidth} % Adjust the height as needed
		\begin{flushleft}
			{\materia}
		\end{flushleft}
	\end{minipage}
}

\rhead{
	\begin{minipage}[c][2cm][c]{0.4\textwidth} % Adjust the height as needed
		\begin{flushright}
			{\theTitle}
		\end{flushright}
	\end{minipage}
	\begin{minipage}[c][2cm][c]{1.3cm}
		\begin{flushright}
			\includegraphics[width=5cm, height=1.4cm, keepaspectratio]{\logoCucei}
		\end{flushright}
	\end{minipage}
}

% Footer
\fancyfoot{}
\setlength{\footskip}{35.27028pt}

\lfoot{
	\begin{minipage}[c][2cm][c]{0.4\textwidth} % Adjust the height as needed
		\begin{flushleft}
			{\small Elaborado por \theAuthor}
		\end{flushleft}
	\end{minipage}
}

\cfoot{\thepage} % Paginación

\rfoot{
	\begin{minipage}[c][2cm][c]{0.4\textwidth} % Adjust the height as needed
		\begin{flushright}
			{\small Curso impartido por \profesor}
		\end{flushright}
	\end{minipage}
}

%% Title

\title{\fontsize{24}{28.8}\selectfont \theTitle}
\author{\theAuthor}

\affil{}


\begin{document}
    \setstretch{1} % Interlineado

    \begin{titlepage}
        \newgeometry{margin=2.5cm, left=3cm, right=3cm} % change margin
        \centering
        %\vspace*{-2cm}
        {\huge\textbf{\universidad}}\par
        \vspace{0.6cm}
        {\LARGE{\cede}}
        \vfill

        \begin{figure}[h]
            \begin{minipage}[t]{0.45\textwidth}
                \centering
                \includegraphics[width=130px, height=160px, keepaspectratio]{\logoUdg}
            \end{minipage}
            \hfill
            \begin{minipage}[t]{0.45\textwidth}
                \centering
                \includegraphics[width=130px, height=160px, keepaspectratio]{\logoCucei}
            \end{minipage}
        \end{figure}
        \vfill

        \Large{
			\division\vfill
			\textbf{\carrera}\vfill
			\textbf{\materia}
			\par\vspace{3pt}
			\seccion\ - \clave\ - \nrc\vfill
		}

        {\LARGE{\textbf{\theTitle}}}
        \vfill

        \begin{figure}[h]
            \centering
            \begin{minipage}[t]{0.61\textwidth}
                {\Large
					\textbf{Profesor}: \profesor\nl
					\textbf{Alumno}: \theAuthor\nl
					\textbf{Código}: \theAuthorCode\nl
					\textbf{Correo}: \theAuthorMail
				}
            \end{minipage}
        \end{figure}
        \vfill

        \begin{tcolorbox}
            [colback=red!5!white, colframe=red!75!black]
            \centering
			Este documento ha sido elaborado con fines estudiantiles.\\
			La información presentada puede contener errores.
        \end{tcolorbox}
        \vfill
        {\large \startDate}\par
    \end{titlepage}

    \restoregeometry % end changed margin

    %% Indexes
    \clearpage
    \tableofcontents

    %\clearpage
    %\listoffigures

    %\clearpage
    %\listoftables

    %\clearpage
    %\lstlistoflistings

    %% Main Title
    \clearpage
    \vspace*{-16pt}
	\begin{center}
		{\textbf{\huge \theTitle}}
	\end{center}
    \vspace*{8pt}

    %% Content

	\section{Introducción}

	El diseño arquitectónico nos permite definir cómo debe organizarse
	un sistema y cómo tiene que diseñarse la estructura global de esté.
	Es el enlace entre el diseño y la ingeniería de requerimientos, que
	identifica los principales componentes estructurales en un sistema
	y la relación entre ellos.

	\subsection{Conceptos de Arquitectura de Software}

	Responde a cada una de las preguntas propuestas:

	\begin{enumerate}
		\item ¿Qué es la arquitectura de software y por qué es importante
			en el desarrollo de aplicaciones?

			La \textbf{arquitectura del software} es la manera esencial en
			que los componentes de un sistemas se encuentran conectados
			entre sí o la organización fundamental de un sistema.

			La \textbf{arquitectura de software} es importante debido a que
			nos permite identificar cada uno de los componentes del sistema
			que serán esenciales para el éxito o fallo de nuestro sistema y
			el desarollo del sistema para servir y proteger dicho componentes
			esenciales.

		\item ¿Qué es un patrón de diseño arquitectónico y por qué son importantes
			en el desarrollo de software? Proporciona ejemplos de patrones
			de diseño comunes.

			Un \textbf{patrón arquitectónico} es una solución general y reutilizable
			a un problema común en la arquitectura de software dentro de un
			contexto dado. Los patrones arquitectónicos son similares al
			patrón de diseño de software, pero tienen un alcance más amplio.

			Los patrones arquitectónicos son importantes debido a que ofrecen
			soluciones a problemas de arquitectura de software, definiendo
			elementos y relaciones entre ellos, junto con restricciones sobre
			su uso.

			Algunos ejemplos de diseño arquitectónico son:

			\begin{itemize}
				\item \textbf{Cliente-Servidor}: Este patrón comprende dos partes,
					uno o varios servidores y múltiples clientes. Mientras los usuarios,
					a través de un sistema, solicitan servicios, recursos y
					demás al servidor, el servidor se encarga de suministrar dichas
					peticiones a dichos usuarios.

				\item \textbf{Capas}: Se puede utilizar para estructurar programas
					que se pueden descomponer en grupos de subtareas, cada una
					de las cuales se encuentra en un nivel particular de
					abstracción. Cada capa proporciona servicios a la siguiente
					capa superior.

				\item \textbf{MVC (Modelo Vista Controlador)}: Este modelo divide
					una aplicación interactiva en 3 partes:

					\begin{itemize}
						\item \textbf{Modelo}: Contiene la funcionalidad y los datos
							básicos

						\item \textbf{Vista}: Muestra la información al usuario

						\item \textbf{Controlador}: Maneja la entrada del usuario
					\end{itemize}
			\end{itemize}

		\item ¿Qué significa el principio de ``separación de preocupaciones''
			en la arquitectura de software? ¿Cómo se logra esto en la
			práctica?

			El principio de Separación de preocupaciones (SoC, por sus
			siglas en inglés) es una guía de diseño de software que se enfoca
			en separar los aspectos diferentes de un sistema en diferentes componentes
			para que cada uno se ocupe de una tarea específica y no haya mezcla
			de responsabilidades.

		\item Habla sobre la diferencia entre un ``framework'' y una ``biblioteca''
			en el contexto de la arquitectura de software. ¿Cómo pueden
			ayudar en el desarrollo de aplicaciones?

			\begin{itemize}
				\item \textbf{Framework}: Establece la estructura y el flujo de
					control de una aplicación. El desarrollador sigue sus
					reglas y el framework llama al código que este escribe (inversión
					de control). Es una solución completa, con convenciones
					predefinidas que aceleran el desarrollo, pero con menos flexibilidad.

					Los \textbf{frameworks} proporcionan estructura y buenas
					prácticas, facilitando aplicaciones organizadas y rápidas de
					crear.

				\item \textbf{Biblioteca}: A diferencia de un \emph{framework},
					la \textbf{biblioteca} es más flexible y modular. El desarrollador
					tiene el control total y decide cuándo usarla. Proporciona
					funciones específicas sin dictar la estructura general de
					la aplicación.

					Las \textbf{bibliotecas} permiten solucionar problemas puntuales
					con más control y personalización.
			\end{itemize}
	\end{enumerate}

	\subsection{Estilos Arquitectónicos}

	Para la resolución de los ejercicios que siguen a continuación sólo
	considere los estilos arquitectónicos: Basado en Capas, Cliente-Servidor,
	Pipes and Filters, Basado en Eventos, Peer-to-Peer.

	Para cada una de las siguientes situaciones, discuta qué estilo o estilos
	arquitectónicos resultan aplicables y justifica brevemente porque:

	\begin{enumerate}
		\item La aplicación necesita procesar la entrada en forma de flujo
			continuo (stream) y producir una salida también en forma de flujo
			continuo.

			Considero que el estilo adecuado es \emph{pipes and filters}
			deido este ha sido diseñado para el procesamiento de datos en
			flujo continuo. Cada filtro procesa los datos y los pasa al siguiente,
			permitiendo una cadena de procesamiento eficiente donde la
			entrada fluye a través de una serie de transformaciones, hasta generar
			la salida.

		\item Hay una gran cantidad de usuarios que se conectarán al sistema
			a través de una red.

			Desde mi punto de vista, el estilo que se adecúa a este caso es
			el \emph{cliente-servidor} principalmente porque permite a múltiples
			clientes (usuarios) conectarse a un servidor a través de internet
			para que este le suministre los servicios y recursos solicitados
			or el cliente.

		\item La aplicación necesita ser ejecutada en diferentes plataformas.

			Si fuera un aplicación web (e.g.~caso anterior) diría que
			cliente-servidor debido a que estas aplicaciones pueden ser
			ejecutadas en múltiples sistemas. Sin embargo, como no es el
			caso, considero que el estilo \emph{basado en capas} es más
			óptimo, especialmente porque puede separar la aplicación en capas,
			tales como presentación y lógica del sistema y, por ende,
			tiende a hacer más sencillo la portabilidad de una aplicación.

		\item El sistema debe tener un alto performance tanto en tiempo como
			en espacio (memoria).

			El estilo que considero que es el más adecuado es el \emph{Peer-to-Peer}.
			Este estilo permite distribuir la carga de procesamiento y memoria
			de manera uniforme.

		\item El sistema debe ser capaz de ser ajustado para alcanzar el mejor
			uso de los recursos de software y hardware, los cuales pueden
			ser cambiados en el futuro.

			El diseño \emph{basado en eventos} considero el más óptimo para
			este caso, ya que este permite que los componentes del sistema
			sean activas solamente cuando ocurre un determinado evento, lo cual
			permite un mejor aprovechamiento de los recursos del sistema.

		\item Un componente clave del sistema es la solución de un problema
			recurrente en el dominio de aplicación. Se desea reusar el
			componente en sistemas similares.

			Pienso que el estilo \emph{basado en capas} es la mejor solución
			a este caso debido a que permite separar el sistema en diferentes
			capas, lo cual facilita la reutilización de componentes.
	\end{enumerate}

	\subsection{Modelo Estilo Arquitectónico}

	Para cada uno de los problemas descritos a continuación, modele el
	estilo arquitectónico solicitado. Para cada caso analice las ventajas
	y desventajas del modelo, e indique si podría utilizar otro estilo
	arquitectónico (o combinación de estilos). En caso afirmativo,
	justifique adecuadamente su decisión y realice además el modelo
	alternativo.

	\subsubsection{Tubos y Filtros}

	Se desea contar con un sistema capaz de mejorar la calidad del sonido
	de escuchas telefónicas, grabación a distancia de conversaciones, etc.
	El sonido será digitalizado y a partir de allí se lo someterá a
	diversas transformaciones para mejorar su calidad. Las transformaciones
	posibles son:

	\begin{itemize}
		\item Dividir los sonidos según su frecuencia, lo que produce
			varias bandas de sonido diferentes que pueden ser analizadas
			separadamente o vueltas a reunir.

		\item Disminuir la velocidad con que se reproduce el sonido de una
			cantidad fijada dinámicamente por el usuario.

		\item Eliminar todos los sonidos de una frecuencia dada por el usuario.

		\item Aumentar y disminuir el volumen en una cantidad fijada por el
			usuario.
	\end{itemize}

	El usuario puede determinar en cualquier paso si desea escuchar el
	resultado. Cada vez que se realiza una mejora a la calidad del sonido,
	es el usuario quien decide qué operación de las disponibles desea
	realizar.

	\paragraph{Modelado}

	El sistema se puede estructurar como una cadena de \textbf{filtros}
	que procesan los datos de sonido (la entrada). Cada filtro aplica
	una de las transformaciones posibles, y los resultados pueden pasar
	a otro filtro o ser escuchados por el usuario. Un posible flujo
	sería:

	\begin{enumerate}
		\item \textbf{Filtro 1}: Dividir los sonidos según su frecuencia.

		\item \textbf{Filtro 2}: Disminuir la velocidad del sonido.

		\item \textbf{Filtro 3}: Eliminar sonidos de una frecuencia específica.

		\item \textbf{Filtro 4}: Aumentar o disminuir el volumen.
	\end{enumerate}

	Los datos fluyen a través de estos filtros secuencialmente, y el
	usuario puede decidir en cualquier punto si desea escuchar el
	sonido procesado hasta ese momento.

	\paragraph{Ventajas}

	\begin{itemize}
		\item \textbf{Modularidad}: Cada operación de mejora de sonido es
			independiente (filtros separados). Esto facilita la adición, eliminación
			o modificación de filtros sin alterar el resto del sistema.

		\item \textbf{Reutilización}: Los filtros pueden ser reutilizados
			en otros sistemas que necesiten manipulación de sonido.

		\item \textbf{Flexibilidad}: El usuario puede elegir qué operaciones
			aplicar y en qué orden, lo que se adapta bien a las necesidades
			específicas del proceso.
	\end{itemize}

	\paragraph{Desventajas}

	\begin{itemize}
		\item \textbf{Orden fijo de procesamiento}: Aunque el usuario puede
			elegir qué filtros aplicar, el flujo de procesamiento sigue
			siendo secuencial, lo que puede limitar la capacidad de reordenar
			dinámicamente las operaciones.

		\item \textbf{Latencia}: Cada filtro introduce un retraso en el procesamiento.
			Si el sonido pasa por muchos filtros, podría generar una latencia
			notable en el resultado final.
	\end{itemize}

	\subsubsection{Sistemas en Capas o Estratificados}

	Los pacientes de una guardia de terapia intensiva son monitoreados por
	dispositivos electrónicos análogos adosados a sus cuerpos. Los
	dispositivos miden los signos vitales de los pacientes. Hay un sensor
	para el ritmo cardíaco (activo, una señal para cada latido del corazón),
	presión sanguínea (pasivo) y la temperatura (pasivo). Se necesita
	un programa que lea los signos vitales con una frecuencia
	especificada para cada paciente. Los valores leídos serán comparados
	con rangos que serán reportadas con mensajes de alarma en el monitor
	del box de enfermería. También debe mostrarse un mensaje apropiado
	si falla un dispositivo. Además, se requiere almacenar los datos y
	proveer reportes estadísticos.

	\paragraph{Modelado}

	El sistema puede estructurarse en capas para separar las diferentes
	responsabilidades del monitoreo de pacientes. Un modelo de capas podría
	ser el siguiente:

	\begin{enumerate}
		\item \textbf{Capa de Sensores}: Gestiona la lectura de datos de los
			dispositivos (ritmo cardíaco, presión sanguínea, temperatura) en
			intervalos de tiempo específicos para cada paciente.

		\item \textbf{Capa de Procesamiento de Señales}: Compara los valores
			leídos con los rangos normales y determina si es necesario generar
			una alarma.

		\item \textbf{Capa de Almacenamiento y Reportes}: Almacena los datos
			recogidos y genera reportes estadísticos.

		\item \textbf{Capa de Presentación}: Muestra los resultados y las
			alarmas en los monitores del personal de enfermería, incluyendo
			fallas de dispositivos.
	\end{enumerate}

	\paragraph{Ventajas}

	\begin{itemize}
		\item \textbf{Separación de responsabilidades}: Cada capa tiene una
			responsabilidad clara, lo que facilita el mantenimiento y la
			evolución del sistema.

		\item \textbf{Modularidad}: Las capas pueden modificarse o mejorarse
			independientemente. Por ejemplo, se puede cambiar la forma en que
			se gestionan los sensores sin afectar la capa de presentación.

		\item \textbf{Escalabilidad}: Nuevas funciones, como la
			incorporación de más sensores o diferentes tipos de alarmas, se
			pueden integrar fácilmente agregando o ajustando capas.
	\end{itemize}

	\paragraph{Desventajas}

	\begin{itemize}
		\item \textbf{Desempeño}: La comunicación entre capas puede introducir
			una sobrecarga en términos de tiempo y uso de recursos.

		\item \textbf{Dependencia entre capas}: Si una capa falla, podría
			afectar negativamente a las demás, especialmente si no está bien
			desacoplada.
	\end{itemize}
    %% References

    \nocite{*} % to include uncited references of .bib file

    \clearpage
    \bibliographystyle{apalike}

    % Generated from .bib file
    \bibliography{ref}
\end{document}
