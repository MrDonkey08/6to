\documentclass[11pt, a4paper]{article} % Formato

% Language and font encodings
\usepackage[spanish]{babel}
\usepackage[utf8]{inputenc}
\usepackage[T1]{fontenc}
\usepackage{times} % Times New Roman

%% Sets page size and margins
\usepackage[margin=2.5cm, includefoot]{geometry}
%\setlength{\columnsep}{0.17in} % page columns separation

%% Useful packages
\usepackage{amsmath}
\usepackage{array} % <-- add this line for m{} column type
\usepackage[hidelinks]{hyperref} % hyperlinks support
\usepackage{graphicx} % images support
\usepackage{listings} % codeblock support
%\usepackage{smartdiagram} % diagrams support
\usepackage[most]{tcolorbox} % callouts support
%\usepackage[colorinlistoftodos]{todonotes}
\usepackage[dvipsnames, table, xcdraw]{xcolor} % Tables support
%\usepackage{zed-csp} % cchemas support

%% Formating
\usepackage{authblk} % to add authors in maketitle
%\usepackage{blindtext} % to gen filler text
\usepackage[figurename=Fig.]{caption} % to change prefix of the image caption

%\usepackage{apacite}
\usepackage{cite} % useful to compress multiple quotations into a single entry
\usepackage{enumitem}
\usepackage{fancyhdr} % to set page style
\usepackage{indentfirst}
\usepackage{../nasm/lang}  % include custom language for NASM assembly.
\usepackage{../nasm/style} % include custom style for NASM assembly.
%\usepackage{natbib}
\usepackage{parskip} % remove first line tabulation
\usepackage{setspace}
%\usepackage{titlesec}
%\usepackage{titling} % to config maketitle

%% Variables
% Main images
\newcommand{\logoUdg}{logo-udg.jpg}
\newcommand{\logoCucei}{logo-cucei.jpg}

% Figures
\newcommand{\figA}{./img/1-start.jpg}
\newcommand{\figB}{./img/2-creating_and_opening.jpg}
\newcommand{\figC}{./img/3-writing.jpg}
\newcommand{\figD}{./img/4-attributes.jpg}
\newcommand{\figE}{./img/5-closing.jpg}
\newcommand{\figF}{./img/6-removing.jpg}

% School data
\newcommand{\universidad}{Universidad de Guadalajara}
\newcommand{\cede}{Centro Universitario de Ciencias Exactas e Ingenierías}

% Subject data
\newcommand{\materia}{Programación de Bajo Nivel}
\newcommand{\carrera}{Ingeniería en Computación}
\newcommand{\division}{División de Tecnologías para la Integración CiberHumana}
\newcommand{\theTitle}{4. Manipulación de Archivos}
\newcommand{\profesor}{José Juan Meza Espinoza}
\newcommand{\seccion}{D02}
\newcommand{\nrc}{209850}
\newcommand{\clave}{IL358}
\newcommand{\startDate}{20 de octubre de 2024}

% Author data
\newcommand{\theAuthor}{Alan Yahir Juárez Rubio}
\newcommand{\theAuthorCode}{218517809}
\newcommand{\theAuthorMail}{alan.juarez5178@alumnos.udg.mx}

%% Declaration
\date{}
\graphicspath{ {../../../img/} }
\addto\captionsspanish{\renewcommand{\contentsname}{Índice}}
\renewcommand{\lstlistingname}{Código} % to change prefix of the code caption
\renewcommand{\lstlistlistingname}{Índice de códigos} % to change listings index title

%% Styles

% Color declaration
\definecolor{greenPortada}{HTML}{69A84F}
\definecolor{LightGray}{gray}{0.9}
\definecolor{codegreen}{rgb}{0, 0.6, 0}
\definecolor{codegray}{rgb}{0.5, 0.5, 0.5}
\definecolor{codepurple}{rgb}{0.58, 0, 0.82}
\definecolor{backcolour}{rgb}{0.95, 0.95, 0.92}

% Hyperlinks
\hypersetup{
    colorlinks=true,
    linkcolor=black,
    filecolor=greenPortada,
    urlcolor=greenPortada,
    pdfpagemode=FullScreen,
}

\urlstyle{same}

% Codeblocks
\lstdefinestyle{mystyle}{
	backgroundcolor=\color{backcolour},
	commentstyle=\color{codegreen},
	keywordstyle=\color{magenta},
	numberstyle=\tiny\color{codegray},
	stringstyle=\color{codepurple},
	basicstyle=\ttfamily\footnotesize,
	breakatwhitespace=false,
	breaklines=true,
	captionpos=b,
	keepspaces=true,
	numbers=left,
	numbersep=5pt,
	showspaces=false,
	showstringspaces=false,
	showtabs=false,
	tabsize=4
}

% Tables
\let\oldtabular\tabular
\renewcommand{\tabular}{\small\oldtabular}
\renewcommand{\arraystretch}{1.1} % <-- Adjust vertical spacing
\addto\captionsspanish{\renewcommand{\tablename}{Tabla}}

\lstset{style=mystyle}

%% Listings

\lstset{
  literate={á}{{\'a}}1 {é}{{\'e}}1 {í}{{\'i}}1 {ó}{{\'o}}1 {ú}{{\'u}}1
           {Á}{{\'A}}1 {É}{{\'E}}1 {Í}{{\'I}}1 {Ó}{{\'O}}1 {Ú}{{\'U}}1
           {ñ}{{\~n}}1 {Ñ}{{\~N}}1
}

%% Spacing
\newcommand{\nl}{\par
\vspace{0.4cm}}
\renewcommand{\baselinestretch}{1.5} % Espaciado de línea anterior
\setlength{\parskip}{6pt} % Espaciado de línea anterior
\setlength{\parindent}{0pt} % Sangría

% Header and footer
\pagestyle{fancy}
\fancyhf{}
\renewcommand{\headrulewidth}{3pt}
\renewcommand{\headrule}{\hbox to\headwidth{\color{greenPortada}\leaders\hrule height \headrulewidth\hfill}}
\setlength{\headheight}{50pt} % Ajuste necesario para evitar warnings

% Header
\setlength{\headheight}{59.9055pt}
\addtolength{\topmargin}{-9.9055pt}

\lhead{
	\begin{minipage}[c][2cm][c]{1.3cm}
		\begin{flushleft}
			\includegraphics[width=5cm, height=1.4cm, keepaspectratio]{\logoUdg}
		\end{flushleft}
	\end{minipage}
	\begin{minipage}[c][2cm][c]{0.5\textwidth} % Adjust the height as needed
		\begin{flushleft}
			{\materia}
		\end{flushleft}
	\end{minipage}
}

\rhead{
	\begin{minipage}[c][2cm][c]{0.4\textwidth} % Adjust the height as needed
		\begin{flushright}
			{\theTitle}
		\end{flushright}
	\end{minipage}
	\begin{minipage}[c][2cm][c]{1.3cm}
		\begin{flushright}
			\includegraphics[width=5cm, height=1.4cm, keepaspectratio]{\logoCucei}
		\end{flushright}
	\end{minipage}
}

% Footer
\fancyfoot{}
\setlength{\footskip}{35.27028pt}

\lfoot{
	\begin{minipage}[c][2cm][c]{0.4\textwidth} % Adjust the height as needed
		\begin{flushleft}
			{\small Elaborado por \theAuthor}
		\end{flushleft}
	\end{minipage}
}

\cfoot{\thepage} % Paginación

\rfoot{
	\begin{minipage}[c][2cm][c]{0.4\textwidth} % Adjust the height as needed
		\begin{flushright}
			{\small Curso impartido por \profesor}
		\end{flushright}
	\end{minipage}
}

%% Title

\title{\fontsize{24}{28.8}\selectfont \theTitle}
\author{\theAuthor}

\affil{}


\begin{document}
    \setstretch{1} % Interlineado

    \begin{titlepage}
        \newgeometry{margin=2.5cm, left=3cm, right=3cm} % change margin
        \centering
        %\vspace*{-2cm}
        {\huge\textbf{\universidad}}\par
        \vspace{0.6cm}
        {\LARGE{\cede}}
        \vfill

        \begin{figure}[h]
            \begin{minipage}[t]{0.45\textwidth}
                \centering
                \includegraphics[width=130px, height=160px, keepaspectratio]{\logoUdg}
            \end{minipage}
            \hfill
            \begin{minipage}[t]{0.45\textwidth}
                \centering
                \includegraphics[width=130px, height=160px, keepaspectratio]{\logoCucei}
            \end{minipage}
        \end{figure}
        \vfill

        \Large{ \division\vfill \textbf{\carrera}\vfill \textbf{\materia}\par\vspace{3pt} \seccion\ - \clave\ - \nrc\vfill }

        \begin{figure}[h]
            \centering
            \begin{minipage}[t]{0.75\textwidth}
                {\Large \textbf{Profesor}: \profesor\nl \textbf{Alumno}: \theAuthor\nl \textbf{Código}: \theAuthorCode\nl \textbf{Correo}: \theAuthorMail }
            \end{minipage}
        \end{figure}
        \vfill

        {\LARGE{\textbf{\theTitle}}}
        \vfill

        \begin{tcolorbox}
            [colback=red!5!white, colframe=red!75!black]
            \centering
            Este documento contiene información sensible.\\ No debería ser impreso o
            compartido con terceras entidades.
        \end{tcolorbox}
        \vfill
        {\large \startDate}\par
    \end{titlepage}

    \restoregeometry % end changed margin

    %% Indexes
    \clearpage
    \tableofcontents

    %\clearpage
    %\listoffigures

    \clearpage
    %\listoftables

    %\clearpage
    %\lstlistoflistings

    %% Main Title
    \clearpage
    \vspace*{6pt}
    \centerline{\huge \theTitle}
    \vspace*{8pt}

    %% Content

    \section{Cuestionario}
    \begin{enumerate}
        \item ¿Qué es el ciclo de vida del desarrollo de software seguro (S-SDLC) y cuales
            son las caracteristicas que se presenta?

            Es una extensión del Ciclo de vida del Desarrollo de Software(SDLC), el cual añade
            pruebas de seguridad en cada una de sus etapas. Este tiene como objetivo
            garantizar que el software sea seguro y resistente a vulnerabilidades.

            Las características que presenta son:

            \begin{itemize}
                \item Integración temprana de la seguridad

                \item Revisión continua de seguridad

                \item Análisis de riesgos

                \item Pruebas de seguridad

                \item Cumplimiento normativo

                \item Capacitación en seguridad

                \item Gestión de parches y actualizaciones

                \item Monitoreo y respuesta ante incidentes
            \end{itemize}

        \item ¿Qué debemos tener en cuenta al seleccionar una metodología?

            Para seleccionar una metodología o mobelo de proceso es de vital imporancia tomar
            en cuenta cuestiones como:

            \begin{itemize}
                \item \textbf{Requisitos del proyecto}: Si los requisitos no están claramente
                    definidos desde el principio o si se anticipan cambios frecuentes, un
                    enfoque más iterativo como el Modelo Ágil o Modelo Espiral podría ser
                    más adecuado.

                \item \textbf{Tiempo}: Ser conscientes de los plazos de tiempo requeridos e
                    identificar si es necesario entregas rápidas.

                \item \textbf{Personal}: La experiencia y la disponibilidad del equipo también
                    influencian la elección. Equipos con experiencia en metodologías
                    ágiles pueden optar por Scrum o Kanban.

                \item \textbf{Gestión de Riesgos}: Proyectos con altos riesgos técnicos pueden
                    beneficiarse del Modelo Espiral, que incorpora análisis de riesgos en
                    cada iteración.

                \item \textbf{Colaboración y comunicación}: Equipos que valoran la colaboración
                    y la comunicación continua pueden beneficiarse del Modelo Ágil o Scrum.
            \end{itemize}

        \item Explicando las razones para su respuesta, y con base en el tipo de sistema a
            desarrollar, sugiera el modelo de proceso de software genérico más adecuado que
            se use como fundamento para administrar el desarrollo de los siguientes sistemas:

            \begin{itemize}
                \item Un sistema para controlar el antibloqueo de frenos en un automóvil

                    Considero que el \textbf{modelo V} sería el más ideal para este caso.
                    Este modelo cuenta con fases de control de calidad, las cuales me parecen
                    esenciales para un sistema que se tiene que entregar sin fallos, para
                    garantizar un correcto funcionamiento y, por ende, mitigar cualquier
                    tipo de falla al momento de la entrega

                \item Un sistema de realidad virtual para apoyar el mantenimiento de software

                    Pienso que el modelo más adecuado sería el \textbf{modelo espiral} debido
                    a que su enfoque interactivo permite la creación de prototipos lo cual
                    permite validar conceptos y funcionalidades del sistema

                \item Un sistema de contabilidad universitario que sustituya a uno existente

                    Desde mi punto de vista, el \textbf{modelo incremental} es el más
                    apropiado. Este permite desarrollar y desplegar el nuevo sistema en partes,
                    lo que facilita la transición gradual. Al trabajar por partes es posible
                    recibir retroalimentación de los usuarios, con el fin de resolver los
                    problemas del antiguo sistema y mejorar y pulir el nuevo sistema.

                \item Un sistema interactivo de programación de viajes que ayude a los usuarios
                    a planear viajes con el menor impacto ambiental.

                    Piendo que el \textbf{modelo ágil} es el más óptimo debido a que es
                    muy probable que las necesidades del usuario y las consideraciones
                    ambientales cambien de manera constante y rápida. Adicionalmente, este
                    modelo permite tiempos de entrega cortos lo que es beneficioso para
                    obtener un producto funcional rápido y mejorar con base en la
                    retroalimentación de los usuarios
            \end{itemize}

        \item Explique por qué los sistemas desarrollados como prototipos por lo general no
            deben usarse como sistemas de producción

            Los prototipos están diseñados para ser herramientas de exploración y
            validación de conceptos, no para ser productos finales. Usarlos como sistemas
            de producción implica riesgos significativos en términos de calidad, seguridad,
            escalabilidad, mantenibilidad y cumplimiento normativo. Por estas razones, es recomendable
            que los prototipos se utilicen como una base para desarrollar un sistema de producción
            más robusto y bien diseñado, que cumpla con los estándares necesarios para su
            uso en entornos críticos y exigentes.

        \item Realiza una tabla comparativa de las metodologias agiles actuales

            \begin{table}[h!]
                \centering
                \begin{tabular}{|p{0.15 \textwidth}|p{0.2 \textwidth}|p{0.2 \textwidth}|p{0.2 \textwidth}|p{0.2 \textwidth}|}
                    \hline
                    \textbf{Metodología Ágil}         & \textbf{Descripción}                                                                                                                                                                                             & \textbf{Ventajas}                                                                                                        & \textbf{Desventajas}                                                                                                                                                & \textbf{Casos de Uso Comunes}                                                                                \\
                    \hline
                    \textbf{Scrum}                    & Un marco ágil que organiza el trabajo en sprints (ciclos de 1-4 semanas) con reuniones diarias (Daily Scrum), y roles específicos como Product Owner, Scrum Master, y equipo de desarrollo.                      & - Iteraciones cortas con entregas frecuentes. - Alta transparencia y colaboración continua. - Flexibilidad ante cambios. & - Requiere alta disciplina y compromiso. - Puede ser difícil de gestionar en equipos grandes. - Puede ser rígido en organizaciones muy jerárquicas.                 & - Desarrollo de software. - Proyectos con requisitos cambiantes. - Equipos pequeños a medianos.              \\
                    \hline
                    \textbf{Kanban}                   & Un método visual que gestiona el trabajo a medida que pasa por un proceso. Utiliza un tablero Kanban con columnas que representan etapas del flujo de trabajo y tarjetas que representan tareas.                 & - Visualización clara del flujo de trabajo. - Flexible y fácil de implementar. - Identificación de cuellos de botella.   & - No tiene iteraciones ni roles definidos. - Puede ser menos efectivo en la planificación a largo plazo. - Falta de estructura comparada con otros enfoques ágiles. & - Mantenimiento continuo de sistemas. - Soporte técnico. - Proyectos con flujo de trabajo constante.         \\
                    \hline
                    \textbf{Extreme Programming (XP)} & Un enfoque ágil centrado en mejorar la calidad del software y la capacidad de respuesta ante cambios mediante prácticas como desarrollo basado en pruebas (TDD), integración continua, y programación en pareja. & - Alta calidad de código. - Retroalimentación constante. - Enfoque en la satisfacción del cliente.                       & - Requiere alta disciplina en el equipo. - Puede ser costoso en términos de tiempo y recursos. - No siempre es adecuado para proyectos con baja incertidumbre.      & - Proyectos con alta incertidumbre y cambio frecuente. - Desarrollo de software crítico. - Equipos pequeños. \\
                    \hline
                \end{tabular}
                \caption{Comparación de metodologías ágiles}
            \end{table}

        \item En el contexto de metodologías ágiles, ¿cómo se manejan los cambios en los
            requisitos del cliente durante el desarrollo del proyecto?

            En las metodologías ágiles, los cambios en los requisitos del cliente se gestionan
            de manera flexible y adaptativa, permitiendo que el equipo los integre durante
            el desarrollo mediante ciclos cortos e iterativos. A través de una retroalimentación
            continua, reuniones de revisión y priorización dinámica del trabajo, el equipo
            puede ajustar el enfoque en función de las necesidades cambiantes del cliente.
            La colaboración cercana entre el equipo de desarrollo y el cliente, junto con la
            flexibilidad en el alcance del proyecto, asegura que el producto final refleje
            mejor las necesidades reales y emergentes del cliente, reduciendo así el riesgo
            de malentendidos y entregas inadecuadas.
    \end{enumerate}

    %% References

    \nocite{*} % to include uncited references of .bib file

    \clearpage
    \bibliographystyle{apalike}

    % Generated from .bib file
    \bibliography{ref}
\end{document}