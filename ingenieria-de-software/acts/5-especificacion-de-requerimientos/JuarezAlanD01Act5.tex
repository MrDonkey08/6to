\documentclass[11pt, a4paper]{article} % Formato

% Language and font encodings
\usepackage[spanish]{babel}
\usepackage[utf8]{inputenc}
\usepackage[T1]{fontenc}
\usepackage{times} % Times New Roman

%% Sets page size and margins
\usepackage[margin=2.5cm, includefoot]{geometry}
%\setlength{\columnsep}{0.17in} % page columns separation

%% Useful packages
\usepackage{amsmath}
\usepackage{array} % <-- add this line for m{} column type
\usepackage[hidelinks]{hyperref} % hyperlinks support
\usepackage{graphicx} % images support
\usepackage{listings} % codeblock support
%\usepackage{smartdiagram} % diagrams support
\usepackage[most]{tcolorbox} % callouts support
%\usepackage[colorinlistoftodos]{todonotes}
\usepackage[dvipsnames, table, xcdraw]{xcolor} % Tables support
%\usepackage{zed-csp} % cchemas support

%% Formating
\usepackage{authblk} % to add authors in maketitle
%\usepackage{blindtext} % to gen filler text
\usepackage[figurename=Fig.]{caption} % to change prefix of the image caption

%\usepackage{apacite}
\usepackage{cite} % useful to compress multiple quotations into a single entry
\usepackage{enumitem}
\usepackage{fancyhdr} % to set page style
\usepackage{indentfirst}
\usepackage{../nasm/lang}  % include custom language for NASM assembly.
\usepackage{../nasm/style} % include custom style for NASM assembly.
%\usepackage{natbib}
\usepackage{parskip} % remove first line tabulation
\usepackage{setspace}
%\usepackage{titlesec}
%\usepackage{titling} % to config maketitle

%% Variables
% Main images
\newcommand{\logoUdg}{logo-udg.jpg}
\newcommand{\logoCucei}{logo-cucei.jpg}

% Figures
\newcommand{\figA}{./img/1-start.jpg}
\newcommand{\figB}{./img/2-creating_and_opening.jpg}
\newcommand{\figC}{./img/3-writing.jpg}
\newcommand{\figD}{./img/4-attributes.jpg}
\newcommand{\figE}{./img/5-closing.jpg}
\newcommand{\figF}{./img/6-removing.jpg}

% School data
\newcommand{\universidad}{Universidad de Guadalajara}
\newcommand{\cede}{Centro Universitario de Ciencias Exactas e Ingenierías}

% Subject data
\newcommand{\materia}{Programación de Bajo Nivel}
\newcommand{\carrera}{Ingeniería en Computación}
\newcommand{\division}{División de Tecnologías para la Integración CiberHumana}
\newcommand{\theTitle}{4. Manipulación de Archivos}
\newcommand{\profesor}{José Juan Meza Espinoza}
\newcommand{\seccion}{D02}
\newcommand{\nrc}{209850}
\newcommand{\clave}{IL358}
\newcommand{\startDate}{20 de octubre de 2024}

% Author data
\newcommand{\theAuthor}{Alan Yahir Juárez Rubio}
\newcommand{\theAuthorCode}{218517809}
\newcommand{\theAuthorMail}{alan.juarez5178@alumnos.udg.mx}

%% Declaration
\date{}
\graphicspath{ {../../../img/} }
\addto\captionsspanish{\renewcommand{\contentsname}{Índice}}
\renewcommand{\lstlistingname}{Código} % to change prefix of the code caption
\renewcommand{\lstlistlistingname}{Índice de códigos} % to change listings index title

%% Styles

% Color declaration
\definecolor{greenPortada}{HTML}{69A84F}
\definecolor{LightGray}{gray}{0.9}
\definecolor{codegreen}{rgb}{0, 0.6, 0}
\definecolor{codegray}{rgb}{0.5, 0.5, 0.5}
\definecolor{codepurple}{rgb}{0.58, 0, 0.82}
\definecolor{backcolour}{rgb}{0.95, 0.95, 0.92}

% Hyperlinks
\hypersetup{
    colorlinks=true,
    linkcolor=black,
    filecolor=greenPortada,
    urlcolor=greenPortada,
    pdfpagemode=FullScreen,
}

\urlstyle{same}

% Codeblocks
\lstdefinestyle{mystyle}{
	backgroundcolor=\color{backcolour},
	commentstyle=\color{codegreen},
	keywordstyle=\color{magenta},
	numberstyle=\tiny\color{codegray},
	stringstyle=\color{codepurple},
	basicstyle=\ttfamily\footnotesize,
	breakatwhitespace=false,
	breaklines=true,
	captionpos=b,
	keepspaces=true,
	numbers=left,
	numbersep=5pt,
	showspaces=false,
	showstringspaces=false,
	showtabs=false,
	tabsize=4
}

% Tables
\let\oldtabular\tabular
\renewcommand{\tabular}{\small\oldtabular}
\renewcommand{\arraystretch}{1.1} % <-- Adjust vertical spacing
\addto\captionsspanish{\renewcommand{\tablename}{Tabla}}

\lstset{style=mystyle}

%% Listings

\lstset{
  literate={á}{{\'a}}1 {é}{{\'e}}1 {í}{{\'i}}1 {ó}{{\'o}}1 {ú}{{\'u}}1
           {Á}{{\'A}}1 {É}{{\'E}}1 {Í}{{\'I}}1 {Ó}{{\'O}}1 {Ú}{{\'U}}1
           {ñ}{{\~n}}1 {Ñ}{{\~N}}1
}

%% Spacing
\newcommand{\nl}{\par
\vspace{0.4cm}}
\renewcommand{\baselinestretch}{1.5} % Espaciado de línea anterior
\setlength{\parskip}{6pt} % Espaciado de línea anterior
\setlength{\parindent}{0pt} % Sangría

% Header and footer
\pagestyle{fancy}
\fancyhf{}
\renewcommand{\headrulewidth}{3pt}
\renewcommand{\headrule}{\hbox to\headwidth{\color{greenPortada}\leaders\hrule height \headrulewidth\hfill}}
\setlength{\headheight}{50pt} % Ajuste necesario para evitar warnings

% Header
\setlength{\headheight}{59.9055pt}
\addtolength{\topmargin}{-9.9055pt}

\lhead{
	\begin{minipage}[c][2cm][c]{1.3cm}
		\begin{flushleft}
			\includegraphics[width=5cm, height=1.4cm, keepaspectratio]{\logoUdg}
		\end{flushleft}
	\end{minipage}
	\begin{minipage}[c][2cm][c]{0.5\textwidth} % Adjust the height as needed
		\begin{flushleft}
			{\materia}
		\end{flushleft}
	\end{minipage}
}

\rhead{
	\begin{minipage}[c][2cm][c]{0.4\textwidth} % Adjust the height as needed
		\begin{flushright}
			{\theTitle}
		\end{flushright}
	\end{minipage}
	\begin{minipage}[c][2cm][c]{1.3cm}
		\begin{flushright}
			\includegraphics[width=5cm, height=1.4cm, keepaspectratio]{\logoCucei}
		\end{flushright}
	\end{minipage}
}

% Footer
\fancyfoot{}
\setlength{\footskip}{35.27028pt}

\lfoot{
	\begin{minipage}[c][2cm][c]{0.4\textwidth} % Adjust the height as needed
		\begin{flushleft}
			{\small Elaborado por \theAuthor}
		\end{flushleft}
	\end{minipage}
}

\cfoot{\thepage} % Paginación

\rfoot{
	\begin{minipage}[c][2cm][c]{0.4\textwidth} % Adjust the height as needed
		\begin{flushright}
			{\small Curso impartido por \profesor}
		\end{flushright}
	\end{minipage}
}

%% Title

\title{\fontsize{24}{28.8}\selectfont \theTitle}
\author{\theAuthor}

\affil{}


\begin{document}
    \setstretch{1} % Interlineado

    \begin{titlepage}
        \newgeometry{margin=2.5cm, left=3cm, right=3cm} % change margin
        \centering
        %\vspace*{-2cm}
        {\huge\textbf{\universidad}}\par
        \vspace{0.6cm}
        {\LARGE{\cede}}
        \vfill

        \begin{figure}[h]
            \begin{minipage}[t]{0.45\textwidth}
                \centering
                \includegraphics[width=130px, height=160px, keepaspectratio]{\logoUdg}
            \end{minipage}
            \hfill
            \begin{minipage}[t]{0.45\textwidth}
                \centering
                \includegraphics[width=130px, height=160px, keepaspectratio]{\logoCucei}
            \end{minipage}
        \end{figure}
        \vfill

        \Large{ \division\vfill \textbf{\carrera}\vfill \textbf{\materia}\par\vspace{3pt} \seccion\ - \clave\ - \nrc\vfill }

        {\LARGE{\textbf{\theTitle}}}
        \vfill

	\begin{figure}[h]
		\centering
		\begin{minipage}[t]{0.75\textwidth}
			{\Large
				\textbf{Integrantes:}\par\vspace{8pt}
				\begin{itemize}
					\item \bAuthor\ - \bAuthorCode
					\item \cAuthor\ - \cAuthorCode
					\item \dAuthor\ - \dAuthorCode
					\item \eAuthor\ - \eAuthorCode
					\item \theAuthor\ - \theAuthorCode
				\end{itemize}
			}
		\end{minipage}
	\end{figure}\vfill


        \begin{tcolorbox}
            [colback=red!5!white, colframe=red!75!black]
            \centering
            Este documento contiene información sensible.\\
			No debería ser impreso o compartido con terceras entidades.
        \end{tcolorbox}
        \vfill
        {\large \startDate}\par
    \end{titlepage}

    \restoregeometry % end changed margin

    %% Indexes
    \clearpage
    \tableofcontents

    \clearpage
    \listoffigures

    \clearpage
    %\listoftables

    %\clearpage
    %\lstlistoflistings

    %% Main Title
    \clearpage
    \vspace*{6pt}
	\centerline{\textbf{\huge \theTitle}}
    \vspace*{8pt}

    %% Content
	\section*{Historial de Revisiones}

	\begin{table}[h!]
		\centering
		\begin{tabular}{|c|c|c|c|}
			\hline
			\textbf{Fecha} & \textbf{Revisión} & \textbf{Descripción}   & \textbf{Autor} \\
			\hline
			30/09/2024     & 1.0               & Introducción           & \bAuthor       \\
			               &                   &                        & \cAuthor       \\
			               &                   &                        & \dAuthor       \\
			               &                   &                        & \eAuthor       \\
			               &                   &                        & \theAuthor     \\
			\hline
			30/09/2024     & 1.0               & Descripción General    & \bAuthor       \\
			               &                   &                        & \cAuthor       \\
			               &                   &                        & \dAuthor       \\
			               &                   &                        & \eAuthor       \\
			               &                   &                        & \theAuthor     \\
			\hline
			30/09/2024     & 1.0               & Requisitos Específicos & \bAuthor       \\
			               &                   &                        & \cAuthor       \\
			               &                   &                        & \dAuthor       \\
			               &                   &                        & \eAuthor       \\
			               &                   &                        & \theAuthor     \\
			\hline
		\end{tabular}
		\caption{Historial de Revisiones}
	\end{table}

	Documento validado por las partes en fecha: \startDate

	\begin{table}[h!]
		\centering
		\begin{tabular}{|c|c|}
			\hline
			\textbf{Por el cliente}    & \textbf{Por la empresa suministradora}   \\
			\hline
			                           &                                          \\
			                           &                                          \\
			\hline
			Fdo. Equipo ``Rustaceans'' & Fdo. Equipo de desarrollo ``Rustaceans'' \\
			\hline
		\end{tabular}
		\caption{Firma de equipos}
	\end{table}

	\clearpage
	\section{Introducción}

	Este documento es la Especificación de Requisitos del Software (ERS) de la aplicación
	web \emph{Link Project}, aplicación que tiene como principal objetivo ser un medio
	entre los estudiantes de CUCEI para buscar miembros de equipo para el proyecto
	modular.

	Este documento ha sido redactado siguiendo el estándar IEE 830-1998, el cual
	establece cada uno de las prácticas recomendadas que debe llevar la
	Especificación de Requisitos del Software.

	\subsection{Propósito}

	El contenido del presente documento sirve para establecer cada uno de los requisitos
	que debe cumplir el proyecto, es decir, cada una de las características,
	funciones y propiedades con las que se plantea que la aplicación web cumpla al
	momento de la entrega.

	Adicionalmente, este documento sirve principalmente como medio de comunicación
	y acuerdo entre el desarrollador y el cliente, de tal manera que no haya
	ambigüedades ni malas interpretaciones, es decir, que haya una correcta
	comunicación y entendimiento de ambas partes sobre qué es lo que debe hacer y
	qué requisitos deberá cumplir la aplicación.

	\subsection{Alcance}

	Nuestro proyecto Link Project busca desarrollar una aplicación web que sirva como
	herramienta para la comunidad estudiantil del Centro Universitario de Ciencias
	Exactas e Ingenierías (CUCEI), para aquellos que busquen desarrollar su
	proyecto modular, específicamente aquellos que necesiten encontrar un equipo de
	trabajo.

	Este sistema permitirá a los estudiantes tanto establecer su propio proyecto modular
	como la posibilidad de incorporarse a uno previamente establecido. Asimismo, brindará
	un medio a los usuarios para administrar el proyecto, en cuestiones como
	planificación y gestión de tareas.

	\subsection{Personal Involucrado}

	\begin{table}[h!]
		\centering
		\begin{tabular}{|l|p{10cm}|}
			\hline
			\textbf{Nombre}                  & \cAuthor                                                       \\
			\hline
			\textbf{Rol}                     & Desarrollador Front-end, encargado de la difusión de la página \\
			\hline
			\textbf{Categoría profesional}   & Ing. en Computación                                            \\
			\hline
			\textbf{Responsabilidades}       & Desarrollar la interfaz de usuario                             \\
			                                 & Optimización del software                                      \\
			                                 & Estructura del contenido                                       \\
			\hline
			\textbf{Información de contacto} & brain.chavez@alumnos.udg.mx                                    \\
			\hline
		\end{tabular}
		\caption{Datos de \cAuthor}
	\end{table}

	\begin{table}[h!]
		\centering
		\begin{tabular}{|l|p{10cm}|}
			\hline
			\textbf{Nombre}                  & \eAuthor                                                 \\
			\hline
			\textbf{Rol}                     & Desarrollador Back-end encargado del desarrollo de la BD \\
			\hline
			\textbf{Categoría profesional}   & Ing. en Computación                                      \\
			\hline
			\textbf{Responsabilidades}       & Creación de BD                                           \\
			                                 & Conexión de BD con el sitio web                          \\
			\hline
			\textbf{Información de contacto} & \eAuthorMail                                             \\
			\hline
		\end{tabular}
		\caption{Datos de \eAuthor}
	\end{table}

	\begin{table}[h!]
		\centering
		\begin{tabular}{|l|p{10cm}|}
			\hline
			\textbf{Nombre}                  & \bAuthor                                                                    \\
			\hline
			\textbf{Rol}                     & Desarrollador Front-end encargado del diseño e interactividad del sitio web \\
			\hline
			\textbf{Categoría profesional}   & Ing. en Computación                                                         \\
			\hline
			\textbf{Responsabilidades}       & Interactividad                                                              \\
			                                 & Diseños visuales                                                            \\
			                                 & Responsive Design del sistema                                               \\
			\hline
			\textbf{Información de contacto} & \bAuthorMail                                                                \\
			\hline
		\end{tabular}
		\caption{Datos de \bAuthor}
	\end{table}

	\begin{table}[h!]
		\centering
		\begin{tabular}{|l|p{10cm}|}
			\hline
			\textbf{Nombre}                  & \dAuthor                                                                   \\
			\hline
			\textbf{Rol}                     & Encargado del soporte y mantenimiento de la página web                     \\
			\hline
			\textbf{Categoría profesional}   & Ing. en Computación                                                        \\
			\hline
			\textbf{Responsabilidades}       & Tener acceso a los recursos de la página web para dar ayuda a los usuarios \\
			                                 & Administrar las actualizaciones correctivas y optimizaciones               \\
			\hline
			\textbf{Información de contacto} & \dAuthorMail                                                               \\
			\hline
		\end{tabular}
		\caption{Datos de \dAuthor}
	\end{table}

	\begin{table}[h!]
		\centering
		\begin{tabular}{|l|p{10cm}|}
			\hline
			\textbf{Nombre}                  & \theAuthor                                                           \\
			\hline
			\textbf{Rol}                     & Desarrollador Back-end encargado del diseño y desarrollo del sistema \\
			\hline
			\textbf{Categoría profesional}   & Ing. en Computación                                                  \\
			\hline
			\textbf{Responsabilidades}       & Modelado de Datos (Estructura)                                       \\
			                                 & Desarrollo y diseño del sistema                                      \\
			                                 & Desarrollador de funciones de sesión                                 \\
			                                 & Búsqueda y funciones varias                                          \\
			\hline
			\textbf{Información de contacto} & \theAuthorMail                                                       \\
			\hline
		\end{tabular}
		\caption{Datos de \theAuthor}
	\end{table}

	\subsection{Definiciones, Acrónimos y Abreviaturas}

	\begin{itemize}
		\item \textbf{ERS:} Especificación de Requisitos del Software

		\item \textbf{IEEE:} Institute of Electrical and Electronics Engineers

		\item \textbf{SGBD:} Sistema Gestor de Base de Datos

		\item \textbf{PostgreSQL:} Es un sistema de gestión de bases de datos SQL

		\item \textbf{JS:} Lenguaje de programación (Javascript)

		\item \textbf{HTML5:} Lenguaje diseñado para el desarrollo contenido de una
			página web (Lenguaje de Marcado de HyperTexto)

		\item \textbf{PHP:} Es un lenguaje de programación interpretado diseñado
			para la creación de páginas web dinámicas

		\item \textbf{BD:} Base de Datos

		\item \textbf{CSS3:} Un tipo de lenguaje que permite definir y crear la
			presentación visual de un documento

		\item \textbf{SQL:} Lenguaje de consulta estructurada
	\end{itemize}

	\subsection{Referencias}

	\textbf{Título:} IEEE.Recommended practice for software requirements
	specifications.

	\textbf{URL:} \url{https://ieeexplore.ieee.org/document/720574}

	\textbf{Autor:} Software Engineering Standards Committee of the IEEE Computer Society

	\subsection{Resumen}

	Introducción: En esta primera sección se presenta el propósito del documento, explicando
	el objetivo general de la especificación de requisitos de software. Se
	establece el alcance del proyecto, detallando los resultados esperados y las responsabilidades
	de cada miembro del equipo de desarrollo. También se incluye un glosario de
	términos, acrónimos y abreviaturas relevantes, junto con las referencias
	bibliográficas y documentales utilizadas.

	Descripción del Proyecto: En la segunda sección se describe la perspectiva general
	del proyecto, incluyendo un diagrama que muestra, a grandes rasgos, la
	estructura y el funcionamiento de la aplicación web. Además, se detallan los
	tipos de usuarios y sus respectivas características, las restricciones técnicas
	y operativas del sistema, así como las suposiciones y dependencias que podrían
	influir en el desarrollo del proyecto.

	Requisitos Funcionales y No Funcionales: En esta tercera y última sección se desglosan
	detalladamente los requisitos funcionales que definen el comportamiento del sistema
	y las acciones que debe realizar. También se explican los requisitos no
	funcionales, que especifican características como el rendimiento, la seguridad,
	la usabilidad y la escalabilidad del software.

	\clearpage
	\section{Descripción General}

	\subsection{Perspectiva del Producto}

	\emph{Link Project} se establece como una aplicación web independiente, es decir,
	no pertenece o depende de algún otro sistema para su funcionamiento.

	\subsection{Funciones del Producto}

	\begin{figure}[h]
		\centering
		\includegraphics[width=\textwidth]{\diagramaCasoDeUso}
		\caption{Diagrama de Caso de Uso}
	\end{figure}

	\subsection{Características de los Usuarios}

	\begin{table}[h!]
		\centering
		\begin{tabular}{|l|p{11cm}|}
			\hline
			\textbf{Tipo de usuario}    & \textbf{Asesor}                                         \\
			\hline
			\textbf{Nivel de educación} & Licenciatura terminada                                  \\
			\hline
			\textbf{Experiencia}        & Gestión de Sistemas de información                      \\
			\hline
			\textbf{Actividades}        & Vista de proyectos modulares, integrantes y descripción \\
			\hline
		\end{tabular}
		\caption{Tipo de usuario: Asesor}
	\end{table}

	\begin{table}[h!]
		\centering
		\begin{tabular}{|l|p{11cm}|}
			\hline
			\textbf{Tipo de usuario}    & \textbf{Estudiante (Líder del Proyecto)}                                                                                                                                 \\
			\hline
			\textbf{Nivel de educación} & Licenciatura en curso                                                                                                                                                    \\
			\hline
			\textbf{Experiencia}        & Manejo básico de Sistemas de Información                                                                                                                                 \\
			\hline
			\textbf{Actividades}        & Generación y gestión del proyecto, Asignación de las tareas de los integrantes del proyecto, Revisión de solicitudes de incorporación al proyecto (aceptación o rechazo) \\
			\hline
		\end{tabular}
		\caption{Tipo de usuario: Estudiante (Líder del Proyecto)}
	\end{table}

	\begin{table}[h!]
		\centering
		\begin{tabular}{|l|p{11cm}|}
			\hline
			\textbf{Tipo de usuario}    & \textbf{Estudiante (Sin Posición)}                                         \\
			\hline
			\textbf{Nivel de educación} & Licenciatura en curso                                                      \\
			\hline
			\textbf{Experiencia}        & Manejo básico de Sistemas de Información                                   \\
			\hline
			\textbf{Actividades}        & Enviar solicitudes de incorporación a los proyectos a líderes de proyectos \\
			\hline
		\end{tabular}
		\caption{Tipo de usuario: Estudiante (Sin Posición)}
	\end{table}

	\begin{table}[h!]
		\centering
		\begin{tabular}{|l|p{11cm}|}
			\hline
			\textbf{Tipo de usuario}    & \textbf{Estudiante (Integrantes de Equipo)}                   \\
			\hline
			\textbf{Nivel de educación} & Licenciatura en curso                                         \\
			\hline
			\textbf{Experiencia}        & Manejo básico de Sistemas de Información                      \\
			\hline
			\textbf{Actividades}        & Gestión de cada una de sus actividades asignadas por el líder \\
			\hline
		\end{tabular}
		\caption{Tipo de usuario: Estudiante (Integrantes de Equipo)}
	\end{table}

	\subsection{Restricciones}

	\begin{itemize}
		\item El uso del software requiere de conexión a internet. Se utilizarán
			lenguajes y tecnologías tales como HTML5, CSS3, JS, PHP.

		\item El SGBD que se utilizará será PostgreSQL.

		\item La aplicación web será funcional en los navegadores más comunes.

		\item Debe ser construida como una aplicación cliente-servidor.

		\item La aplicación web contará con un sistema de validación de sesión.

		\item Los equipos en donde sea desplegada la aplicación deben contar con un mínimo
			de recursos para el correcto funcionamiento.
	\end{itemize}

	\subsection{Suposiciones y Dependencias}

	\begin{itemize}
		\item El usuario deberá tener descargado al menos un navegador web para poder
			utilizar el software.

		\item Dependerá de conexión a internet y cualquier tipo de computador o laptop.

		\item Se estima que el usuario tenga familiarización con el uso de internet,
			computadoras, y software básicos.
	\end{itemize}

	\subsection{Evolución Previsible del Sistema}

	El sistema estará optimizado y correctamente documentado para facilitar su
	adaptación a cualquier otro estilo de plataforma. La evolución previsible incluye
	la posibilidad de ajustar las diferentes secciones del menú (ya sean de gran
	importancia o secundarios) y la presentación de las solicitudes de los usuarios,
	adaptándose a nuevas plataformas con rapidez y eficiencia.

	Uno de los cambios más comunes será la adaptabilidad a los diferentes
	dispositivos que puedan llegar a existir. Esto incluirá el diseño responsivo que
	permitirá una visualización óptima en distintos tamaños de pantalla. Además, se
	deberá prestar atención a los diseños visuales, asegurando que tanto las
	pantallas principales como los menús secundarios se adapten de forma coherente
	y eficaz, optimizando así la experiencia de usuario independientemente del dispositivo.

	También se espera que el sistema pueda adaptarse a futuras tecnologías como Inteligencia
	Artificial o incluso el añadir un chatBot de ayuda al usuario, mejorando así
	la usabilidad del usuario y la ayuda que se le puede brindar al mismo para
	lograr convertir en un sistema más intuitivo, esto no se descarta que se logren
	añadir dependiendo la complejidad del mismo o incluso el tiempo de entrega, se
	deja como parte de futuras actualizaciones que se pueden llegar implementar al
	mismo sistema.

	\clearpage
	\section{Requerimientos}

	\subsection{Requerimientos Funcionales}

	\subsubsection{Registro de Usuarios}

	\begin{enumerate}
		\item El sistema permitirá a los alumnos registrarse mediante un formulario en
			el que les solicitará datos tales como nombre, código de estudiante, correo
			electrónico, número de contacto, carrera, contraseña. Adicionalmente tendrán
			la opción de añadir una foto de perfil

		\item El sistema permitirá a los asesores registrarse mediante un formulario
			el cual les solicitará datos como nombre, código de profesor, correo electrónico,
			departamento y número de contacto. Adicionalmente tendrán la opción de añadir
			una foto de perfil

		\item El formulario de registro incluirá campos opcionales como área de especialización,
			experiencia previa, y una breve biografía para hacer coincidir mejor las solicitudes
			con las necesidades del proyecto.
	\end{enumerate}

	\subsubsection{Inicio de Sesión}

	\begin{enumerate}
		\item Los usuarios registrados podrán acceder a sus cuentas mediante un sistema
			de inicio de sesión seguro.

		\item Se incluirá una opción para la recuperación de contraseñas a través del
			correo electrónico en caso de que el usuario olvide sus credenciales.

		\item Los usuarios pueden optar por configurar su inicio de sesión con autenticación
			de 2 pasos, esto con el fin de garantizar una mayor protección de la
			cuenta.
	\end{enumerate}

	\subsubsection{Búsqueda de Proyectos}

	\begin{enumerate}
		\item El sistema permitirá a los usuarios realizar búsquedas de proyectos modulares
			disponibles utilizando filtros como palabras clave, categoría, áreas de interés
			o habilidades requeridas.

		\item Los resultados de la búsqueda mostrarán detalles como el nombre del proyecto,
			una descripción general, requisitos técnicos o de experiencia, el número
			de vacantes disponibles, y el estatus actual del proyecto (abierto o
			cerrado a nuevas solicitudes).

		\item Los proyectos destacados o recomendados según los intereses del usuario
			también serán visibles en la interfaz de búsqueda.
	\end{enumerate}

	\subsubsection{Creación e Integración de Proyectos}

	\begin{enumerate}
		\item Un alumno solamente podrá pertenecer a un único proyecto a la vez.

		\item Los alumnos tendrán la posibilidad de crear un proyecto y, por ende, obtendrán
			el rol de líder dentro del proyecto, permitiéndoles administrar el
			proyecto y los integrantes del equipo.

		\item Los alumnos tendrán la posibilidad de integrarse a un equipo a través del
			uso de las solicitudes de admisión y la bandeja de notificaciones.
	\end{enumerate}

	\subsubsection{Gestión de Actividades}

	\begin{enumerate}
		\item Los líderes de proyecto tendrán acceso a un panel de administración donde
			podrán gestionar todas las actividades relacionadas con el proyecto.

		\item Asignación de tareas específicas a los integrantes del equipo y el seguimiento
			del progreso de dichas tareas.

		\item Los integrantes del equipo tendrán la opción de gestionar cada una sus
			tareas asignadas a través de un canvas, en el que les permitirá
			catalogarlas como pendientes, en proceso y finalizadas.
	\end{enumerate}

	\subsubsection{Bandeja de Solicitudes}

	\begin{enumerate}
		\item Los líderes de proyecto podrán gestionar las solicitudes de los usuarios
			interesados en unirse a sus proyectos.

		\item Las solicitudes recibidas serán visualizadas en una bandeja dedicada,
			donde el líder podrá revisar la información del solicitante (perfil,
			experiencia, intereses) y aceptar o rechazar la solicitud.

		\item Los líderes tendrán la opción de eliminar a un integrante del equipo y,
			a su vez, el integrante del equipo tendrá la posibilidad de abandonar el equipo.

		\item Se incluirá un historial de decisiones para un mejor seguimiento.

			Adicionalmente, el líder tendrá la opción de proporcionar
			retroalimentación al solicitante en caso de rechazo.
	\end{enumerate}

	\subsubsection{Agenda de Reuniones}

	\begin{enumerate}
		\item Los integrantes recibirán notificaciones automáticas sobre la creación,
			cancelación o modificación de reuniones.

		\item El asesor podrá programar reuniones, las cuales podrán ser aceptadas o,
			en su defecto, rechazadas por los alumnos debido al no serles posible
			asistir a dicha reunión. En caso de ser rechazada una solicitud el asesor podrá
			proponer otra reunión en un horario diferente.
	\end{enumerate}

	\subsection{Requerimientos No Funcionales}

	\subsubsection{Interfaz de Usuario (UI)}

	\begin{enumerate}
		\item La interfaz debe ser intuitiva y estéticamente agradable, facilitando
			la navegación a los usuarios sin necesidad de conocimientos técnicos
			avanzados.

		\item Se utilizarán estándares de diseño como Material Design, con el fin de
			asegurar coherencia visual y usabilidad.

		\item El diseño debe adaptarse a diferentes tamaños de pantalla,
			garantizando una experiencia fluida tanto en dispositivos móviles como en
			ordenadores de escritorio.
	\end{enumerate}

	\subsubsection{Temas y Colores}

	\begin{enumerate}
		\item Se busca que la paleta de colores sea mayormente colores neutros que ayudan
			una mejor importancia a los debidos elementos además de un apoyo a la visualización
			del usuario.
	\end{enumerate}

	\subsubsection{Íconos y Botones}

	\begin{enumerate}
		\item Se implementarán íconos fácilmente reconocibles para las funciones principales,
			acompañados de botones claros y consistentes que ofrezcan
			retroalimentación visual (como cambios de color o animaciones sutiles)
			para indicar la interactividad.

		\item Cada ícono y botón debe estar alineado con las convenciones de diseño,
			reduciendo la curva de aprendizaje para los nuevos usuarios.
	\end{enumerate}

	\subsubsection{Privacidad y Seguridad de Datos}

	\begin{enumerate}
		\item Privacidad: El sistema debe cumplir con las regulaciones de privacidad
			de datos, asegurando que la información personal de los usuarios y los proyectos
			se almacene y procese de manera confidencial.

		\item Seguridad: Todos los datos sensibles, como contraseñas y detalles personales,
			se almacenarán en bases de datos encriptadas. Las transferencias de datos entre
			el cliente y el servidor deben realizarse bajo protocolos seguros como HTTPS.
	\end{enumerate}

	\subsubsection{Rendimiento}

	\begin{enumerate}
		\item El sistema debe ser eficiente, respondiendo a las interacciones del
			usuario de manera rápida.

		\item Los tiempos de carga de las páginas y funciones críticas no deben superar
			los 5 segundos bajo condiciones normales de operación.
	\end{enumerate}

	\subsubsection{Almacenamiento de Datos}

	\begin{enumerate}
		\item El sistema utilizará una base de datos relacional o no relacional,
			según las necesidades del proyecto, que ofrezca redundancia y respaldo automático
			de datos.

		\item Se garantizará la integridad de los datos mediante la implementación de
			técnicas de rollback y copias de seguridad periódicas, para evitar la
			pérdida de información
	\end{enumerate}
\end{document}
