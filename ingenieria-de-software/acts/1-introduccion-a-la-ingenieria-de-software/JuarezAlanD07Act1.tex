\documentclass[11pt, a4paper]{article} % Formato

% Language and font encodings
\usepackage[spanish]{babel}
\usepackage[utf8]{inputenc}
\usepackage[T1]{fontenc}
\fontfamily{georgia}\selectfont

%% Sets page size and margins
\usepackage[margin=2.5cm, includefoot]{geometry}
%\setlength{\columnsep}{0.17in} % page columns separation

%% Useful packages
\usepackage{amsmath}
\usepackage{array} % <-- add this line for m{} column type
\usepackage[hidelinks]{hyperref} % hyperlinks support
\usepackage{graphicx} % images support
%\usepackage{listings} % codeblock support
%\usepackage{smartdiagram} % diagrams support
\usepackage[most]{tcolorbox} % callouts support
%\usepackage[colorinlistoftodos]{todonotes}
\usepackage[dvipsnames, table, xcdraw]{xcolor} % Tables support
%\usepackage{zed-csp} % cchemas support

%% Formating
\usepackage{authblk} % to add authors in maketitle
%\usepackage{blindtext} % to gen filler text
\usepackage[figurename=Fig.]{caption} % to change prefix of the image caption

\usepackage{apacite}
\usepackage{cite} % useful to compress multiple quotations into a single entry
\usepackage{enumitem}
\usepackage{fancyhdr} % to set page style
\usepackage{indentfirst}
%\usepackage{natbib}
\usepackage{parskip} % remove first line tabulation
\usepackage{setspace}
%\usepackage{titlesec}
%\usepackage{titling} % to config maketitle

%% Variables
% Main images
\newcommand{\logoUdg}{logo-udg.jpg}
\newcommand{\logoCucei}{logo-cucei.jpg}

% School data
\newcommand{\universidad}{Universidad de Guadalajara}
\newcommand{\cede}{Centro Universitario de Ciencias Exactas e Ingenierías}

% Subject data
\newcommand{\materia}{Ingeniería de Software}
\newcommand{\carrera}{Ingeniería en Computación}
\newcommand{\division}{División de Tecnologías para la Integración CiberHumana}
\newcommand{\theTitle}{Introducción a la Ingeniería de Software}
\newcommand{\profesor}{Thelma Isabel Morales Ramírez}
\newcommand{\seccion}{D07}
\newcommand{\nrc}{210896}
\newcommand{\clave}{CB224}
\newcommand{\startDate}{17 de agosto de 2024}

% Author data
\newcommand{\theAuthor}{Alan Yahir Juárez Rubio}
\newcommand{\theAuthorCode}{218517809}
\newcommand{\theAuthorMail}{alan.juarez5178@alumnos.udg.mx}

%% Declaration
\date{}
\graphicspath{ {../../../img/} }
\addto\captionsspanish{\renewcommand{\contentsname}{Índice}}
\renewcommand{\lstlistingname}{Código} % to change prefix of the code caption
\renewcommand{\lstlistlistingname}{Índice de códigos} % to change listings index title

%% Styles

% Color declaration
\definecolor{greenPortada}{HTML}{69A84F}
\definecolor{LightGray}{gray}{0.9}
\definecolor{codegreen}{rgb}{0, 0.6, 0}
\definecolor{codegray}{rgb}{0.5, 0.5, 0.5}
\definecolor{codepurple}{rgb}{0.58, 0, 0.82}
\definecolor{backcolour}{rgb}{0.95, 0.95, 0.92}

% Hyperlinks
\hypersetup{
    colorlinks=true,
    linkcolor=black,
    filecolor=greenPortada,
    urlcolor=greenPortada,
    pdfpagemode=FullScreen,
}

\urlstyle{same}

% Codeblocks
\lstdefinestyle{mystyle}{ backgroundcolor=
\color{backcolour}
, commentstyle=
\color{codegreen}
, keywordstyle=
\color{magenta}
, numberstyle=\tiny
\color{codegray}
, stringstyle=
\color{codepurple}
, basicstyle=\ttfamily\footnotesize, breakatwhitespace=false, breaklines=true, captionpos=b,
keepspaces=true, numbers=left, numbersep=5pt, showspaces=false, showstringspaces=false, showtabs=false,
tabsize=2 }

% Tables
\let\oldtabular\tabular
\renewcommand{\tabular}{\small\oldtabular}
\renewcommand{\arraystretch}{1.2} % <-- Adjust vertical spacing
\addto\captionsspanish{\renewcommand{\tablename}{Tabla}}

\lstset{style=mystyle}

%% Spacing
\newcommand{\nl}{\par
\vspace{0.4cm}}
\renewcommand{\baselinestretch}{1.5} % Espaciado de línea anterior
\setlength{\parskip}{6pt} % Espaciado de línea anterior
\setlength{\parindent}{0pt} % Sangría

% Header and footer
\pagestyle{fancy}
\fancyhf{}
\renewcommand{\headrulewidth}{3pt}
\renewcommand{\headrule}{\hbox to\headwidth{\color{greenPortada}\leaders\hrule height \headrulewidth\hfill}}
\setlength{\headheight}{50pt} % Ajuste necesario para evitar warnings

% Header
\pagestyle{fancy}
\fancyhf{}
\lhead{ \begin{minipage}[c][2cm][c]{1.3cm}\begin{flushleft}\includegraphics[width=5cm, height=1.4cm, keepaspectratio]{\logoUdg}\end{flushleft}\end{minipage} \begin{minipage}[c][2cm][c]{0.5\textwidth} % Adjust the height as needed
\begin{flushleft}{\materia}\end{flushleft}\end{minipage} }
\rhead{ \begin{minipage}[c][2cm][c]{0.4\textwidth} % Adjust the height as needed
\begin{flushright}{\theTitle}\end{flushright}\end{minipage} \begin{minipage}[c][2cm][c]{1.3cm}\begin{flushright}\includegraphics[width=5cm, height=1.4cm, keepaspectratio]{\logoCucei}\end{flushright}\end{minipage} }

% Footer
\fancyfoot{}
\lfoot{\small\materia}
\cfoot{\thepage} % Paginación
\rfoot{\small Curso impartido por \profesor}

\title{\fontsize{24}{28.8}\selectfont \theTitle}
\author{\theAuthor}

\affil{}

\begin{document}
    \setstretch{1} % Interlineado

    \begin{titlepage}
        \newgeometry{margin=2.5cm, left=3cm, right=3cm} % change margin
        \centering
        %\vspace*{-2cm}
        {\huge\textbf{\universidad}}\par
        \vspace{0.6cm}
        {\LARGE{\cede}}
        \vfill

        \begin{figure}[h]
            \begin{minipage}[t]{0.45\textwidth}
                \centering
                \includegraphics[width=130px, height=160px, keepaspectratio]{\logoUdg}
            \end{minipage}
            \hfill
            \begin{minipage}[t]{0.45\textwidth}
                \centering
                \includegraphics[width=130px, height=160px, keepaspectratio]{\logoCucei}
            \end{minipage}
        \end{figure}
        \vfill

        \Large{ \division\vfill \textbf{\carrera}\vfill \textbf{\materia}\par\vspace{3pt} \seccion\ - \clave\ - \nrc\vfill }

        \begin{figure}[h]
            \centering
            \begin{minipage}[t]{0.75\textwidth}
                {\Large \textbf{Profesor}: \profesor\nl \textbf{Alumno}: \theAuthor\nl \textbf{Código}: \theAuthorCode\nl \textbf{Correo}: \theAuthorMail }
            \end{minipage}
        \end{figure}
        \vfill

        {\LARGE{\textbf{\theTitle}}}
        \vfill

        \begin{tcolorbox}
            [colback=red!5!white, colframe=red!75!black]
            \centering
            Este documento contiene información sensible.\\ No debería ser impreso o
            compartido con terceras entidades.
        \end{tcolorbox}
        \vfill
        {\large \startDate}\par
    \end{titlepage}

    \restoregeometry % end changed margin

    %% Indexes
    \clearpage
    \tableofcontents

    %\clearpage
    %\listoffigures

    \clearpage
    \listoftables

    %\clearpage
    %\lstlistoflistings

    %% Main Title
    \clearpage
    \vspace*{6pt}
    \centerline{\huge \theTitle}
    \vspace*{8pt}

    %% Content

    \section{Qué es un Software}

    Si bien, el término programa y software se utilizan indistintamente, en Ingeniería de software
    se hace la distinción entre estos dos.

    Para entender qué es un software primero debemos entender qué es un programa. Un
    \textbf{programa} es una secuencia de instrucciones dadas a un dispositivo de cómputo para
    que realice una tarea en específico.

    Por otra parte, un \textbf{software} no solo se refiere a los programas en sí, sino
    también a la documentación asociada, a los datos de configuración y los datos de configuración
    requeridos para hacer que estos programas operen correctamente.

    \section{Uso del Sofware en Nuestra Sociedad}

    En la actualidad, el software hace parte de nuestras vidas diarias, facilitándonos y
    haciendo posible muchísimas cosas que sin él no sería posible, desde la comunicación a
    larga distancia hasta la adquisición y venta de productos y servicios sin contacto directo
    y a distacncias enormes. Sin embargo, el uso innadecuado del software puede traer consigo
    consecuencias desfavorables. A continuación una tabla con algunos de los \textit{benificios
    y perjuicios} del uso del software:

    \begin{table}[h]
        \centering
        \begin{tabular}{|p{0.15 \textwidth}|p{0.325 \textwidth}|p{0.325 \textwidth}|}
            \hline
            Aspecto       & Beneficios                                                                       & Perjucios                                                                      \\
            \hline
            Comunicación  & Mejora de la comunicación global instantánea a través de diversas plataformas    & Pérdida de habilidades sociales y transmisión masiva de información incorrecta \\
            \hline
            Productividad & Automatización de tareas y procesos                                              & Dependencia excesiva de la tecnología                                          \\
            \hline
            Privacidad    & Mejora en la seguridad de datos mediante la encriptación y la autenticación      & Riesgo de privacidad ocasionado por la recolección de datos                    \\
            \hline
            Seguridad     & Protección de datos sensibles mediante herramientas avanzadas de cirberseguridad & Aumento de ataques ciberneticos, tal como hackeos y fraudes en línea           \\
            \hline
            Innovación    & Facilitación de innovación en multiples sectores, desde ciencias hasta artes     & Rápido desfase de tecnologías y software                                       \\
            \hline
        \end{tabular}
        \caption{Beneficios y perjuicios del uso del software en la sociedad}
    \end{table}

    \clearpage
    \section{Qué es la Ingeniería de Software}

    La \textbf{ingeniería de software} es una disciplina de la ingeniería que se interesa
    por todos los aspectos de la producción del software, desde las primeras etapas de la especificación
    del sistema hasta el mantenimiento del sistema después de que se pone en operación.

    \subsection{Propósito de la Ingeniería de Software}

    La ingeniería de software principalmente busca conseguir el desarrollo de software de
    calidad dentro de una determinada fecha. Para conseguirlo, la ingenería de software
    busca seleccionar el método más adecuado para el desarrollo de software, desde métodos
    con un enfoque sistemático y organizado, hasta métodos con un desarrollo más creativo
    e informal.

    \subsection{¿Por Qué se Estimuló la Ingeniería de Software?}

    \begin{enumerate}
        \item \textbf{Gestión de la complejidad}: A medida que los sistemas de software se
            volvieron más grandes y complejos, surgió la necesidad de metodologías
            estructuradas para gestionar su desarrollo, mantenimiento y escalabilidad. La
            Ingeniería del software proporciona las herramientas y técnicas necesarias
            para manejar esta complejidad de manera efectiva.

        \item \textbf{Calidad y fiabilidad}: La demanda de software seguro, confiable y de
            alta calidad impulsó la creación de prácticas estandarizadas en la Ingeniería
            del software. Estas prácticas ayudan a minimizar errores, mejorar el
            rendimiento y asegurar que los sistemas cumplan con los requisitos funcionales
            y no funcionales.

        \item \textbf{Eficiencia en el desarrollo}: La necesidad de desarrollar software de
            manera más rápida y eficiente, sin sacrificar la calidad, motivó la creación de
            procesos y metodologías que optimizan cada fase del ciclo de vida del software.
            Esto incluye el uso de herramientas de automatización, técnicas de gestión deproyectos
            y enfoques ágiles que permiten adaptarse a cambios de requisitos durante el desarrollo.
    \end{enumerate}

    \subsection{ Áreas del Conocimiento Según la ``Guía del cuerpo de conocimiento de la ingeniería
    de software'' }

    \begin{enumerate}
        \item \textbf{Requisitos de software}: Se ocupa de la obtención, análisis, especificación
            y validación de los requisitos así como la gestión de requerimientos durante la
            vida entera del producto de software.

        \item \textbf{Diseño de software}: Es definido como el proceso de definición de la
            arquitectura, componentes, interfaces y otras características de un sistema o componente
            y el resultado de ese proceso. Esta área también abarca la gestión de la
            complejidad y la evolución del diseño.

        \item \textbf{Construcción de software}: Se centra en el proceso de codificación,
            pruebas unitarias y depuración del software. Abarca prácticas de programación,
            estándares de codificación, y la aplicación de herramientas y técnicas para la
            implementación eficiente de software.

        \item \textbf{Pruebas de software}: Consiste en la verificación dinámica de que un
            programa proporciona los comportamientos esperados de un conjuto de casos de prueba,
            convenientemente de un dominio de ejecución, normalmente infinito.

        \item \textbf{Mantenimiento de software}: Es una parte integral de ciclo de vida de
            un software. Este es definido como el total de actividades requeridas para
            proveer soporte costo-efectivo a un software.

        \item \textbf{Gestión de la configuración de software}: La configuración de un sistema
            son las características funcionales y físicas del hardware o software tal y como
            en la documentación técnica. La gestión de la configuración, entonces, es la
            disciplina de identificar la configuración de un sistema en distintos tiempos
            con el propósito de controlar cambios sistemáticamente a la configuración y
            mantener la integridad y trazabilidad de la configuración a lo largo del ciclo
            de vida del sistema.

        \item \textbf{Gestión de la ingeniería de software}: Puede ser definido como la aplicación
            de la gestión de actividades-planificando, coordinando, midiendo, monitoreando,
            controlando y reportando-para asegurarse de que los productos de software y servicios
            de ingeniería de software son eficientemente, efectivamente, y al beneficio de
            las partes interesadas.

        \item \textbf{Proceso de ingeniería de software}: Consiste de un conjunto de actividades
            interrelacionadas que traansofrman una o más entradas en salidas mientras se consumen
            recursos para alcanzar la transformación.

        \item \textbf{Modelos y métodos de ingeniería de software}: Imponen estructura en la
            ingeniería de software con el objetivo de hacer esa actividad sistemática repetible,
            y en última instancia más orientada al éxito

        \item \textbf{Calidad de software}: Puede referirse a características deseadas de productos
            de software, en la medida en el cual un producto de software en particular
            posee dichas características, y los procesos, herramientas y técnicas usadas
            para alzanzar esas características. Recientemente la calidad de software es definida
            como la "capacidad de un producto" de software de producir para satisfacer necesidades
            establecidas e implicadas bajo ciertas condiciones especificadas.

        \item \textbf{Práctica profesional de ingeniería de software}: A esta área le preocupa
            el conocimiento, habilidades, y actitudes que los ingenieros de software deben
            poseer para practicaar ingeniería de software en un manera profesional,
            responsable y ética.

        \item \textbf{Ingenería de software económica}: Consiste en tomar decisiones relacionadas
            a la ingeniería de software en un contexto de negocios. El éxito de un
            servicio, y solucióon depende en buen gestión de negocios. Economics es el
            estudio del valor, costos, recursos , y sus relaciones en un contexto dado o
            situación. En la disciplina de ingeniería de software, las actividades tienen
            costos, pero el mismo software resultante también tiene atributos económicos.

        \item \textbf{Fundamentos informáticos}: El núcleo de la ingeniería de software es
            entend la ciencia de la computación. Ingeniería de software le concierne la
            aplicación de computadoras, informática fines prácticos. La mayoría de los
            temas discutods en la Fundamentos de informática son discutods en cursos
            básicos de ciencias de computación. Algunos cursos incluyen programación,
            estructura de datos, algorimos, organización de computadoras, sistemas operativos,
            bases de datos, redes, sistemas distribuidos y mucho más.

        \item \textbf{Fundamentos matemáticos}: Esta área ayuda ingenieros de software a comprender
            la lógica requerida para el desarrollo de un programa, la cual es traducida en
            código de lenguaje de programación. La lógica y el razonamiento son la esencia
            de las matemáticas que un ingeniero de sofware debe apuntar.

        \item \textbf{Fundamentos de ingeniería}: IEEE define ingeniería como "la
            aplicación de un enfoque sitemático, disciplinado, cuantificable" a estructuras,
            máquinas, productos, sistemas o procesos". Esta área describe algunos de las
            habilidades y técnicas básicas que son útiles pasa un ingeniero de software.
    \end{enumerate}

    \subsection{ Desafíos y Oportunidades que Enfrenta la Ingeniería de Software en el Siglo
    XXI }

    \subsubsection{Desafíos}
    \begin{itemize}
        \item \textbf{Ciberseguridad}: Asegurar el software contra ataques y vulnerabilidades
            es cada vez más difícil con la creciente conectividad.

        \item \textbf{Escalabilidad y Complejidad}: Los sistemas son más grandes y complejos,
            lo que requiere mejores diseños y arquitecturas.

        \item \textbf{Desarrollo Ágil y DevOps}: Mantener la calidad y consistencia en entornos
            de desarrollo rápidos y distribuidos es un reto.

        \item \textbf{Inteligencia Artificial y Machine Learning}: Integrar IA de manera ética
            y transparente es crucial.

        \item \textbf{Privacidad y Regulaciones}: Cumplir con leyes de privacidad mientras
            se aprovechan los datos es un desafío constante.

        \item \textbf{Sostenibilidad}: Desarrollar software que minimice el consumo de energía
            es fundamental para reducir el impacto ambiental.
    \end{itemize}

    \subsubsection{Oportunidades}

    \begin{itemize}
        \item \textbf{Tecnologías Emergentes}: Liderar en áreas como IA, blockchain y
            realidad aumentada puede transformar industrias.

        \item \textbf{Automatización}: Herramientas avanzadas están mejorando la eficiencia
            y calidad del desarrollo de software.

        \item \textbf{Sostenibilidad}: Innovar en software ecológico y eficiente ofrece un
            camino hacia un futuro más sostenible.

        \item \textbf{Globalización y Trabajo Remoto}: La colaboración global permite acceder
            a un talento diverso y especializado.

        \item \textbf{Educación Continua}: Hay un gran potencial en desarrollar programas de
            capacitación para preparar mejor a los profesionales.

        \item \textbf{Inclusión y Diversidad}: Promover equipos diversos puede llevar a soluciones
            más creativas y efectivas.
    \end{itemize}

    \section{Ingeniería}

    \subsection{ Cómo Influye la Sostenabilidad en el Diseño y Desarrollo de Proyectos de Ingeniería
    }

    La sostenibilidad juega un papel cada vez más importante en el diseño y desarrollo de
    proyectos de ingeniería. Su influencia se manifiesta en varios aspectos clave:

    \begin{itemize}
        \item \textbf{Selección de Materiales}: Se eligen materiales reciclables y de bajo
            impacto ambiental.

        \item \textbf{Eficiencia Energética}: Se diseñan sistemas que minimizan el consumo
            de energía y se integran fuentes de energía renovable.

        \item \textbf{Impacto Ambiental}: Se realizan análisis del ciclo de vida para reducir
            el impacto ambiental y minimizar residuos.

        \item \textbf{Desarrollo Sostenible}: Se promueve la durabilidad y la economía circular
            en los proyectos.

        \item \textbf{Responsabilidad Social}: Se considera el impacto en las comunidades y
            se cumplen normativas de sostenibilidad.

        \item \textbf{Innovación Tecnológica}: Se usan herramientas digitales y se invierte
            en I+D para mejorar la sostenibilidad.

        \item \textbf{Economía}: A pesar de los costos iniciales, los proyectos
            sostenibles suelen ser más rentables a largo plazo y pueden recibir incentivos
            financieros.
    \end{itemize}

    \subsection{Importancia de la Ética en la Ingeniería}

    La ética en la ingeniería es crucial porque garantiza que los proyectos diseñados y construidos
    sean seguros, responsables y justos. Los ingenieros tienen la responsabilidad de proteger
    la seguridad pública, cumplir con normas y regulaciones, y tomar decisiones que beneficien
    a la sociedad en su conjunto. La ética también impulsa el uso responsable de los
    recursos y la protección del medio ambiente, promoviendo la sostenibilidad y asegurando
    que los proyectos no perjudiquen a las futuras generaciones.

    Además, la ética mantiene la integridad profesional, fomenta la transparencia y evita conflictos
    de interés, lo que es esencial para mantener la confianza pública en la ingeniería. A través
    de la ética, los ingenieros no solo aseguran la calidad técnica de sus trabajos, sino que
    también actúan de manera que respalden la justicia social y el desarrollo sostenible,
    asegurando que su trabajo tenga un impacto positivo y duradero en la sociedad y el
    medio ambiente.

    %% References

    \nocite{*} % to include uncited references of .bib file

    \clearpage
    \bibliographystyle{apalike}

    % Generated from .bib file
    \bibliography{ref}
\end{document}