\documentclass[11pt, a4paper]{article} % Formato

% Language and font encodings
\usepackage[spanish]{babel}
\usepackage[utf8]{inputenc}
\usepackage[T1]{fontenc}
\usepackage{times} % Times New Roman

%% Sets page size and margins
\usepackage[margin=2.5cm, includefoot]{geometry}
%\setlength{\columnsep}{0.17in} % page columns separation

%% Useful packages
\usepackage{amsmath}
\usepackage{array} % <-- add this line for m{} column type
\usepackage[hidelinks]{hyperref} % hyperlinks support
\usepackage{graphicx} % images support
%\usepackage{listings} % codeblock support
%\usepackage{smartdiagram} % diagrams support
\usepackage[most]{tcolorbox} % callouts support
%\usepackage[colorinlistoftodos]{todonotes}
\usepackage[dvipsnames, table, xcdraw]{xcolor} % Tables support
%\usepackage{zed-csp} % cchemas support

%% Formating
\usepackage{authblk} % to add authors in maketitle
%\usepackage{blindtext} % to gen filler text
\usepackage[figurename=Fig.]{caption} % to change prefix of the image caption

%\usepackage{apacite}
\usepackage{cite} % useful to compress multiple quotations into a single entry
\usepackage{enumitem}
\usepackage{fancyhdr} % to set page style
\usepackage{indentfirst}
%\usepackage{natbib}
\usepackage{parskip} % remove first line tabulation
\usepackage{setspace}
%\usepackage{titlesec}
%\usepackage{titling} % to config maketitle

%% Variables
% Main images
\newcommand{\logoUdg}{logo-udg.jpg}
\newcommand{\logoCucei}{logo-cucei.jpg}
\newcommand{\newAmazonSection}{./img/productos-nuevos.jpg}

% School data
\newcommand{\universidad}{Universidad de Guadalajara}
\newcommand{\cede}{Centro Universitario de Ciencias Exactas e Ingenierías}

% Subject data
\newcommand{\materia}{Interacción Humano Computadora}
\newcommand{\carrera}{Ingeniería en Computación}
\newcommand{\division}{División de Tecnologías para la Integración CiberHumana}
\newcommand{\theTitle}{3. Comprender la Toma de Deciciones Humanas}
\newcommand{\profesor}{José Luis David Bonilla Carranza}
\newcommand{\seccion}{D01}
\newcommand{\nrc}{209754}
\newcommand{\clave}{IL367}
\newcommand{\startDate}{20 de septiembre de 2024}

% Author data
\newcommand{\theAuthor}{Juárez Rubio Alan Yahir}
\newcommand{\theAuthorCode}{218517809}
\newcommand{\theAuthorMail}{alan.juarez5178@alumnos.udg.mx}

%% Declaration
\date{}
\graphicspath{ {../../../img/} }
\addto\captionsspanish{\renewcommand{\contentsname}{Índice}}
\renewcommand{\lstlistingname}{Código} % to change prefix of the code caption
\renewcommand{\lstlistlistingname}{Índice de códigos} % to change listings index title

%% Styles

% Color declaration
\definecolor{greenPortada}{HTML}{69A84F}
\definecolor{LightGray}{gray}{0.9}
\definecolor{codegreen}{rgb}{0, 0.6, 0}
\definecolor{codegray}{rgb}{0.5, 0.5, 0.5}
\definecolor{codepurple}{rgb}{0.58, 0, 0.82}
\definecolor{backcolour}{rgb}{0.95, 0.95, 0.92}

% Hyperlinks
\hypersetup{
    colorlinks=true,
    linkcolor=black,
    filecolor=greenPortada,
    urlcolor=greenPortada,
    pdfpagemode=FullScreen,
}

\urlstyle{same}

% Codeblocks
\lstdefinestyle{mystyle}{
	backgroundcolor=\color{backcolour},
	commentstyle=\color{codegreen},
	keywordstyle=\color{magenta},
	numberstyle=\tiny\color{codegray},
	stringstyle=\color{codepurple},
	basicstyle=\ttfamily\footnotesize,
	breakatwhitespace=false,
	breaklines=true,
	captionpos=b,
	keepspaces=true,
	numbers=left,
	numbersep=5pt,
	showspaces=false,
	showstringspaces=false,
	showtabs=false,
	tabsize=2
}

% Tables
\let\oldtabular\tabular
\renewcommand{\tabular}{\small\oldtabular}
\renewcommand{\arraystretch}{1.2} % <-- Adjust vertical spacing
\addto\captionsspanish{\renewcommand{\tablename}{Tabla}}

\lstset{style=mystyle}

%% Spacing
\newcommand{\nl}{\par
\vspace{0.4cm}}
\renewcommand{\baselinestretch}{1.5} % Espaciado de línea anterior
\setlength{\parskip}{6pt} % Espaciado de línea anterior
\setlength{\parindent}{0pt} % Sangría

% Header and footer
\pagestyle{fancy}
\fancyhf{}
\renewcommand{\headrulewidth}{3pt}
\renewcommand{\headrule}{\hbox to\headwidth{\color{greenPortada}\leaders\hrule height \headrulewidth\hfill}}
\setlength{\headheight}{50pt} % Ajuste necesario para evitar warnings

% Header
\pagestyle{fancy}
\fancyhf{}
\lhead{
	\begin{minipage}[c][2cm][c]{1.3cm}
		\begin{flushleft}
			\includegraphics[width=5cm, height=1.4cm, keepaspectratio]{\logoUdg}
		\end{flushleft}
	\end{minipage}
	\begin{minipage}[c][2cm][c]{0.5\textwidth} % Adjust the height as needed
		\begin{flushleft}
			{\materia}
		\end{flushleft}
	\end{minipage}
}

\rhead{
	\begin{minipage}[c][2cm][c]{0.4\textwidth} % Adjust the height as needed
		\begin{flushright}
			{\theTitle}
		\end{flushright}
	\end{minipage}
	\begin{minipage}[c][2cm][c]{1.3cm}
		\begin{flushright}
			\includegraphics[width=5cm, height=1.4cm, keepaspectratio]{\logoCucei}
		\end{flushright}
	\end{minipage}
}

% Footer
\fancyfoot{}
\lfoot{\small\materia}
\cfoot{\thepage} % Paginación
\rfoot{\small Curso impartido por \profesor}

%% Title

\title{\fontsize{24}{28.8}\selectfont \theTitle}
\author{\theAuthor}

\affil{}


\begin{document}
    \setstretch{1} % Interlineado

    \begin{titlepage}
        \newgeometry{margin=2.5cm, left=3cm, right=3cm} % change margin
        \centering
        %\vspace*{-2cm}
        {\huge\textbf{\universidad}}\par
        \vspace{0.6cm}
        {\LARGE{\cede}}
        \vfill

        \begin{figure}[h]
            \begin{minipage}[t]{0.45\textwidth}
                \centering
                \includegraphics[width=130px, height=160px, keepaspectratio]{\logoUdg}
            \end{minipage}
            \hfill
            \begin{minipage}[t]{0.45\textwidth}
                \centering
                \includegraphics[width=130px, height=160px, keepaspectratio]{\logoCucei}
            \end{minipage}
        \end{figure}
        \vfill

        \Large{ \division\vfill \textbf{\carrera}\vfill \textbf{\materia}\par\vspace{3pt} \seccion\ - \clave\ - \nrc\vfill }

        {\LARGE{\textbf{\theTitle}}}
        \vfill

	\begin{figure}[h]
		\centering
		\begin{minipage}[t]{0.75\textwidth}
			{\Large
				\textbf{Integrantes:}\par\vspace{8pt}
				\begin{itemize}
					\item \theAuthor\ - \theAuthorCode
					\item \bAuthor\ - \bAuthorCode
					\item \cAuthor\ - \cAuthorCode
					\item \dAuthor\ - \dAuthorCode
					\item \eAuthor\ - \eAuthorCode
				\end{itemize}
			}
		\end{minipage}
	\end{figure}\vfill


        \begin{tcolorbox}
            [colback=red!5!white, colframe=red!75!black]
            \centering
            Este documento contiene información sensible.\\
			No debería ser impreso o compartido con terceras entidades.
        \end{tcolorbox}
        \vfill
        {\large \startDate}\par
    \end{titlepage}

    \restoregeometry % end changed margin

    %% Indexes
    \clearpage
    \tableofcontents

    \clearpage
    \listoffigures

    \clearpage
    %\listoftables

    %\clearpage
    %\lstlistoflistings

    %% Main Title
    %% Main Title
    \clearpage
    \vspace*{6pt}
	\centerline{\textbf{\huge \theTitle}}
    \vspace*{8pt}

    %% Content

	\section{Recolección de Datos}

	\subsection{Entorno}

	\begin{itemize}
		\item ¿En qué Sistemas Operativos se piensa utilizar el programa?

			El programa será desarrollado de tal manera que pueda ser utilizado en sistemas
			operativos Windows, Linux y MacOs.

		\item ¿En qué dispositivos se piensa utilizar el programa?

			El programa se utilizará con cualquier dispositivo a través de un navegador web,
			dispositivos tales como celulares, tablets y computadoras.

		\item ¿Qué tipo de aplicación se desea desarrollar (e.g.~de escritorio, móvil,
			web)?

			El tipo de aplicación que se desarrollará será una aplicación web.
	\end{itemize}

	\subsection{Usuarios}

	\begin{itemize}
		\item ¿A quiénes va dirigido el programa?

			Los usuarios a los que va dirigido el programa específicamente son para alumnos
			de CUCEI que cuenten con la problemática de encontrar un proyecto modular,
			llevar a cabo su proyecto modular y bucar integrantes de equipo.

		\item ¿Cuáles son los tipos de usuario dentro del sistema y cuáles son sus principales
			funciones dentro del sistema?

			Los tipos de usuario que existirán dentro del sistema serán:

			\begin{itemize}
				\item Estudiantes

					\begin{itemize}
						\item Líderes de equipo: Creará el proyecto, reclutará a nuevos integrantes,
							asignar tareas.

						\item Integrantes del equipo: Unirse a un proyecto y gestionar cada
							una de sus tareas.
					\end{itemize}

				\item Asesor: Visualizar el progreso del proyecto y convocar reuniones con
					el equipo.
			\end{itemize}

		\item ¿Qué datos necesitará el usuario para registrarse en la plataforma?

			\begin{itemize}
				\item Nombre

				\item Código de estudiante

				\item Correo institucional: Validación

				\item Contacto

				\item Carrera

				\item Foto

				\item Clave
			\end{itemize}

		\item ¿Cuáles datos del usuario serán públicos y cuáles privados?

			Todos los datos serán públicos, a excepción de los datos de inicio de sesión

		\item ¿Cómo se relacionan cada uno de los tipos de usuario entre sí?

			Se espera que el líder del proyecto pueda tener control de los usuarios partícipes
			en su proyecto, que pueda visualizar las descripciones de cada usuario y
			pueda admitir solicitudes de usuarios; por otro lado los usuarios podrán visualizar
			los proyectos existentes en la plataforma, además de los integrantes, líder y
			asesor que lo conforman; por último el asesor podrá visualizar los proyectos
			(esto incluye participantes, líderes, descripción de proyecto, etc).

		\item ¿Qué datos se mostrarán en la tarjeta de presentación del usuario?

			\begin{itemize}
				\item Nombre

				\item Código de estudiante

				\item Correo

				\item Semestre

				\item Habilidades técnicas y blandas

				\item Promedio

				\item Status (con equipo o sin equipo)
			\end{itemize}
	\end{itemize}

	\subsection{Datos del Proyecto}

	\begin{itemize}
		\item ¿Qué datos se deben proporcionar obligatoria y opcionalmente acerca del proyecto?

			Los datos de carácter obligatorio para el registro del proyecto serán:

			\begin{itemize}
				\item Nombre

				\item Descripción del proyecto

				\item Áreas de desarrollo

				\item Cupos

				\item Estado

				\item Conocimientos requeridos

				\item Nivel de innovación
			\end{itemize}

			Los datos de carácter opcional para el registro del proyecto serán:

			\begin{itemize}
				\item Asesor

				\item Logo personalizado
			\end{itemize}

		\item ¿Cuáles datos de un proyecto deben ser públicos y cuáles privados?

			Se espera que sea de carácter privado la parte de la documentación,
			planteamiento o la descripción completa del proyecto, que se espera que esta
			pueda ser visible una vez y el usuario forme parte del equipo.

		\item ¿El usuario puede optar por cuáles datos de su proyecto serán públicos y
			cuáles privados?

			No, el sistema se encargará de mostrar los datos privados del proyecto
			solamente a los integrantes del equipo.

		\item ¿Para acceder a la plataforma será obligatorio acceder con una cuenta?

			Sí, el usuario necesitará tener una cuenta para poder acceder, esto con la finalidad
			de que pueda visualizar el estatus de cada proyecto en tiempo real y en todo
			caso la información de los distintos proyectos existentes.

		\item ¿Para crear un proyecto será necesario contar con asesor?

			No, puesto que se espera que el equipo pueda subir su propuesta de ``proyecto
			modular'' y este pueda tener la opción de que el asesor escoja participar en
			el proyecto o incluso que el equipo pueda conseguir un asesor a lo largo del
			desarrollo.
	\end{itemize}

	\subsection{Búsqueda}

	\begin{itemize}
		\item ¿Cómo se ordenarán los proyectos por defecto?

			Los proyectos tendrán algunos filtros, ya que los que ya estén completos los
			usuarios del equipo, se pasarán hasta abajo que ya no aparezca, se mostraran
			los que tengan menos usuarios o por orden alfabético, basándonos en el nombre
			del proyecto

		\item ¿Qué datos se mostrarán en la tarjeta del proyecto?

			Se mostrará el nombre del proyecto, junto con cuántos integrantes están ya
			dentro del proyecto, innovación, que carreras lo pueden conformar y una
			breve descripción del propio proyecto para que el usuario vea si le agrada
			el proyecto.

		\item ¿Qué filtros de búsqueda se podrán aplicar?

			Algunos de los filtros que se podrían aplicar son: tecnologías, carreras a las
			que aplica, cantidad de integrantes restantes, estado del proyecto

		\item ¿Un líder de equipo tendrá la posibilidad de buscar dentro de la plataforma
			a integrantes del equipo?

			Es una función que se busca implementar, sin embargo, si este provoca que el
			proyecto sufra un retraso, se optará por no implementarlo
	\end{itemize}

	\subsection{Inicio de Sesión}

	\begin{itemize}
		\item ¿Cómo se maneja la recuperación de la contraseña?

			El proceso de recuperación será mediante el correo de la persona, al cual se
			le hará llegar un correo con las instrucciones a seguir para la recuperación
			de la cuenta

		\item ¿La autenticación de 2 pasos será obligatoria?

			La autenticación en dos pasos no será estrictamente obligatoria, puesto si
			el usuario quiere reforzar su seguridad puede habilitar esta opción.
	\end{itemize}

	\clearpage
	\section{Clasificación y Priorización de los Requerimientos del Sistema}

	\subsection{Requerimientos Funcionales}

	\subsubsection{Registro de Usuarios}

	\begin{enumerate}
		\item El sistema permitirá a los alumnos registrarse mediante un formulario en
			el que les solicitará datos tales como nombre, código de estudiante, correo electrónico,
			número de contacto, carrera, contraseña. Adicionalmente tendrán la opción de
			añadir una foto de perfil

		\item El sistema permitirá a los asesores registrarse mediante un formulario el
			cual les solicitará datos como nombre, código de profesor, correo electrónico,
			departamento y número de contacto. Adicionalmente tendrán la opción de añadir
			una foto de perfil

		\item El formulario de registro incluirá campos opcionales como área de especialización,
			experiencia previa, y una breve biografía para hacer coincidir mejor las solicitudes
			con las necesidades del proyecto.
	\end{enumerate}

	\subsubsection{Inicio de Sesión}

	\begin{enumerate}
		\item Los usuarios registrados podrán acceder a sus cuentas mediante un sistema
			de inicio de sesión seguro.

		\item Se incluirá una opción para la recuperación de contraseñas a través del correo
			electrónico en caso de que el usuario olvide sus credenciales.

		\item Los usuarios pueden optar por configurar su inicio de sesión con autenticación
			de 2 pasos, esto con el fin de garantizar una mayor protección de la cuenta.
	\end{enumerate}

	\subsubsection{Búsqueda de Proyectos}

	\begin{enumerate}
		\item El sistema permitirá a los usuarios realizar búsquedas de proyectos modulares
			disponibles utilizando filtros como palabras clave, categoría, áreas de interés
			o habilidades requeridas.

		\item Los resultados de la búsqueda mostrarán detalles como el nombre del proyecto,
			una descripción general, requisitos técnicos o de experiencia, el número de
			vacantes disponibles, y el estatus actual del proyecto (abierto o cerrado a
			nuevas solicitudes).

		\item Los proyectos destacados o recomendados según los intereses del usuario también
			serán visibles en la interfaz de búsqueda.
	\end{enumerate}

	\subsubsection{Creación e Integración de Proyectos}

	\begin{enumerate}
		\item Un alumno solamente podrá pertenecer a un único proyecto a la vez.

		\item Los alumnos tendrán la posibilidad de crear un proyecto y, por ende, obtendrán
			el rol de líder dentro del proyecto, permitiéndoles administrar el proyecto
			y los integrantes del equipo.

		\item Los alumnos tendrán la posibilidad de integrarse a un equipo a través del
			uso de las solicitudes de admisión y la bandeja de notificaciones.
	\end{enumerate}

	\subsubsection{Gestión de Actividades}

	\begin{enumerate}
		\item Los líderes de proyecto tendrán acceso a un panel de administración donde
			podrán gestionar todas las actividades relacionadas con el proyecto.

		\item Asignación de tareas específicas a los integrantes del equipo y el seguimiento
			del progreso de dichas tareas.

		\item Los integrantes del equipo tendrán la opción de gestionar cada una sus tareas
			asignadas a través de un canvas, en el que les permitirá catalogarlas como
			pendientes, en proceso y finalizadas.
	\end{enumerate}

	\subsubsection{Bandeja de Solicitudes}

	\begin{enumerate}
		\item Los líderes de proyecto podrán gestionar las solicitudes de los usuarios
			interesados en unirse a sus proyectos.

		\item Las solicitudes recibidas serán visualizadas en una bandeja dedicada,
			donde el líder podrá revisar la información del solicitante (perfil,
			experiencia, intereses) y aceptar o rechazar la solicitud.

		\item Los líderes tendrán la opción de eliminar a un integrante del equipo y,
			a su vez, el integrante del equipo tendrá la posibilidad de abandonar el equipo.

		\item Se incluirá un historial de decisiones para un mejor seguimiento.

			Adicionalmente, el líder tendrá la opción de proporcionar retroalimentación
			al solicitante en caso de rechazo.
	\end{enumerate}

	\subsubsection{Agenda de Reuniones}

	\begin{enumerate}
		\item Los integrantes recibirán notificaciones automáticas sobre la creación,
			cancelación o modificación de reuniones.

		\item El asesor podrá programar reuniones, las cuales podrán ser aceptadas o, en
			su defecto, rechazadas por los alumnos debido al no serles posible asistir a
			dicha reunión. En caso de ser rechazada una solicitud el asesor podrá proponer
			otra reunión en un horario diferente.
	\end{enumerate}

	\subsection{Requerimientos No Funcionales}

	\subsubsection{Interfaz de Usuario (UI)}

	\begin{enumerate}
		\item La interfaz debe ser intuitiva y estéticamente agradable, facilitando la
			navegación a los usuarios sin necesidad de conocimientos técnicos avanzados.

		\item Se utilizarán estándares de diseño como Material Design, con el fin de
			asegurar coherencia visual y usabilidad.

		\item El diseño debe adaptarse a diferentes tamaños de pantalla, garantizando
			una experiencia fluida tanto en dispositivos móviles como en ordenadores de
			escritorio.
	\end{enumerate}

	\subsubsection{Temas y Colores}

	\begin{enumerate}
		\item Se busca que la paleta de colores sea mayormente colores neutros que ayudan
			una mejor importancia a los debidos elementos además de un apoyo a la visualización
			del usuario.
	\end{enumerate}

	\subsubsection{Íconos y Botones}

	\begin{enumerate}
		\item Se implementarán íconos fácilmente reconocibles para las funciones principales,
			acompañados de botones claros y consistentes que ofrezcan retroalimentación
			visual (como cambios de color o animaciones sutiles) para indicar la
			interactividad.

		\item Cada ícono y botón debe estar alineado con las convenciones de diseño,
			reduciendo la curva de aprendizaje para los nuevos usuarios.
	\end{enumerate}

	\subsubsection{Privacidad y Seguridad de Datos}

	\begin{enumerate}
		\item Privacidad: El sistema debe cumplir con las regulaciones de privacidad
			de datos, asegurando que la información personal de los usuarios y los proyectos
			se almacene y procese de manera confidencial.

		\item Seguridad: Todos los datos sensibles, como contraseñas y detalles personales,
			se almacenarán en bases de datos encriptadas. Las transferencias de datos entre
			el cliente y el servidor deben realizarse bajo protocolos seguros como HTTPS.
	\end{enumerate}

	\subsubsection{Rendimiento}

	\begin{enumerate}
		\item El sistema debe ser eficiente, respondiendo a las interacciones del
			usuario de manera rápida.

		\item Los tiempos de carga de las páginas y funciones críticas no deben superar
			los 5 segundos bajo condiciones normales de operación.
	\end{enumerate}

	\subsubsection{Almacenamiento de Datos}

	\begin{enumerate}
		\item El sistema utilizará una base de datos relacional o no relacional, según las necesidades del proyecto, que ofrezca redundancia y respaldo automático de datos.

		\item Se garantizará la integridad de los datos mediante la implementación de técnicas de rollback y copias de seguridad periódicas, para evitar la pérdida de información
	\end{enumerate}

	\clearpage
	\section{Priorización de Requerimientos}

	\begin{figure}[h]
		\centering
		\includegraphics[width=\textwidth]{\moscow}
		\caption{Matriz de Priorización de MoSCoW}
	\end{figure}

\end{document}
